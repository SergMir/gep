\section{Статья 4: Применение программирования с экспрессией генов}

Построение модели объекта на основе некоторых данных о нём~--- одна из основных задач, для решения которых применяется ПЭГ. На вход алгоритму ПЭГ подаются данные в формате, аналогичном входным данным при обучении ИНС: набор кортежей, содержащих параметры объекта, численные оценки воздействий на него, и количественные характеристики состояния объекта. Цель алгоритма ПЭГ~--- обнаружить явные и скрытые взаимосвязи между входными значениями~--- параметрами объекта и воздействиями на него,~--- и выходными~--- реакцией объекта и его состоянием. По окончании работы алгоритм ПЭГ выдаёт модель объекта, в зависимости от желаемого набора функций представленную в виде математической формулы (символьная регрессия), компьютерной программы, дерева принятия решений, правила классификации, числового вектора (вырождение ПЭГ в ГА в случае пустого функционального множества), и т.п.

Полученные модели применяются в различных целях. Выявленные зависимости могут помочь при дальнейшем анализе объекта, например, для автоматического открытия законов. В большинстве случаев модель отлично подходит для аппроксимации объекта, в ряде случаев~--- и для прогнозирования.

При построении модели, как и при обучении ИНС, набор известных об объекте данных разбивается на выборки: обучающую и тестовую. Возможен вариант деления на три выборки, в таком случае к указанным двум добавляется валидационная. Обучающая выборка используется для оценки фитнеса особей популяции и дальнейшего отбора. Тестовая выборка используется для определения лучшей особи популяции в последнем либо каждом поколении. Валидационная выборка используется для финальной оценки пригодности модели, полученной алгоритмом ПЭГ.

\subsection{Аппроксимация и символьная регрессия}

Среди применений ПЭГ известно использование алгоритма для моделирования толщины слоёв дорожного покрытия~\cite{Terzi:2005:JAS, saltan:2005:IJEMS}. Данная задача поставлена необходимостью оценки состояния, структурной целостности и ресурса покрытия для определения как целесообразности проведения восстановительных работ, так и вида этих работ.

Слои дорожного покрытия характеризуются модулями упругости, которые на практике принято оценивать исходя из степени деформации, проводя неразрушающее тестирование дефлектометром. Методика получения данных: удар при падении груза известной массы с различной высоты создаёт волну, длина которой измеряется сенсорами. На вход модели подаются данные сенсоров и высота, с которой сбрасывается груз; на выход модели поступает толщина верхнего слоя дорожного покрытия.

В работе~\cite{saltan:2005:IJEMS} было проведено сравнение моделей, полученных при решении описанной задачи двумя недетерминированными эвристическими методами: обучение однослойной ИНС и получение математической формулы с помощью ПЭГ. Оба метода показали способность решать задачу, используя набор данных без априорных знаний. Однако обученная ИНС показала более точные результаты (коэффициент корреляции $R^2=0.9995$), чем формула, выданная ПЭГ ($R^2=0.7963$). В то же время полученная простая математическая формула более удобна в обращении, чем ИНС, количество структурных элементов которой равно произведению количества входов, нейронов входного слоя, выходов, функций суммирования и функций активации.

ПЭГ также нашло применение в задаче оценки испарений, основываясь на данных метеорологической станции возле озера Эгирдир, таких как температура воздуха, влажность и количество солнечного излучения. Точность полученной модели (простой формулы) в~\cite{OzlemTerzi:2005:JAS} оказалась выше, чем у общепринятого в отрасли метода Пенмана-Монтейта.

В работе~\cite{Baykasoglu:2005:ICRM} успешно проведено моделирование производственной линии с использованием GEP и многоцелевого поиска табу.

При разработке цифровых сигнальных процессоров широко используются фильтры с конечной импульсной характеристикой (FIR-фильтры~--- finite impulse response). Преимущества таких фильтров~--- в их устойчивости, нерекурсивности (отсутствии обратных связей), возможности реализации с линейной фазой. Один из параметров, указываемых перед разработкой нового фильтра~--- его порядок. Поиск формул для оценки минимального порядка фильтров с линейной фазой с помощью ПЭГ описан в работе~\cite{GonzalezMunoz:2005:RVK}. Анализ полученных моделей показал, что их точность не уступает другим методам оценки порядка, приведённым в статье.

Аналитическое моделирование компонентов турбомашин проводится при разработке новых устройств и изучении внутренних состояний и потоков на протяжении более чем столетия. Для построения моделей на основе больших баз данных в отрасли помимо линейной регрессии применяются как ИНС, так и алгоритмы генетического программирования, в частности ПЭГ~\cite{MECE2005-79414R3, IMECE2005-79416-R3}. Однако системы, полученные с помощью ИНС, хоть и могут обладать большой предсказательной силой, являются лишь упрощением, в то время как выходная формула ПЭГ может представлять собой фундаментальную модель (при условии подачи независимых аргументов). Стоит отметить, что зачастую полученные с помощью ПЭГ модели имеют слишком большую сложность и требуют дальнейшего упрощения с участием человека.

В работе~\cite{al2010new} предложено использовать ПЭГ для моделирования механических конструкций, применяемых в строительстве, в критических условиях. Механические свойства стали стремительно ухудшаются с ростом температуры (например, под воздействием огня): снижение жёсткости и допустимой нагрузки могут привести к крушению постройки. Особенно описанному явлению подвержены стыки и соединения. Для изучения влияния разрушающих факторов на полужёсткие соединения стальных структур активно применяется компьютерное моделирование с применением таких техник, как ИНС, ГА.

Для накопления обучающих данных, подаваемых на вход алгоритму ПЭГ, и данных для тестирования полученных моделей были проведены эксперименты по измерению характеристик четырёх различных стандартных соединений, применяемых в строительстве, под воздействием пламени с регулируемой температурой. На вход ПЭГ подаются обучающая выборка из 331 элемента, состоящая из 16 параметров: температура, механический момент, механическая прочность, количество болтов, геометрические свойства компонентов соединения.
Полученная модель представляет собой формулу, значение которой отражает поворот конструкции с высокой точностью~--- коэффициент корреляции составил 0.89~--- и которая поэтому может быть использована при проектировании строительных конструкций.


\subsection{Глубинный анализ данных, построение правил классификации}

Правила классификации, как правило, представляются в виде синтаксических деревьев, содержащих логические операторы (<<И>>, <<ИЛИ>>, <<НЕ>>, сравнение и пр.), либо операторы нечёткой логики. Деревья легко поддаются линеаризации (записи в геноме), потом чего к ним становится применим механизм ПЭГ. Полученные правила с лёгкостью записывается в виде небольшой функции на любом языке программирования, например, C.

При бинарной классификации в качестве функции фитнеса зачастую используется следующая формула:
$$
f = \frac{t_p}{t_p+f_n} \times \frac{t_n}{t_n+f_p},
$$
где $t_p$~--- количество верно распознанных положительных элементов данных, $t_n$~--- верно распознанных отрицательных элементов, $f_p$~--- неверно распознанных положительных элементов, $f_n$~--- неверно распознанных отрицательных элементов.

Проведённое в работах~\cite{P1120535113, P1120535114} сравнение производительности ПЭГ с системой классификации {C4.5} на тестовых наборах данных (ирисы Фишера и пр.) показало некоторое преимущество ПЭГ в точности: 65\% верно классифицированных данных {C4.5} и 71.51\% ПЭГ.

Сравнение различных систем классификации также было проведено в~\cite{conf/adma/WeinertL06}: ПЭГ (точность классификации составила 88.425\%), {C4.5} (точность~--- 82.125\%), CSGP (Constrained-Syntax GP~--- ГП с ограниченным синтаксисом, точность~--- 79.375\%) и BGP (Booleanized GP~--- ГП с набором функций, ограниченным до булевых <<И>>, <<ИЛИ>> и <<НЕ>>, точность~--- 85.55\%). Данные для эксперимента были взяты в репозитории машинного обучения Калифорнийского университета в Ирвайне: медицинская статистика состояний пациентов с диагнозом рака молочной железы, болями в груди, дерматологическими проблемами.

Из результатов эксперимента можно сделать вывод о практически равной точности указанных методов, однако система на основе ПЭГ требует меньшего количества вычислений (1500 поколений ПЭГ против 25000 поколений CSGP). Кроме того, ПЭГ позволяет устанавливать желаемую сложность правил, управляя количеством генов.

Аналогичный эксперимент по построению системы диагностирования рака молочной железы описан в~\cite{ferreira:2004:rdbic}. Данные архива PROBEN1 университета Карнеги-Меллон содержат 9 атрибутов. Алгоритмом было обнаружено правило бинарной классификации, корректно описывающиее все 100\% элементов обучающей выборки и 97.14\% тестовой.

При построении бинарного классификатора пациентов из базы данных Национального центра биотехнологической информации США~\cite{ncbinlm2006} на предмет предрасположенности к раку груди на основе 12625 атрибутов-генов нескольких тысяч человек авторы~\cite{conf/rskt/ValdesB06} столкнулись с необходимостью предварительной обработки данных с целью выделения необходимого подмножества атрибутов в многомерном наборе данных. После сокращения количества атрибутов путём кластеризации до десяти система на основе ПЭГ успешно справилась с построением классификатора, при этом было задействовано девять атрибутов. Анализ полученной формулы позволил легко определить наиболее существенные атрибуты: при переводе в полиномиальную форму два атрибута (из девяти) имели высшие степени. При дальнейшем анализе данных удалось выявить единственный атрибут, достаточный для проведения по нему классификации.

Задача другого типа рассматривается в~\cite{journals/iajit/EldrandalyN08}: разработка системы прогнозирования скачков давления в гидросистемах на основе ПЭГ. Такие системы применяются в целях обеспечения равномерного распределения энергии потоков воды в системе, поддержки уровня воды в оросительных каналах и т.п. На вход алгоритму ПЭГ подаётся большой набор обучающих данных, включающих в себя такие параметры, как плотность и динамическая вязкость жидкости, скорость потока, длина отрезка трубы и пр. В целях сравнения была построена модель процессов с использованием множественной регрессии, основанная на двух предположениях: линейное соотношение входных и выходных данных, и независимость входных переменных от выходных. Сравнение моделей проводилось на основании их коэффициентов корреляции $R^2$ и среднеквадратичного отклонения. Модель ПЭГ показала лучшие результаты ($R^2=0.914$ против $R^2=0.873$), однако её формула в то же время обладает б\'{о}льшей сложностью.

Исследованию террористических организаций~--- сетей, состоящих из небольших групп людей, связанных одной целью,~--- и тому, как отличить террористов от главных подозреваемых, посвящена работа~\cite{conf/wisi/QiaoTPFX06}. Обучение и тестирование классификаторов проводились на наборах данных, содержащих несколько сотен записей с такими атрибутами: религия, происхождение, пол, образование, возраст, судимости. Эффективность системы, построенной с помощью ПЭГ, сравнивалась с {C4.5}: при значительно меньшем времени выполнения получены формулы в несколько раз компактнее.

ПЭГ был применён также для построения классификатора, отделяющего сигнал от шума среди событий, зарегистрированных детекторами ускорителя элементарных частиц в Национальной ускорительной лаборатории SLAC в Стенфорде. В ходе работы~\cite{Teodorescu:2006:IEEETNS} было выяснено, что размер функционального множества не оказывает влияния на точность модели. В то же время меньшее количество функций как правило ведёт к б\'{о}льшему размеру синтаксического дерева (количеству узлов в нём), однако данный эффект не проявляется, если ввести в функцию фитнеса штраф за размер дерева.

Эксперименты с увеличением количества входных переменных показали, что алгоритм ПЭГ нечувствителен к их числу (при наличии должного количества релевантных) и способен игнорировать переменные, не представляющие ценности в задаче классификации.

Кроме ПЭГ, к данной задаче были также применены обучение ИНС и построение деревьев принятия решений методом градиентного добавления (BDT~--- Boosted Decision Trees). Разница в эффективности полученных моделей составила 1--3\%. В отличие от ИНС, ПЭГ не подвержен проблеме перетренированности. Скорость работы всех трёх предложенных алгоритмов можно считать равной равной, при б\'{о}льшей обобщающей способности (разнице между результатами на тренировочных и тестовых выборках) метода ПЭГ.

Помимо прочего, существующие системы поиска правил классификации, построенные на основе ГП, могут быть адаптированы под использование ПЭГ. В работе~\cite{conf/iwcls/Wilson07} рассмотрено изменение механизма эволюционирования системы XCSF с ГП на ПЭГ. Модифицированная система значительно превзошла в производительности исходную.

Наиболее часто используемый метод определения принадлежности единицы данных $X$ к определённому классу заключается в проверке условия $gep(X) - t > 0$, где $gep$~--- функция, обнаруженная алгоритмом ПЭГ, $t$~--- граница принадлежности к классу. Значение $t$ зачастую выбирается субъективно, эмпирически, что требует множества запусков алгоритма для определения оптимальной величины.

Для динамического определения границы классов t предлагается применить следующий метод разделения. Пусть $G = \{g_i, i=1, 2, ..., n\}$ будет множество значений, возвращённых gep - функцией, обнаруженной алгоритмом ПЭГ для определения принадлежности данных к классу $c$ - для всех $n$ элементов данных обучающей выборки.

Тогда среднее значение этих значений задаётся как $Mean_c = \frac{g_1 + g_2 + ... + g_n}{n}$.

Пусть $G' = \{g'_i, i=1, 2, ..., n\}$ будет отсортированный по возрастанию массив $G$, тогда значение медианы можно извлечь из срединного элемента: $Median_c = g'_{n/2 + 1}$.

Значение любой из указанных метрик может быть использовано в качестве значения $t$. В случае нескольких классов берётся среднее арифметическое значений метрик для каждого из них. Тестирование данного подхода на наборе данных ирисов Фишера показало его эффективность применительно к формированию правил классификации~\cite{conf/adma/DuanTZWZ06}.

Многоцелевая классификация представляет собой задачу поиска правил классификации, которые удовлетворяли бы сразу нескольким критериям одновременно, таким как точность правил, их понятность (выраженная в количестве вовлечённых в правило атрибутов). Из множественности критериев следует их конкуренция между собой и взаимная противоречивость. Свойства эволюционных алгоритмов привлекают исследователей возможностью получать группы правил классификации.

Системы классификации, основанные на ГА разделяются на две категории, в основе которых лежат два принципиально различных подхода: мичиганский и питтсбургский. Основое различие между ними заключается в схеме кодирования хромосомы.

В мичиганском подходе хромосома фиксированной длины кодирует одно правило классификации. Для поиска всех правил группы требуется либо отдельный запуск алгоритма для вывода каждого из них, что требует б\'{о}льших вычислительных затрат, либо расширение алгоритма путём кодирования особью всей группы правил.

В питтсбургском подходе каждая особь представляется строкой переменной длины и кодирует группу правил целиком. Питтсбургский подход показывает лучшие результаты в пакетом режиме, когда все образцы обучающей выборки доступны на протяжении всего процесса обучения, в то время как мичиганский более подходит для обучения онлайн, в котором домены переменных динамически изменяются.

Система классификации, разработанная в рамках работы~\cite{Dehuri:2008:MCR:1471604.1472090}, кодирует правила в хромосомах фиксированной длины, используя свойство ПЭГ хранить синтаксические деревья переменного размера. Каждая особь хранит одно правило. В функциональном наборе задействованы арифметические действия, правила записываются в двух формах: <<ЕСЛИ <выражение> > <число> ТО ПРИНАДЛЕЖИТ КЛАССУ>> и <<ЕСЛИ <число> < <выражение> < <число> ТО ПРИНАДЛЕЖИТ КЛАССУ>>. Результаты тестирования системы на распространённых наборах данных, таких как ирисы Фишера, показали высокую способность системы к поиску простых, удобных для дальнейшего анализа оператором правил классификации, выявляющих скрытые взаимосвязи в поданных данных.


\subsection{Сжатие изображений}

Первые попытки применения ПГ к задаче сжатия изображений путём их представления в виде математических функций отражены в работе~\cite{fukunaga1998evolving}. Спустя почти десять лет с развитием доступных вычислительных мощностей оказались возможными первые практические результаты, описанные в отчётах об исследованиях~\cite{techrep/Ashworth06, techrep/Gempeler06}.

Первый этап предложенного алгоритма~--- препроцессинг: представление изображения в форме, подходящей для дальнейшей подачи на вход эволюционному процессу. Для этого выполняются первые шаги спецификации сжатия изображений JPEG: разбиение исходного изображения на блоки 8x8 пикселей, дискретное косинусное преобразование каждого из них, умножение на матрицу квантования, округление до целых чисел, линеаризация путём обхода элементов полученной матрицы по зигзагу, отбрасывание элементов, следущих за последним ненулевым \cite{jpeg1993}. Кроме того, от каждого элемента полученного массива отнимается среднее значение массива, таким образом не требуется поиск константы вертикального смещения.

Второй этап алгоритма заключается в эволюционном поиске формул, описывающих каждый блок. На последнем этапе осуществляется кодирование полученных формул, записывая элементы в виде соответствующих им двоичных последовательностей.

При помощи представленного алгоритма удалось сжать тестовое изображение 256x256 пикселей в оттенках серого со среднеквадратичным отклонением 16.3/255 с коэффициентом сжатия 1.5 за несколько часов. Таким образом, средств исходного алгоритма ПЭГ, простых техник препроцессинга изображений и выбранного способа кодирования недостаточно для эффективного решения задачи сжатия.

Применение ПЭГ и других эволюционных технологий описано в авторской работе~\cite{SergMir_06_2013_imagecompress_all}.


\subsection{Использование ПЭГ при построении систем}

Построение систем определения допусков отклонений радиокомпонентов~--- одна из важнейших задач при построении аналоговых цепей, наряду c автоматическим выравниванием компонентов на печатной плате. Суть данной NP-сложой задачи состоит в поиске оптимальных допустимых отклонений таких параметров элементов электрической цепи, как сопротивление, ёмкость и индуктивность. Из-за отклонений этих параметров от их номинальных значений невозможно достичь удовлетворения всех устройств всем спецификациям. При решении задачи с помощью ПЭГ на вход алгоритму подаются векторы, содержащие допуски параметров и соответствующую производительность цепи. Формулы, полученные в ходе проведённых в рамках~\cite{DT_ICSES} экспериментов, были верифицированы при помощи метода Монте-Карло, достигнута приемлемая точность в 10\% при значительном превосходстве в скорости про сравнению с требующим больших вычислительных затрат методом Монте-Карло.

Одно из исследований ПЭГ было проведено на задаче определения большинства голосов при анонимном голосовании на выборах из двух кандидатов~\cite{conf/eurogp/Ferreira02}. В качестве подхода к решению был построен механизм клеточных автоматов.

Простейший клеточный автомат представляет собой кольцевой буфер из N клеток, каждая из которых соединена с соседями с обеих сторон. Двоичное состояние каждой клетки обновляется в соответствии с определённым правилом. Правило применяется одновременно ко всем клеткам, процесс повторяется на протяжении $t$ шагов.

Начальная конфигурация (состояние клеток) отображает голоса избирателей: 0 означает голос за одного кандидата, 1~--- за второго. Требуется обнаружить с помощью ПЭГ правило, переводящее состояние всех клеток в 0, если большинство проголосовало за первого кандидата (б\'{о}льшая плотность нулей в начальной конфигурации), либо в 1 в противном случае (б\'{о}льшая плотность единиц). Использование такого правила позволяет сохранить анонимность голоса и избежать прямого опроса избирателей.

В наиболее часто изучаемой версии задачи классификации клеточных автоматов количество клеток N=149. Центральная клетка обозначается символом u, три клетки слева от неё - <<c>>, <<b>>, <<a>>, три клетки справа - <<1>>, <<2>>, <<3>>.

Плотность начальной конфигурации~--- это функция от N аргументов, потому действия каждой клетки с ограниченными информацией и коммуникационными возможностями должны быть согласованы со всеми остальными для корректной классификации начальной конфигурации. Более того, ручная обрабока пространства поиска, включающего $2^{2^7}=2^{128}$ правил перехода, является практически невозможной задачей, потому для поиска правил, лучших чем уже известное правило Гача-Курдюмова-Левина, был использован эволюционный подход. При помощи коэволюционного обучения Поллаком и Жиллем были обнаружены два новых правила (Коэволюция\_1 и Коэволюция\_2), значительно превосходящие по характеристикам все известные до этого: их эффективность составляет 85.1\% и 86.0\% соответственно. Однако эти правила представлены в виде битовых таблиц, а булевы выражения этих правил неизвестны, потому извлечение знаний из них не представляется возможным.
Пространство вариантов правил булевых функций семи аргументов огромно~--- $2^{2^7}=2^{128}$ правил, но ещё больше пространство компьютерных программ, состоящих как из функциональных, так и терминальных элементов. Потому поиск булевых функций, представляющих правила Коэволюция\_1 и Коэволюция\_2, является нетривиальной задачей.

Применение ПЭГ к этой задаче потребовало несколько предварительных оптимизационных запусков. Лучшее решение запуска использовалось в последующем запуске. Такая стратегия применяется при решении сложных задач, т.к. при поиске с нуля непросто найти даже промежуточное хорошее решение. Функциональное множество было составлено из булевых НЕ, И, ИЛИ, ИСКЛЮЧАЮЩЕЕ ИЛИ, И-НЕ, ИЛИ-НЕ, терминальное~--- из c, b, a, u, 1, 2, 3. Фитнес отражает количество корректно вычисленных элементов выборки. Алгоритмом ПЭГ были успешно обнаружены формулы, со 100\% точностью описывающие искомые правила.

В работе~\cite{Banks:gecco05lbp} рассматривается применение ПЭГ в задаче поиска захоронённых боеприпасов путём анализа показаний электромагнитных сенсоров. Испрользовались данные эксперимента JPG-IV, проводимом армией США на полигоне Джефферсон. Обучение и тестирование системы проводилось по данным, полученным в результате эксперимента JPG-IV, проводимом армией США на полигоне Джефферсон. Полученная система показала эффективность, превосходящую большинство описанных в работе решений.

В работе~\cite{visoiu2011deriving} показано использование ПЭГ для построения рыночной модели по биржевым данным.

Задача автоматического реферирования (Automatic text summarization) исследуется на протяжении десятков лет. Большинство подходов к её решению сводится к комбинированию статистических методов и лингвистического анализа. Значительно реже предлагается воспользоваться машинным обучением. В работе~\cite{ZhuliXie:2004:COLING} предлагается применить для обучения эволюционный алгоритм ПЭГ. В реализованной системе каждое предложение $s$ представлено пятью нормализованными характеристиками:
\begin{enumerate}
  \item Положение абзаца $P = Y / M$, где $M$~--- общее количество абзацев, $Y$~--- индекс абзаца, в котором находится предложение.
  \item Положение предложения $S = X / N$, где $N$~--- общее количество предложений в абзаце, $X$~--- индекс предложения.
  \item Длина предложения $L$: $$L = \frac{1-e^{-\alpha}}{1+e^{-\alpha}}, \alpha=\frac{l(s) - \mu(l(s))}{std(l(s))},$$ где $l(s)$~--- количество слов в предложении, $u(l(s))$~--- среднее значение этой величины, $std(l(s))$~--- её стандартное отклонение.
  \item Заголовок $H=1$, если предложение является названием, заголовком либо подзаголовоком, иначе $H=0$.
  \item Част\'{о}ты слов $F$:$$F=\frac{1-e^{-\alpha}}{1+e^{-\alpha}}, \alpha=\frac{CW(s) - \mu(CW(s))}{std(CW(s))},$$ $$CW(s)=-\sum\limits_{i=1}^k\log{[Freq(w_i)]}, w_i\in{s},$$ где $Freq(w_i)$~--- количество употреблений слова $w_i$ в статье.
\end{enumerate}

Основой системы является предположение о том, что к каждому определённому типу документов применим один и тот же механизм реферирования, потому для построения рефератов набора документов достаточно найти один алгоритм. Функция фитнеса такой системы~--- количественно оценённая похожесть полученного реферата с составленным человеком, путём векторизации текстов и их последующего скалярного произведения. В качестве реферата возвращаются $N$ предложений с максимальной оценкой, выданной системой, $N$ задаётся в зависимости от желаемой длины реферата.

За недоступностью других систем автоматического реферирования для сравения с ними эффективности полученной системы были разработаны три простых метода:
\begin{itemize}
  \item Составление реферата из первых предложений первых пяти абзацев.
  \item Случайным образов выбираются пять предложений из полного текста.
  \item Из случайном образом выбранных пяти абзацев отбираются первые предложения.
\end{itemize}

Полученная система, основанная на ПЭГ, на 58--160\% превосходит приведённые простые методы. Тем не менее, приемлемая эффективность получена не была. Основная причина~--- субъёктивность оценки путём сравнения с <<эталонными>> рефератами, написанными человеком, т.к. стили написания могут различаться и даже противоречить друг другу.

В работе~\cite{journals/jucs/AbrahamG06} предлагаются три варианта применения ГП для мониторинга электронных цепей и систем в реальном времени. Надёжность цепи оценивается на основе показаний сенсоров восприимчивости цепи на электромагнитные воздействия, которые затем подаются на обработку эволюционному алгоритму для вынесения конечного решения о состоянии цепи.

Вероятность отказа компонента зависит как от стрессора (электрического, механического либо другого физического воздействия на компонент на протяжении всего срока его работы), так и чувствительности компонента к данному стрессору. Кроме того, большинство компонентов имеют несколько механизмов возникновения отказа по причинам: перегрева (стрессоры: рассеивание тепла, высокая температура окружающей среды, термическое сопротивление, теплоёмкость), различных видов пробоев (изменение проводимости компонента из-за примесей в материалах либо вследствие изменения температуры, воздействие электромагнитного поля), коррозии, утечки тока и пр.

Существуют два способа получения набора стрессоров цепей: компьютерная симуляция моделей компонентов и цепей, либо анализ экспериментальных измерений и построение моделей <<стрессор-восприимчивость>>. Второй способ позволяет применить алгоритм ПЭГ к решению поставленной задачи.

Проведено сравнение эффективности решения поставленной задачи следующими методами: линейное генетическое программирование (LGP~--- Linear Genetic Programming), мультигенное ГП (MEP~--- Multy Expression Programming), ПЭГ, ИНС, а также деревья классификации и регрессии (CART~--- Classification and Regression Trees).

Линейное ГП является ответвлением классического ГП, в котором программа представляется не древовидной, и линейной структурой, наиболее похожей по форме на построчный ассемблерный листинг, каждая строка которого~--- один оператор. Сходство подтверждает также использование временных переменных-<<регистров>> для передачи данных между строками программы.

В мультигенном ГП каждая хромосома состоит из заданного количества генов - элементов, представляющих либо терминал, либо функцию и индексы аргументов. Аргументы для исключения циклических зависимостей ограничены лишь ссылками на предыдущие гены хромосомы. Таким образом, результатом декодирования хромосомы является синтаксическое дерево. Некоторые элементы генома могут пропущены при построении дерева, если на них в конкретно взятом образце не было ссылок, а потому не использованы.

На вход алгоритма подаются значения напряжения и силы тока, проходящего через цепь, выход (моделируемая величина) - температура узлов цепи и сила токов утечки. Достоинством ГП являето то, что полученная программа легко может быть реализована аппаратно. Модели, полученные при помощи ГП отличаются высокой точностью решения задачи и лёгкостью в использовании.

В работе~\cite{conf/ijcnn/ValdesB06} представляется метод конструирования пространств виртуальной реальности для визуального анализа данных при помощи многоцелевой оптимизации средствами ПЭГ. 

Поставленная задача состоит в составлении компактного аналитического представления отображения многомерного пространства на двух- или трёхмерное. Данный подход был применён к практической задаче обнаружения подземных полостей, как правило, заполненных водой. Карты подземных течений местности зачастую составлены неточно, т.к. пустоты в большинстве случаев не имеют выхода на поверхность, потому для их обнаружения требуется исследование местности геофизическими методами. В их число входят: измерения электрического потенциала поверхности почвы в сухое и дождливое времена года, значение вертикальной компоненты электромагнитного поля в низкочастотном спектре, показания интенсивности гамма-излучения, топография местности (высотная карта). Результатом измерения каждой физической характеристики является поверхность - набор значений, соответствующих точкам местности.

После обработки алгоритмом многоцелевой оптимизации NSGA-II эти данные подаются на вход ПЭГ. Критерием успешности модели была выбрана минимизация среднеквадратичного отклонения от ожидаемых величин.
Полученные в результате эксперимента формулы с высокой точностью описали распределение моделируемых величин, а задействованные переменные позволили выявить величины, анализируя которые можно диагностировать циркулирующие аномалии.

В работе~\cite{Kwasnicka:2006:FIMCSIT} рассматривается задача автоматизации компьютерной анимации трёхмерных моделей. Под анимацией понимается описание движения моделей персонажей, представленных в виде подвижно сочленённых недеформируемых блоков. Был разработан новый метод создания компьютерной анимации трёхмерных моделей с помощью ПЭГ. Обычно в этих целях использются следующие методы:
\begin{itemize}
  \item Метод ключевых кадров~--- задание главным аниматором позиций моделей в ключевых кадрах и последующая отрисовка деталей движения в промежуточных кадрах менее искусными аниматорами либо автоматически.
  \item Прямая, обратная кинематики~--- преобразование скелетной анимации, заданной в координатах суставов скелета с учётом их свойств, в декартову систему координат сцены.
  \item Физическая симуляция: динамика. Применяется для достижения максимальной реалистичности, учёта сил и моментов, гравитации и инерции.
  \item Поведенческие техники~--- моделируемая система представляется в виде частиц, управляемых набором относительно простых правил, позволяющих создать сложное движение.
  \item Оптимизационные методы:
  \begin{itemize}
    \item Минимизация энергии.
    \item Задание пространственно-временных условий и ограничений.
    \item Эволюционные алгоритмы.
  \end{itemize}
\end{itemize}

Большинство описанных методов, в особенности метод ключевых кадров, прямая и обратная кинематики, требуют высокой квалификации аниматора для установления соответствия движений природе персонажа, а деталей визуализации~--- физическим законам. Потому даже частичная автоматизация процесса позволит существенно сократить количество ручной работы.

Основные правила создания <<идеальной>> анимации были разработаны в начале XX века, в частности усилиями студии Диснея. Всего таких принципов 12:
\begin{itemize}
  \item Темп~--- скорость движения объекта должна соответствовать причине его движения, давать представление о его массе, упругости, пр.
  \item Плавность движения~--- движение не может начинаться и заканчиваться рывком.
  \item Плавность траектории~--- следует избегать изломов в направлении движения, заменять их изгибами. При моделировании живых персонажей соединения костей скелета описываются вращательными движениями суставов.
  \item Обусловленность~--- каждое действие анимируемого персонажа обычно состоит из трёх фаз: подготовка, само движение и его окончание. Обусловленность относится к подготовке к движению: перед прыжком персонаж, как правило, сгибает ноги в коленях, что связано с физикой прыжка.
  \item Гиперболизация~--- усиление эффекта движеня для концентрации на нём внимания зрителя.
  \item Деформация (сжатие и разжатие) отражает воздействие на объект силы, приводящей его в движение.
  \item Второстепенные движения~--- свойственные живым организмам разнообразные движения, такие как дыхание, сложная механика бега, включающая слежение взглядом за окружением.
  \item Совмещение движений~--- плавный переход окончания первого движения в начало второго.
  \item Покадровая и позиционная анимация~--- два взаимоисключающих подхода к созданию анимации. Первый предполагает последовательную ручную отрисовку от начального кадра до конечного. Второй~--- создание ключевых кадров с автоматическим построением промежуточных на основе свойств сочленений.
  \item Постановка~--- ясное и понятное представление основной идеи анимации.
  \item Обращение к зрителю~--- привлекательность происходящего, сценария и персонажей.
  \item Реалистичность.
\end{itemize}

В предложенном решении на основе ПЭГ движения персонажей (моделей) контролируются компьютерными программами, создаваемыми в ходе эволюции. Участие аниматора заключается в количественном оценивании полученных вариантов анимации, эти оценки служат критерием успешности при построении очередной популяции вариантов. Программа-контроллер управляет подвижными частями модели учитывая при этом заданные ограничения, такие как силу мышц персонажа, его массу. Таким образом можно косвенно управлять поведением модели: увеличивая параметр, отвечающий за массу персонажа, ему становится доступным только ходьба, но не прыжки.

При оценке учитывается расстояние между требуемым и полученным местоположениями модели и <<стиль>> движения (персонаж может достичь цели ползком, прыжком, бегом). Стиль включает в себя следующие категории: безопасность~--- отсутствие столкновений с другими объектами; время~--- назначение штрафа за слишком медленное или слишком быстрое передвижение и рывки; достижение равновесия~--- степень нейтральности конечного положение (персонаж стоит, а не лежит); прочее~--- в зависимости от типа движения.

Для демонстрации работоспособности предложенной автоматизации процесса анимации были получены следующие сюжеты: приземление персонажа, состоящего из пяти элементов, в заданную область тремя прижками; передвижение паукообразного существа в заданную точку. Была достигнута высокая реалистичность поведения персонажей, на что в случае ручной отрисовки потребовалось бы длительное время.

Другая задача компьютерной анимации~-- создание трёхмерных моделей деревьев и других растений для дальнейшей визуализации в составе сцены. Обычно в этих целях используются процедурные техники, состоящие из инструкций по построению дерева, но не описывающие явным образом его геометрию. Представление в виде программы позволяет применить к поставленной задаче эволюционный подход. Качество полученных деревьев, как и в предыдущей работе, предлагается оценивать аниматору на основе их эстетического вида, эти оценки используются для оператора отбора особей~\cite{conf/afrigraph/VenterH07}.

Одной из лучших процедурных техник для построения моделей растений являются L-системы (L-Systems~--- Lindnenmayer Systems~--- системы Линденмайера), вернее, их детерминированная контексто-независимая разновидность, состоящая из алфавита символов, аксиомы и набора правил, заменяющих символ-предшественник строкой-последователем. Вариант решения при этом случае описывается аксиомой (начальной строкой) и набором правил. Для кодирования таких решений используются мультигенные хромосомы. Первый ген кодирует аксиому, остальные - правила.

Несмотря на то, что оценка аниматором зависит не только от характеристик оцениваемого дерева, но и от характеристик других дереьев популяции и состояния самого аниматора, это не мешает работе алгоритма, потому как имеет значение лишь отношение качества оцениваемого дерева к качеству остальных деревьев популяции. Так, если одна особь по решению оператора в два раза лучше другой~--- её фитнес также будет в два раза выше, как и вероятность отбора.

Для ускорения работы алгоритма количество рекурсивных итераций было сокращено~--- полученные деревья при этом достаточны для оценки. При дальнейшем использовании полученной L-системы для построения финального дерева количество рекурсивных итераций устанавливается максимальным.

Полученная система позволяет автоматизированно получить качественную модель растения, не прибегая к трудоёмкому ручному построению.

Сети беспроводных сенсоров (Wireless Sensor Network ~--- WSN) широко применяются для обнаружения, определения местоположения и отслеживания перемещений движущихся объектов. В помещении определённым либо произвольным образом устанавливается множество сенсоров, определяющих и запоминающих местоположение цели в моменты времени $t_0, t_1, ..., t_i$. Поставленной задачей является расчёт траектории цели и прогнозирование её положения в моменты времени $t_{i+1}, t_{i+2}, ...$ При использовании сети сенсоров в больших помещениях стоит задача балансирования между энергопотреблением аппаратных сенсоров сети (временем включения и выключения) и точностью распознавания движения.

Одно из возможных решений задачи заключается в следующем. Состояние цели включает в себя положение, направление и скорость. На каждом такте работы системы сенсоры, находящиеся рядом с целью, кластеризуют показания, а центры полученных кластеров подаются на вход фильтра Калмана. Реализация такого подхода не представляет сложностей в централизованной среде, однако затруднена в распределённых сетях, таких как WSN, состоящих из компактных устройств с ограниченными вычислительными возможностями.

В работе~\cite{Dai:2009:ETA:1726588.1727798} была предложена система, лишённая указанного недостатка. Её конструирование состоит из трёх этапов.

На первом этапе для распознавания движения был разработан алгоритм распределённого ПЭГ, выполняющийся на нескольких взаимодействующих сенсорах. Движение каждой цели описывается формулой, полученной на основании $h$ последних координат цели~--- т.н. алгоритм скользящего окна.

На втором этапе алгоритм скользящего окна настраивается на быстрое обучение алгоритма распределённого ПЭГ: при большом расхождении предсказанного движения с наблюдаемым предыдущие накопленные данные о цели отбрасываются, определяя таким образом изменения характера движения цели и позволяя алгоритму ПЭГ построить новую формулу движения цели.

Третий этап представляет собой моделирование работы системы.

Алгоритм распределённого ПЭГ:
\begin{enumerate}
  \item Активация сенсоров планировщиком либо при ожидаемом поступлении цели в область видимости.
  \item Запуск узлами алгоритма ПЭГ при наличии в области видимости целей для поиска их траекторий. Каждый узел-сенсор, построивший формулу движения цели, отправляет её всем соседним узлам.
  \item Узлы, получившие рассчитанную траекторию цели от другого узла, прерывают запущенный алгоритм ПЭГ, и начинают использовать полученную формулу движения для дальнейшего слежения за целью.
\end{enumerate}

Критерии останова алгоритма ПЭГ: произведено максимальное количество поколений, превышено время выполнения, получение траектории от другого узла, успешный расчёт траектории.

В ходе моделирования системы была показана 25\% экономия энергопотребления по сравнению с системой на основе фильтра Калмана и системой на основе алгоритма максимального сближения (Enhanced Closest Point of Approach~--- ECPA).


\subsection{Системы дифференциальных уравнений}

Одна из задач идентификации системы состоит в построении модели, представленной обыкновенным дифференциальным уравнением (максимальный порядок задаётся пользователем), выведенным из поданных данных~\cite{conf/iscis/FloresG05}.

При таком подходе синтаксическое дерево представляет собой правую часть дифференциального уравнения, записаного в нормальной форме:
$$
y^{(n)} = f(t, y, y', y'', ..., y^{(n-1)}).
$$

Терминальное множество дополняется производными функциями, как показано на рисунке:

\begin{figure} [h]
  \center
  \begin{tikzpicture}[level distance=1cm,
    level 1/.style={sibling distance=3cm},
    level 2/.style={sibling distance=2cm},
    level 3/.style={sibling distance=1cm}]
    \tikzstyle{every node}=[-,thick]
    \node { $+$ }
      child[->] { node { $\times$ }
        child { node { $7$ } }
        child { node { $\dfrac{dx}{dt}$ } }
      }
      child { node { $+$ }
        child { node { $12$ } }
        child { node { $\times$ }
          child { node { $10$ } }
          child { node { $x$ } }
        }
      }
    ;
  \end{tikzpicture}
  \caption{Пример дерева с терминалами-производными}
  \label{img:example_tree_with_differentials}
\end{figure}

Моделируемая функция рассчитывается из полученной синтаксической конструкции методом Рунге-Кутты. Проверка эффективности данного подхода проводилась на задачах моделирования линейного маятника, нелинейного маятника с трением, колебательной системы <<пружина-масса>> и линейной электрической цепи. Во всех четырёх экспериментах был достигнут коэффициент корреляции $R^2 > 0.99$. Данный подход обладает следующими преимуществами: возможность моделирования системы, представленной в виде ОДУ, автоматическое определение порядка системы, возможность получения линейной модели.

Аналогичные исследования систем дифференциальных уравнений проведены в работе~\cite{conf/acsc/ZarnegarVS09}. При изучении генома, например, при сравнении развития раковых и нормальных клеток, широко используются ДНК-микрочипы, упрощающие исследование экспрессии генов, их функций. Последовательности ДНК при таком подходе закрепляются на твёрдой основе ДНК-микрочипа, формируя двумерный микромассив. Информация об изменениях экспрессии генов в микромассиве с течением времени записывается в виде генной сети. Генная сеть может быть представления либо вероятностными методами, такими как байесовские сети, либо детерминированными, например, в виде систем дифференциальных уравнений~--- наиболее широко распространённый метод.

Системы дифференциальных уравнений, полученные из набора данных с помощью ПЭГ, численно решаются методом Монте-Карло.

Несмотря на способность ПЭГ обнаруживать структуру решений, метод в чистом виде не эффективен в оптимизации параметров-констант. Поэтому для этих целей к каждой особи популяции был применён метод наименьших квадратов, что существенно улучшило точность результатов. Наложение гауссового шума на данные лишь незначительно влияет на точность алгоритма.


\subsection{Построение искусственных нейронных сетей}

Применение ИНС чрезвычайно популярно при решении задач классификации, машинного обучения, идентификации систем и пр. Однако от оператора требуется, как правило, предварительно указать топологию сети, перед дальнейшим обучением. Структура сети определяется набором таких параметров как количество слоёв, количество нейронов в каждом из них, связи между нейронами и соотвествующие функции активации. Тестирование каждой конфигурации параметров требует отдельного запуска процедуры обучения и тестирования сети.

Для автоматического выбора наиболее эффективной топологии сети возможно использование эволюционных алгоритмов. Применимость генетических алгоритмов к указанной задаче показана в работе~\cite{stanley1996efficient}. Процедура кодирования генотипа и фенотипа ПЭГ может быть легко применена к древовидной структуре ИНС, аналогичной структуре синтаксического дерева. Таким образом возможно закодировать ИНС целиком в геноме особи. В работах~\cite{ferreira:2004:wsc9, Ferreira:wsc9, li2005new, conf/gecco/JohnsS09} в качестве функционального множества используются элементы, представляющие нейроны разных типов. Хромосома дополняется массивом весов связей и массивом пороговых значений функций активации.

Таким образом, в ходе эволюции популяция может содержать особи с разной структурой, следовательно, становится возможным как автоматический выбор структуры, так и настройка весов (обучение).

В качестве примеров использования таких систем приводится построение правил классификации.


\subsection{Кластеризация}

Большинство существующих методов кластеризации требуют задания определённых параметров, например, количество кластеров либо их радиус. Задание этих параметров требует априорных знаний о данных и на практике затруднительно. Кроме того, наиболее популярный метод кластеризации~--- К-средних~--- к тому же чувствителен к начальным значениям центров кластеров, вследствие чего зачастую сходится к локальному оптимуму. Для устранения данных недостатков в рабоет~\cite{Chen:2007:CWP:1304603.1305730} был предложен метод на основе алгоритма ПЭГ.

Функциональный набор содержит всего два элемента: $\bigcup$~--- оператор сегментации, объединяющий две точки в множество, и $\bigcap$~--- оператор аггрегации, возвращающий центроид - точку с координатами, равными полусуммам координат точек-аргументов. В случае, когда аргумент~--- множество, а не точка, участвуют все входящие в него точки. Функция финтеса~--- величина, обратная средней сумме квадратов расстояний точек до центров их кластеров. По окончании работы алгоритма близкие кластеры объединяются в один.

Проведённые эксперименты с синтетическими данными показали высокую (96\%) способность предложенного метода обнаруживать адекватный набор кластеров. В то же время эффективность алгоритма снижается при увеличении размерности данных. Другой существенный недостаток~--- чувствительность к шуму.


\subsection{Параметрическая регрессия}

Параметрическая символьная регрессия~--- задача поиска математических выражений, отличается отличие от вышеописанных случаев видом искомых выражений: вместо записи $y(x)$ производится поиск системы вида $y(t), x(t)$, где $t$ - параметр.

Перед алгоримом ПЭГ была поставлена~\cite{banks:2004:msa:erban} задача поиска брахистохроны~--- кривой скорейшего запуска, описываемой следующими уравнениями:
\begin{eqnarray}
X & = a\times(\theta - \sin{\theta})\\
Y & = a\times(1 - \cos{\theta})
\end{eqnarray}

Параметрическое описание выбрано по причине невозможности описания в виде $Y=Y(X)$ (хотя существует обратное решение в виде $X=X(Y)$).

Несмотря на многократные запуски алгоритма с различными параметрами, точные выражения искомой кривой обнаружены не были, хотя и было найдено множество решений, очень близких к оптимальному. На основании этого сделан вывод о том, что ГП не способно открыть уравнения брахистохроны, потому были исследованы возможные причины этого.

Множество кривых, с приемлемой точностью описывающих заданный набор данных, является множеством локальных оптимумов. Пространство решений и их присобленности можно визуализировать в виде зубной щётки, каждая щетинка которой практически неотличима от другой, и представляет собой локальный оптимум. Их обилие делает крайне сложным различие среди них определённого решения. Данная аналогия позволяет понять, почему истинные параметрические уравнения не будут обнаружены алгоритмом даже после сколь угодно большого числа часов работы.

Проведённным в работе~\cite{chen2012some} математическим анализом исходного ПЭГ, представленного в виде марковской модели, было доказано, что ПЭГ не является алгоритмом оптимизации, гарантированно сходящимся к глобальному оптимуму даже за бесконечное время. Однако вдалее было показано, что таким свойством обладает многократный запуск ПЭГ с механизмом сквозного простого элитизма (копированием лучшей особи предыдущего запуска в начальную популяцию следующего).


\subsection{Прогнозирование временных рядов}

Прогнозирование временн\'{ы}х рядов~--- одна из важных задач глубинного анализа данных. Традиционно при её решении применяется метод скользящего окна, и применимо к ПЭГ данный метод выглядит следующим образом: задаются набор данных (точек), полученных через фиксированные промежутки времени $\Delta t$ $X = (x_0, x_1, ... x_n)$, длина истории $h$, требуется найти формулу $f$, описывающую для любого $m, (n - h + 1 \leq m \leq n)$ прогнозируемое значение $\hat{x_m}$:
$$
\hat{x_m} = f(x_{m-h}, x_{m-h+1}, ..., x_{m-2}, x_{m-1}), (h < m \leq n)
$$

Оценкой формулы может быть абсолютная погрешность $|\hat{x_m} - x_m|$, относительная погрешность $\frac{|\hat{x_m} - x_m|}{x_m}$ либо другие измерения, например коэффициент корреляции.

Процедура может быть описана как скользящее по временн\'{о}й шкале окно шириной $h + 1$, формула $f$ прогнозирует будущие значения на основе $h$ предыдущих. Однако несмотря на простоту и скорость, методу скользящего окна не хватает семантической силы, возможностей раскрывать ключевые свойства, скрытые во временн\'{ы}х рядах. Для преодоления этого недостатка предложен~\cite{conf/waim/ZuoTLYC04, conf/cilamce/Lopes04} метод обыкновенных дифференциальных уравнений, состоящий из двух этапов: построение дифференциальных уравнений высших порядков из обучающей выборки данных, и прогнозирование с использованием полученных уравнений и начальных условий. Таким образом, в терминальное множество кроме символа $x(t)$ добавляются символы $x'(t), x''(t), ...$. Полученные дифференциальные уравнения затем вычисляются методом Рунге-Кутты.

Для устранения влияния зашумлённости данных на поиск начальных условий высокочастотная составляющая данных удаляется с применением преобразований Фурье.

Система, реализующая прогнозирование временных рядом, описана в работе~\cite{viento}, посвященной предсказанию ветра для дальнейшего использования этих данных при планировании размещения ветряных электростанций при помощи GEP, и сравнению полученных результатов с аналогичными полученными с использованием интегрированной модели авторегрессии со скользящим средним (ARIMA~--- autoregressive integrated moving average). Эксперимент показал, что модель, полученная в результате эволюционного процесса ПЭГ, имеет значительно меньшую погрешность, чем полученная статистической процедурой ARIMA. Кроме того, в работе~\cite{buarbulescu2009time} представлен анализ применения к данной задаче адаптивной модификации ПЭГ~--- AdaGEP.
