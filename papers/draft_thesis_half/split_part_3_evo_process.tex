\section{Статья 3: Исследование модификаций эволюционного процесса в программировании с экспрессией генов}

\subsection{Введение}

В ходе анализа существующих методов и средств, направленных на ускорение получения модели алгоритмом ПЭГ и повышение её качества, а также проведённых экспериментов был выявлен ряд ограничений производительности алгоритма, таких как длительность вычисления фитнеса, отсутствие тонкой подстройки числовых констант, вляние размера хромосомы на скорость сходимости, проблемы с поиском сложных моделей. Далее описаны методы и подходы, созданные в ходе преодоления данных ограничений.

Алгоритм ПЭГ заключается в выполнении следующих этапов:
\begin{enumerate} \itemsep0pt \parskip0pt \parsep0pt
  \item Создание начальной популяции
  \item Декодирование особей
  \item Выполнение программ особей
  \item Вычисление фитнеса особей
  \item Проверка критерия останова алгоритма (достигнута требуемая точность, либо истекло лимитированное время выполнения)
  \item Копирование лучшей особи (Элитизм)
  \item Отбор
  \item Репликация
  \item Мутация
  \item Операторы переноса
  \item Операторы рекомбинации
  \item Возврат к пункту 2
\end{enumerate}



%--------------------------------------------------------------------



\subsection{Методика тестирования}

В предыдущей статье цикла [ссылка на статью 1] описана методика оценки эффективности модификаций алгоритма ПЭГ, заключающаяся в статистической обработке результатов множества запусков с целью получения таких основных метрик: СКО наилучшей модели среди всех запусков ($e_{b}$) и доля успешных запусков ($r_{f}$)~--- таких, в ходе которых была получена модель с приемлемой для данной задачи точностью.

Тестирование производилось на следующих задачах:
\begin{enumerate}
  \item $y(x) = \sin x, x \in [-5, 4.6]$;
  \item Функция Розенброка: $z(x, y) = {(1 - x)}^2 + 100 {(y - x^2)}^2, x \in [-3, 3], y \in [-3, 3]$;
  \item Сумма четырёх сигмоид:
    \begin{multline}
      z(x, y) = \frac{1}{2} \times \exp\left(-\frac{(9x-2)^2 - (9y-2)^2}{4}\right) + \frac{3}{4} \times \exp\left(-\frac{(9x+1)^2}{49} + \frac{(9y+1)^2}{10}\right) + \\
      + \frac{1}{2} \times \exp\left(-\cfrac{(9x-7)^2 + (9y-3)^2}{4}\right) -\frac{1}{5} \times \exp\left(-(9x-4)^2 - (9y-7)^2\right), \\
      x \in [-2, 2], y \in [-2, 2]
    \end{multline}
\end{enumerate}



%--------------------------------------------------------------------



\subsection{Обеспечение разнообразния начальной популяции}

Для наиболее эффективного исследования пространства поиска особи начальной популяции должны быть как можно меньше похожи между собой. Создание начальной популяции случайным образом не предполагает никаких процедур по обеспечению генетического разнообразия.

Сравнение особей между собой наиболее удобно производить по генотипу по причине его линейности и лёгкости считывания. Целесообразно при этом сравнивать только кодирующие участки, участвующие в построении синтаксического дерева.

Для количественного измерения близости хромосом между собой в качестве метрики предлагается~\cite{Duan:2007:SID:1304604.1305918} использовать правило, возвращающее максимальное количество $r$ подряд идущих совпадающих элементов. Если две сравниваемые последовательности равны друг другу, $r$ будет равно длине последовательности. Два гена считаются близкими, если значение $r$ превышает определённый порог, авторами использовалось значение 7.
Процедура создания начальной хромосомы с использованием описанной техники выглядит следующим образом:

\begin{enumerate} \itemsep0pt \parskip0pt \parsep0pt
  \item Создание пустой популяции.
  \item Создание новой особи случайным образом.
  \item Сравнение этой особи с каждой особью, добавленной в популяцию.
  \item Если новая особь близка какой-либо особи в популяции, перейти к шагу~2, иначе добавить особь в популяцию.
  \item Если популяция полностью заполнена, завершить процедуру, иначе перейти к шагу~2.
\end{enumerate}

Были проведены эксперименты по моделированию различных наборов данных, предоставляемых Лёвенским университетом. Полученные модели обладают б\'{о}льшей корреляцией с тестовыми данными, чем результаты работы исходного алгоритма ПЭГ.



%--------------------------------------------------------------------



\subsection{Модификации эволюционного процесса}

\subsubsection{Возврат в исходное состояние}

Когда эволюционный процесс проходит определённое количество поколений, средний фитнес популяции достаточно высок, однако разнообразие практически устранено, что приводит к преждевременному схождению к локальному оптимуму, уменьшая шансы успешной глобальной оптимизации. Для решения этой проблемы было предложено~\cite{zhong2006improve} реализовать явление атавизма~--- появление свойств далёких предков. 

Современная генетика описывает следующие причины возникновения атавизмов: рекомбинация утраченного гена предка в результате скрещивания или мутации, и устранение стопового элемента генома, заблокировавшего на определённом этапе экспрессию гена предка. Из природы рассмотренных причин следует обратимость процесса схождения популяции. Кратковременно развернуть процесса эволюции в обратном направлении можно при помощи реализации возврата популяции ПЭГ в исходное состояние (Backtraced GEP).

В основе алгоритма возврата лежит структура данных стек, хранящая <<контрольные точки>>~--- состояния популяции. Если фитнес лучшей особи в новой популяции (полученной в результате репликации и применения генетических операторов), выше, чем у лучшей особи в популяции на верхушке стека (в последней контрольной точке), это означает верное направление эволюции, потому новая популяция заталкивается в стек, формируя новую контрольную точку. В противном случае можно сделать вывод о тупиковой ветви эволюции, потому последняя контролная точка выталкивается из верхушки стека.

Применение методики возврата к предыдущему состоянию привела к значительному улучшению качества получаемых решений.

%--------------------------------------------------------------------

\subsubsection{Группировка особей и их параллельная эволюция}

Начиная с определённого поколения, находясь в поздней фазе эволюции, популяция обладает следующими признаками: сложная структура синтаксических деревьев, замедление скорости эволюции, незначительное разнообразие популяции. Однако было замечено~\cite{Jiang:2008:MPT:1473248.1474007}, что разбиение популяции на группы и их параллельная обработка позволяет избежать преждевременной сходимости.

Разбиение популяции на группы происходит следующим образом. Особи популяции сортируются в порядке возрастания значений фитнеса:

\begin{equation}
\label{eq:niches_G}
G = \{I_i | f(I_{i+1}) \ge f(I_i), 1 \le i \le N\}.
\end{equation}

Группой считается последовательность, в которой разница значений фитнесов соседних особей не превышает заданую величину $d$:

\begin{equation}
\label{eq:niches_G_i}
G_i = \{I_j | f(I_{j+1}) - f(I_j) \le d, 1 \le j \le N, 1 \le i \le Q\}
\end{equation}
где $Q$~--- количество образованных групп. Под плотностью группы понимается отношение её размера (количества особей) к размеру популяции:

\begin{equation}
\label{eq:niches_p_i}
p_i = \frac{|G_i|}{N}
\end{equation}

Тогда сумма плотностей групп одной популяции всегда будет равна единице:

\begin{equation}
\label{eq:niches_p_i_sum_1}
\sum\limits_{i=1}^Q{p_i} = 1
\end{equation}

Энтропия популяции, математическое ожидание и дисперсия фитнеса вычисляются таким образом:

\begin{equation}
\label{eq:niches_E}
E(G) = - \sum\limits_{i=1}^Q{p_i \times \log{p_i}}
\end{equation}

\begin{equation}
\label{eq:niches_M}
M(G) = \sum\limits_{i=1}^Q{\hat{f_i} \times p_i}
\end{equation}

\begin{equation}
\label{eq:niches_D}
D(G) = \frac{1}{N} \times \sum\limits_{i=1}^{N}{{f(I_i) - \hat{f}}^2}
\end{equation}

\begin{equation}
\label{eq:niches_f_i_hat}
\hat{f_i} = \frac{1}{|G_i|} \times \sum\limits_{j=1}^{|G_i|}{f(I_j)}
\end{equation}

\begin{equation}
\label{eq:niches_f_hat}
\hat{f} = \frac{1}{N} \times \sum\limits_{i=1}^{N}{f(I_i)}
\end{equation}

Если энтропия и дисперсия популяции меньше определённых заданых пороговых значений, можно сделать вывод о недостаточной степени генетического разнообразия. В этом случае имеет смысл заменить худшие 10\% особей новыми, созданными случайным образом. Указанные операции требуется применять к каждому поколению. Итоговый алгоритм работы алгоритма представлен в листинге~\ref{algo:niches}.

\begin{algorithm}
\SetAlgoLined
\While{Не достигнуто максимальное поколение}
{
  Создание начальной популяции\;
  Разбиение популяции на группы\;
  \ForEach{группы популяции}
  {
    Перенос лучшей особи\;
    Применение генетических операторов\;
  }
  Расчет энтропии и дисперсии популяции\;
  Замена худших особей популяции при необходимости\;
  Разбиение популяции на группы\;
}
Вывести лучшую особь\;
\caption{Алгоритм ПЭГ с группировкой особей}
\label{algo:niches}
\end{algorithm}

Авторами не было произведено сравнение производительности полученной системы с исходным ПЭГ или другими его модификациями.

Вляние отдельно взятой модификации, удаляющие худшие особи, показано в таблице~\ref{tbl:cmp_replace_worst}. Из этих результатов следует изменение баланса двух основных вероятностных показателей алгоритмов: улучшается точность наилучшей возможной модели при большом количестве запусков, но в то же время страдает вероятность обнаружения приемлемого решения.

\input{cmp_replace_worst}

%--------------------------------------------------------------------

\subsubsection{Популяции неоднородных особей}

Одна из проблем алгоритма ПЭГ~--- определение оптимального размера головы гена (и размера решения). Из-за отсутствия процедуры априорного задания приходится запускать алгоритм множество раз с разными параметрами для поиска наиболее подходящих. В качестве другого способа предлагается~\cite{Lopes:2004:AMCS} использовать в одной популяции хромосомы различной длины: половина популяции заполняется особями пропорционально с диапазоном размеров, заданным пользователем, длина генов особей второй половины устанавливается случайным образом. Операторы рекомбинации в таком случае применяются только к особям с хромосомами равной длины.

Для подстройки констант предлагается использовать градиентный алгоритм, обладающий высокой вычислительной стоимостью, потому применяющийся с определённой вероятностью. Константа либо заменяется случайно выбранной, либо изменяется в пределах 10\%. Если мутированная особь лучше исходной, то заменяет её, иначе константа заменяется полусуммой последних двух значений. Процесс повторяется до тех пор, пока мутировавшая особь не будет хуже исходной, либо по достижении предела в 10 итераций.

Аналогичные исследования проведены в работе~\cite{journals/acisc/BrowneS10}. Предложено использовать хромосомы с переменным числом генов и переменным размером каждого гена. Введены новые операторы, направленные на изменение длины генома:
\begin{itemize} \itemsep0pt \parskip0pt \parsep0pt
  \item Удаление одного ген из хромосомы.
  \item Создание и добавление одного нового гена в хромосому.
  \item Перенос участка головы одного гена в голову другого, что приводит к укорачиванию первого и удлиннению второго.
  \item Рекомбинация разнородных хромосом.
\end{itemize}

Такой подход привёл к двукратному сокращению длины гена и, соответственно, размера синтаксического дерева решения.

%--------------------------------------------------------------------

\subsubsection{Дополнительная популяция}

В качестве одной из мер повышения вероятности обнаружения решения в авторских работах~\cite{SergMir_03_2013_vkntu, SergMir_04_2013_smolensk} было предложено использование дополнительной параллельной независимой популяции. Если при очередной итерации (поколении) работы алгоритма фитнес лучшей особи дополнительной популяции превысит фитнес лучшей особи основной популяции~--- данная особь копируется на место худшей особи основной популяции. Из результатов, представленных в таблице~\ref{tbl:cmp_additional_population} можно сделать вывод об успешном решении поставленной задачи при использовании данного подхода. Негативным следствием является снижение точности наилучшей из получаемых моделей. Это происходит по причине возрастания вычислений в расчёте на итерацию алгоритма, следовательно при равном отводимом времени работы модифицированный алгоритм успеет расчитать меньшее количество поколений.

\input{cmp_additional_population}

%--------------------------------------------------------------------

\subsubsection{Инкрементальная эволюция}

В приведённой таблице~\ref{tbl:cmp_tree_depths} отчётливо заметно наличие некоторого оптимального размера генома, при котором достигается максимум производительности алгоритма. Данный размер требует отдельного определения для каждой задачи и каждого способа кодирования, для его выбора могут применяться как простой перебор параметров, так и описанные выше популяции неоднородных особей.

\input{cmp_tree_depths}

Более эффективным показал~\cite{SergMir_04_2013_varna, SergMir_04_2013_sovr} себя метод инкрементальной эволюции, в ходе которого производится ряд последовательных запусков алгоритма с наращиванием длины хромосомы при каждом запуске и копированием лучшей особи предыдущего запуска в начальную популяцию текущего. Такой подход требует меньших вычислительных ресурсов по сравнению с независимыми запусками, т.к. позволяет производить постепенное усложнение дерева. Результаты работы данной модификации приведены в таблице~\ref{tbl:cmp_incremental}.

\input{cmp_incremental}



%--------------------------------------------------------------------



\subsection{Комбинирование множества запусков алгоритма}

\subsubsection{Взвешенная сумма моделей}

Полученные в ходе нескольких запусков алгоритма ПЭГ модели можно~\cite{journals/jikm/AbrahamG06, guo2012novel} обобщить в одну, представляя итоговую формулу в виде:
\begin{equation}
\label{eq:ensembles}
M = a \times M_1 + b \times M_2 + c \times M_3 + \ldots
\end{equation}
где $a$, $b$, $c$, $\ldots$~--- комбинирующие коэффициенты, $M_1$, $M_2$, $M_3$, $\ldots$~--- модели, полученные в ходе запусков ПЭГ.

Для подобра комбинирующих коэффициентов с целью минимизации погрешности итоговой модели, повышению её корреляции с выборками данных был использован генетический алгоритм NSGA II (non-dominated sorting genetic algorithm II).

Как правило, обобщающая модель обладает лучшими характеристиками, чем её составляющие по отдельности.

%--------------------------------------------------------------------

\subsubsection{Итеративный разностный подход}

Одна из наиболее распространённых задач, для решения которой применяется ПЭГ~--- поиск математической формулы, описывающей набор численных данных, этот процесс называется символьной регрессией, либо, в более простом варианте, аппроксимацией функций. Примером такой задачи может служить обнаружение формулы, описывающей сложную поверхность. В ряде случаев сложность моделируемого объекта такова, что обеспечить приемлемую точность при описании компактной формулой невозможно, и требуется увеличить размер искомого дерева, что приводит к резкому росту пространства поиска, а следовательно и вычислительного времени.

Популярным средством повышения сложности формулы с минимальным влиянием на производительность алгоритма является развитие идеи сложных мультигенных хромосом с иерархической структуров, которые были описаны выше.

Ещё более действенным показал себя разработанный в ходе данных исследований разностный подход~\cite{SergMir_03_2014_info_problem, SergMir_04_2014_smolensk}, основанный на простоте комбинирования синтаксических деревьев: математические формулы могут быть легко объединены, например, при помощи функции арифметического сложения, правила классификаторов~--- булевыми <<И>> и <<ИЛИ>>, и т.д. Эта особенность позволяет составить сложную формулу из ряда простых, компактных и быстро вычисляемых по отдельности.

Суть подхода заключается в последовательном применении алгоритма ПЭГ с неизменным набором параметров к поверхностям ошибки~--- наборам численных данных, полученных путём вычитания очередной полученной модели из моделируемых данных. Тем самым разностных подход принципиально отличается от идеи эволюции мультигенных хромосом, где алгоритм пытается обнаружить решение с первой же итерации. При первом запуске на вход алгоритма ПЭГ подаётся набор данных~$T(O_{0})$, с ожиданием на выходе модели~$M$, возвращающей набор данных~$O$:

\begin{equation}
\label{eq:zerg_diff_m_1}
\{M_{1}, O_{1}\} = GEP(T = O_{0})
\end{equation}

На каждом следующем этапе на вход подаётся разность моделей:

\begin{equation}
\label{eq:zerg_diff_m_i}
\{M_{i+1}, O_{i+1}\} = GEP(O_{i} - O_{i - 1})
\end{equation}

Итоговой моделью после N запусков является:

\begin{equation}
\label{eq:zerg_diff_model}
M = M_{1} + M_{2} + \ldots + M_{N}
\end{equation}

В таблице~\ref{tbl:cmp_differential} показано сильнейшее положительное влияние разностного подхода на показатели алгоритма ПЭГ. Следует, однако, учитывать при этом возрастающее пропорционально количеству разностей время, затрачиваемое алгоритмом на поиск каждого слагаемого формулы модели.

\input{cmp_differential}



%--------------------------------------------------------------------



\subsection{Параллелизация}

Выполнение алгоритма требует значительных вычислительных ресурсов. Самым ресурсоёмким является этап расчета приспособленности программы~--– эту операцию требуется выполнять для каждой особи популяции по всему обучающему набору входных и выходных данных на каждой итерации алгоритма. Как правило, фитнес-функция в ПЭГ при решении задачи регрессии основывается на среднеквадратичном отклонении.

Для ускорения процесса целесообразно задействовать такой ресурс компьютера, как наличие нескольких процессоров. Одним из способов автоматизации распараллеливания программ является использование библиотеки, реализующей стандарт OpenMP.

При расчете фитнеса особи не возникает зависимости по данным от остальных особей популяции. Доступ к входным и выходным значениям восстанавливаемой функции предоставляется только на чтение, что позволяет обезопасить разделяемые данные. Выполнение этих условий необходимо и достаточно для эффективного распараллеливания расчета фитнеса популяции~--– к циклу программы можно добавить соответствующую директиву компилятора.

Кроме того, распараллеливанию поддается эволюционный этап алгоритма: текущая популяция объявляется разделяемыми данными с доступом на чтение, при этом в промежуточной популяции (заполняемой в ходе последовательного выполнения генетических операторов) нет зависимости по данным между особями.

В описываемом алгоритме значительная доля общего объема вычислений может быть получена параллельными расчетами, что позволило добиться ускорения выполнения в симметричных многопроцессорных системах~\cite{SergMir_05_2013_sevas}.



%--------------------------------------------------------------------


\subsection{Выводы}

Какие-то выводы
