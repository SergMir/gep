\section*{Введение}

Эволюционные вычисления~--- термин, применяемый к техникам глобальной оптимизации, характерной особенностью которых является подобие их процессов биологической эволюции. Вместо работы с одним определённым вариантом решения, эволюционные техники начинают процесс поиска с популяции вариантов решения, созданных случайным образом. Начальная популяция эволюционирует в набор улучшенных решений, итеративно проходя три этапа: отбор, рекомбинация и мутация. Наиболее приспособленные решения отбираются для участия в рекомбинации для создания нового поколения, а оператор мутации вносит в него разнообразие. В ходе этого процесса лучшие особи передают свои характеристики следующим поколениям. При достаточном количестве итераций эволюционные алгоритмы способны находить решения многих сложных задач.
Полученные решения моделируются для расчёта функции фитнеса, которая служит количественной оценкой приспособленности~--- степени соответствия полученного варианта решения (модели) условию задачи. При таком подходе генетическое программирование и другие эволюционные техники могут за короткое время находить оптимальные либо около-оптимальные решения задач и ситуаций. Используя эти техники, не требуется индивидуальный ручной поиск оптимального решения каждой задачи~--- этот процесс автоматизирован и выполняется компьютером, однако для этого требуются значительные вычислительные ресурсы~\cite{Nunez:2006:msoec}.

Генетические алгоритмы (ГА) были созданы в 1960-е, когда Джон Холланд применил теорию биологической эволюции к компьютерным системам. Как и все эволюционные компьютерные системы, ГА представляет собой упрощённую модель билогической эволюции. При таком подходе варианты решения задач кодируются символьными строками (обычно состоящими из нулей и единиц), и популяция этих вариантов решения участвует в процессе эволюции, цель которой~--- получение наилучшего решения.

Развитие идеи кодирования нелинейных структур разных размеров и форм Джоном Козой привело к появлению в 1992 году генетического программирования (ГП).

Программирование с экспрессией генов (ПЭГ), как в целом и ГА и ГП, является генетическим алгоритмом с такими характерными чертами как популяция особей, их отбор на основе приспособленности (фитнеса), изменчивость на основе генетических операторов~\cite{Ferreira2001}. Фундаментальное отличие этих трёх алгоритмов заключается в природе особей: в ГА это строки фиксированной длины (хромосомы); в ГП~--- нелинейные сущности переменной длины и формы (синтаксические деревья). В ПЭГ особи кодируются в виде строк фиксированной длины (называемых геномом, либо хромосомами~\cite{ferreira:2001:wsc6Aa}), которые затем декодируются для получения синтаксических деревьев.

Различие между ГА и ГП выражено не сильно: обе системы используют один вид объектов, который служит как генотипом, так и фенотипом. Такой подход имеет одно из двух ограничений: простота применения генетических операторов к объектам означает их недостаточную выразительность и сложность этих объектов (в случае ГА), а их сложность ведёт к трудностям при воспроизводстве и модификации (в случае ГП).

ПЭГ лишено указанных ограничений и обладает следующими преимуществами:
\begin{itemize}
  \item Простота хромосом: линейность, компактность, относительно небольшой размер, простота применения операторов, таких как репликация, мутация, рекомбинация, перенос и пр.
  \item Экспрессия генома порождает фенотип~--- синтаксическое дерево, которое может представлять математическую формулу либо программу. Значения вычисленной формулы, выполненной программы служат для численной оценки приспособленности особи с помощью функции фитнеса.
\end{itemize}

Таким образом, участие хромосомы в процессе воспроизведения зависит от фитнеса дерева, которое она кодирует. Для этого требуется универсальная система перевода формата генома в формат дерева и обратно, при которой любое изменение хромосомы с помощью операторов всегда приводит к синтаксически корректному дереву, обеспечивая приспосабливаемость и эволюционирование.
