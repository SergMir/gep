\documentclass[a4paper,12pt]{article}

%%% Поля и разметка страницы %%%
\usepackage{lscape}                               % Для включения альбомных страниц
\usepackage{geometry}                             % Для последующего задания полей

%%% Кодировки и шрифты %%%
\usepackage{cmap}                                 % Улучшенный поиск русских слов в полученном pdf-файле
\usepackage[T2A]{fontenc}                         % Поддержка русских букв
\usepackage[utf8]{inputenc}                       % Кодировка utf8
\usepackage[english, russian]{babel}              % Языки: русский, английский

%%% Математические пакеты %%%
\usepackage{amsmath,amsfonts,amssymb,amsthm,mathtools,amscd} % AMS

%%% Оформление абзацев %%%
\usepackage{indentfirst}                          % Красная строка

%%% Цвета %%%
\usepackage[usenames]{color}
\usepackage{color}
\usepackage{colortbl}

%%% Таблицы %%%
%\usepackage{longtable}                            % Длинные таблицы
\usepackage{multirow,makecell,array}              % Улучшенное форматирование таблиц

%%% Общее форматирование
\usepackage[singlelinecheck=off,center]{caption}  % Многострочные подписи
\usepackage{soul}                                 % Поддержка переносоустойчивых подчёркиваний и зачёркиваний

%%% Библиография %%%
\usepackage{cite}                                 % Красивые ссылки на литературу

%%% Гиперссылки %%%
\usepackage[linktocpage=true,plainpages=false,pdfpagelabels=false,unicode]{hyperref}

%%% Изображения %%%
\usepackage{graphicx}                             % Подключаем пакет работы с графикой

%%% Оглавление %%%
\usepackage{tocloft}

\usepackage{alltt}

\usepackage{algorithm2e}

\usepackage{fancyhdr}

%%% Диаграммы %%%
\usepackage{tikz}
\usepackage{pgfplots}
\usetikzlibrary{shapes, arrows, positioning}
\usepgfplotslibrary{external}
\tikzexternalize[mode=list and make]

\usepackage{subcaption}

%%% Макет страницы %%%
\geometry{a4paper,top=2cm,bottom=2cm,left=3cm,right=1cm}

%%% Выравнивание и переносы %%%
\sloppy                                           % Избавляемся от переполнений
\clubpenalty=10000                                % Запрещаем разрыв страницы после первой строки абзаца
\widowpenalty=10000                               % Запрещаем разрыв страницы после последней строки абзаца

%% Номера формул
\mathtoolsset{showonlyrefs=true}                  % Показывать номера только у тех формул, на которые есть \eqref{} в тексте.

%%% Библиография %%%
\makeatletter
\bibliographystyle{ugost2008ls}
\renewcommand{\@biblabel}[1]{#1.}                 % Заменяем библиографию с квадратных скобок на точку:
\makeatother

\renewcommand{\labelitemi}{-}                     % Дефис в качестве значка списка

%%% Оглавление %%%
%\renewcommand{\cftchapdotsep}{\cftdotsep}

%%% Интервалы %%%
\renewcommand{\arraystretch}{0.8}                 % Расстояние между строками таблицы
\captionsetup[table]{skip=1pt}                    % Расстояние между caption и таблицей
\renewcommand{\baselinestretch}{1.5}              % Междустрочный интервал

%%% Изображения %%%
\graphicspath{{./images/}}                        % Пути к изображениям

%%% Цвета гиперссылок %%%
\definecolor{linkcolor}{rgb}{0,0,0}
\definecolor{citecolor}{rgb}{0,0,0}
\definecolor{urlcolor}{rgb}{0,0,0}
\hypersetup{
    colorlinks, linkcolor={linkcolor},
    citecolor={citecolor}, urlcolor={urlcolor}
}

\pagestyle{fancy}
\fancyhf{}
\fancyhead[R]{\thepage}
\fancyheadoffset{0mm}
\fancyfootoffset{0mm}
\setlength{\headheight}{17pt}
\renewcommand{\headrulewidth}{0pt}
\renewcommand{\footrulewidth}{0pt}
\fancypagestyle{plain}
{ 
  \fancyhf{}
  \rhead{\thepage}
}


%%% Стили диаграмм %%%
\tikzstyle{decision} =
[
  diamond,
  draw,
  text width = 7em,
  text badly centered,
  node distance = 5mm,
  inner sep = 0pt
]

\tikzstyle{block} =
[
  rectangle,
  draw,
  text width = 15em,
  text centered,
  rounded corners,
  minimum height = 1em
]

\tikzstyle{line} =
[
  draw,
  -latex'
]

\pgfplotsset{
ZCmpAxis/.style={mark repeat=10, ylabel={СКО}, xlabel={Время, мс}}
}

%%% Строки алгоритмов %%%
\SetKwIF{If}{ElseIf}{Else}{если}{тогда}{иначе\ если}{иначе}{конец\ условия}
\SetKwFor{While}{до\ тех\ пор,\ пока}{выполнять}{конец\ цикла}
\SetKwBlock{Begin}{начало\ блока}{конец\ блока}
\SetKwFor{ForEach}{для\ каждого}{выполнять}{конец\ цикла}
\SetAlgorithmName{Алгоритм}{алгоритм}{Список алгоритмов}


\author{С. Мирошниченко}
\title{Модификации алгоритма программирования с экспрессией генов}

\begin{document}

%%% Переопределение именований %%%
\renewcommand{\abstractname}{Аннотация}
\renewcommand{\alsoname}{см. также}
\renewcommand{\appendixname}{Приложение}
\renewcommand{\bibname}{Литература}
\renewcommand{\ccname}{исх.}
\renewcommand{\chaptername}{Глава}
\renewcommand{\contentsname}{СОДЕРЖАНИЕ}
\renewcommand{\enclname}{вкл.}
\renewcommand{\figurename}{Рисунок}
\renewcommand{\headtoname}{вх.}
\renewcommand{\indexname}{Предметный указатель}
\renewcommand{\listfigurename}{Список рисунков}
\renewcommand{\listtablename}{Список таблиц}
\renewcommand{\pagename}{Стр.}
\renewcommand{\partname}{Часть}
\renewcommand{\refname}{Список литературы}
\renewcommand{\seename}{см.}
\renewcommand{\tablename}{Таблица}

\section{Статья 1: Исследование способов кодирования и создания синтаксических деревьев в программировании с экспрессией генов}

\subsection{Введение}

Эволюционные вычисления~--- термин, применяемый к техникам глобальной оптимизации, характерной особенностью которых является подобие их процессов биологической эволюции. Вместо работы с одним определённым вариантом решения, эволюционные техники начинают процесс поиска с популяции вариантов решения, созданных случайным образом. Начальная популяция эволюционирует в набор улучшенных решений, итеративно проходя три этапа: отбор, рекомбинация и мутация. Наиболее приспособленные решения отбираются для участия в рекомбинации для создания нового поколения, а оператор мутации вносит в него разнообразие. В ходе этого процесса лучшие особи передают свои характеристики следующим поколениям. При достаточном количестве итераций эволюционные алгоритмы способны находить решения многих сложных задач.

Функция, определяющая количественную оценку особи~--- степень соответствия построенной модели условию задачи~--- называется функцией фитнеса (fitness function), чаще всего применяется значение среднеквадратичного отклонения (СКО). При таком подходе генетическое программирование и другие эволюционные техники могут за короткое время находить оптимальные либо около-оптимальные решения задач и ситуаций. Используя эти техники, не требуется индивидуальный ручной поиск оптимального решения каждой задачи~--- описанный процесс автоматизирован и выполняется компьютером, однако для этого требуются значительные вычислительные ресурсы~\cite{Nunez:2006:msoec}.

Генетические алгоритмы (ГА) были созданы в 1960-е, когда Джон Холланд применил теорию биологической эволюции к компьютерным системам~\cite{Holland:1975}. Как и все эволюционные компьютерные системы, ГА представляет собой упрощённую модель билогической эволюции. При таком подходе варианты решения задач кодируются символьными строками (обычно состоящими из нулей и единиц), и популяция этих вариантов решения участвует в процессе эволюции, цель которой~--- получение наилучшего решения.

Развитие идеи кодирования нелинейных структур разных размеров и форм Джоном Козой привело к появлению в 1992 году генетического программирования (ГП)~\cite{Koza92}.

Программирование с экспрессией генов (ПЭГ), как в целом и ГА и ГП, является генетическим алгоритмом с такими характерными чертами как популяция особей, их отбор на основе приспособленности (фитнеса), изменчивость на основе генетических операторов~\cite{Ferreira2001}. Фундаментальное отличие этих трёх алгоритмов заключается в природе особей: в ГА это строки фиксированной длины (хромосомы); в ГП~--- нелинейные сущности переменной длины и формы (синтаксические деревья). В ПЭГ особи кодируются в виде строк фиксированной длины (называемых геномом, либо хромосомами~\cite{ferreira:2001:wsc6Aa}), которые затем декодируются для получения синтаксических деревьев.

Различие между ГА и ГП выражено не сильно: обе системы используют один вид объектов, который служит как генотипом, так и фенотипом. Такой подход имеет одно из двух ограничений: простота применения генетических операторов к объектам означает их недостаточную выразительность и сложность этих объектов (в случае ГА), а их сложность ведёт к трудностям при воспроизводстве и модификации (в случае ГП).

ПЭГ лишено указанных ограничений и обладает следующими преимуществами:
\begin{itemize}
  \item Простота хромосом: линейность, компактность, относительно небольшой размер, простота применения операторов, таких как репликация, мутация, рекомбинация, перенос и пр.
  \item Экспрессия (декодирование компактной структуры) генома порождает фенотип~--- синтаксическое дерево, которое может представлять математическую формулу либо программу. Значения вычисленной формулы, выполненной программы служат для численной оценки приспособленности особи с помощью функции фитнеса.
\end{itemize}

Таким образом, участие хромосомы в процессе воспроизведения зависит от фитнеса дерева, которое она кодирует. Для этого требуется универсальная система перевода формата генома в формат дерева и обратно, при которой любое изменение хромосомы с помощью операторов всегда приводит к синтаксически корректному дереву, обеспечивая приспосабливаемость и эволюционирование.



%--------------------------------------------------------------------



\subsection{Исходный алгоритм ПЭГ}

На рисунке~\ref{img:GEP_flowchart} показана блок-схема алгоритма ПЭГ~\cite{Ferreira2001}.

\begin{figure} [h]
\center
\begin{tikzpicture}[node distance = 5mm, every node/.style={transform shape}]
  \node [block] (create) {Создание начальной популяции};
  \node [block,    below = of create]   (express)  {Декодирование особей};
  \node [block,    below = of express]  (execute)  {Выполнение программ особей};
  \node [block,    below = of execute]  (evaluate) {Вычисление фитнеса особей};
  \node [decision, below = of evaluate] (condterm) {Останов};
  \node [block,    right = of condterm, xshift=1cm] (end) {Вернуть лучшую особь};
  \node [block,    below = of condterm, yshift=-1cm] (keepbest) {Копирование лучшей особи};
  \node [block,    below = of keepbest] (select)   {Отбор};
  \node [block,    below = of select]   (replica)  {Репликация};
  \node [block,    below = of replica]  (mutation)   {Мутация};
  \node [block,    below = of mutation] (transpos)   {Операторы переноса};
  \node [block,    below = of transpos] (recombin)   {Операторы рекомбинации};
  \path [line] (create) -- (express);
  \path [line] (express) -- (execute);
  \path [line] (execute) -- (evaluate);
  \path [line] (evaluate) -- (condterm);
  \path [line] (condterm) -- node [anchor=north] {Да}  (end);
  \path [line] (condterm) -- node [anchor=east] {Нет} (keepbest);
  \path [line] (keepbest) -- (select);
  \path [line] (select) -- (replica);
  \path [line] (replica) -- (mutation);
  \path [line] (mutation) -- (transpos);
  \path [line] (transpos) -- (recombin);
  \path [line] (recombin) -- ++(-4cm,0) |- (execute);
\end{tikzpicture}
\caption{Блок-схема исходного алгоритма ПЭГ}
\label{img:GEP_flowchart}
\end{figure}

Процесс начинается с конструирования популяции случайным образом созданных хромосом. После экспрессии (декодирования) хромосом каждая особь выполняется на заданном наборе входных данных (в виде координат точек для математических задач, таблиц истинности при поиске булевых выражений, наборов признаков при решении задач классификации и т.д.). На основании результатов выполнения особей вычисляется фитнес каждой из них. На основании значений фитнеса особи отбираются и подвергаются изменениям, нацеленным на получение потомства~--- популяции с новыми характеристиками. С этой популяцией проводятся те же операции: декодирование генома, отбор, воспроизведение с модификациями. Процесс повторяется заданное количество итераций либо до обнаружения решения.

Воспроизведение включает в себя не только репликацию, но также выполнение генетических операторов, вносящих разнообразие. Во время репликации геном особи в точности, без изменений копируется и переносится в новое поколение. Очевидно, этот оператор не способен вносить разнообразие, потому требуется вмешательство оставшихся операторов, случайным образом выбирающих особи, к которым будут применены. Таким образом, в ПЭГ особь может быть модифицирована сразу несколькими операторами, а может оказаться и не изменённой вовсе~\cite{ferreira:2001:wsc6Aa}.

%--------------------------------------------------------------------

\subsubsection{Открытые рамки считывания}

Структурная организация генома в ПЭГ более понятна в биологической терминологии открытых рамок считывания (open reading frames~--- ORF)~--- последовательностей генома, начинающихся стартовым кодоном, заверщающихся стоповым кодоном и содержащих кодирующие элементы между ними. В ПЭГ последовательность всегда начинается с первого элемента гена, но её конец не всегда совпадает с последним элементом. Таким образом, в ПЭГ нередка ситуация, когда участок от последнего элемента последовательности до конца гена не участвует в построении дерева, такие участки называют некодирующими, либо интронами.

В качестве примера рассмотрим следующее алгебраическое выражение:
\begin{equation}
\label{eq:GEP_sample_code_1}
\sqrt{(a+b)\times(c-d)}
\end{equation}

Выразить формулу~\eqref{eq:GEP_sample_code_1} можно в виде синтаксического дерева, показанного на рисунке~\ref{img:GEP_ET_sample_1}.
\begin{figure} [h]
  \center
  \begin{tikzpicture}[level distance=1cm,
    level 1/.style={sibling distance=3cm},
    level 2/.style={sibling distance=2cm},
    level 3/.style={sibling distance=1cm}]
    \tikzstyle{every node}=[-,thick]
    \node { $\sqrt{}$ }
      child[->] { node { $\times$ }
        child { node { $+$ }
          child { node { $a$ } }
          child { node { $b$ } }
        }
        child { node { $-$ }
          child { node { $c$ } }
          child { node { $d$ } }
        }
      }
      ;
  \end{tikzpicture}
  \caption{Пример синтаксического дерева}
  \label{img:GEP_ET_sample_1}
\end{figure}

Количество ответвлений (дочерних узлов) у узлов, представляющих функцию, равно количеству аргументов этой функции. Узлы, соответствующие терминалам (переменным и константам) не имеют дочерних узлов. Изображённое дерево и является фенотипом особи в ПЭГ, в то время как генотип записывается следующим образом (в верхнем ряду показаны индексы узлов, в нижнем~--- значения закодированных символов):

\begin{samepage}
\begin{verbatim}
01234567
Q*+-abcd,
\end{verbatim}
\end{samepage}

где очередность элементов обусловлена обходом графа в ширину. Это выражение является открытой рамкой считывания, в ПЭГ такие записи называются K-выражениями, а язык преобразования K-выражения в дерево и обратно~--- языком KARVA. Построение дерева из генома завершается, когда каждому ответвлению узла-функции поставлен в соответствие дочерний узел. Листья дерева, следовательно, являются терминалами.

Такая структура генома ПЭГ позволяет кодировать деревья различных размеров и форм, оперируя генами фиксированной длины, каждый из которых состоит из рамки считывания переменной длины и, при необходимости, некодирующего участка для заполнения пространства между концом кодирующей последовательности и концом гена. Хромосома особи состоит из одного или нескольких таких генов. Функция некодирующих участков, таким образом, состоит в обеспечении того, что длина рамки считывания всегда будет меньше или равна длине гена. Это позволяет гарантировать декодирование синтаксически правильного дерева после любых модификаций генома без ограничений и необходимости проведения сложных процедур проверки и редактирования. Отсутствие множества ограничений на генетические операторы является коренным отличием ПЭГ от~ГП.

%--------------------------------------------------------------------

\subsubsection{Геном в ПЭГ}

Каждый ген ПЭГ состоит из головы и хвоста. Голова содержит символы как функций, так и терминалов. Хвост содержит исключительно терминалы. Для каждой конкретной задачи подбирается длина головы гена $h$. Обозначим максимальное количество аргументов среди всех функций функционального множества как $n$, так для большинства арифметических функций $n=2$; для оператора <<ЕСЛИ ТО ИНАЧЕ>> $n=3$. Тогда маскимальную возможную длину хвоста гена $t$ можно вычислить по следующей формуле:

\begin{equation}
\label{eq:GEP_tail_size}
t(h,n) = h \times (n -1) + 1
\end{equation}

Следует отметить, что такая структура соответствует схеме кодирования путём обхода дерева в ширину, принятого в оригинальном алгоритме ПЭГ. Другие схемы кодирования, которые будут описаны ниже, не требуют описанного условного деления генома на голову и хвост.

Ещё один параметр, требующий подбора под каждую задачу~--- количество генов (участков кода фиксированной длины, содержащих одно дерево) в хромосоме. Обычно используется больше одного гена, такие хромосомы называют мультигенными. Длина всех генов устанавливается равной для простоты применения операторов. Хромосома представляет собой символьную строку, полученную путём конкатенации всех генов.

Таким образом, ген~--- объект, состоящий из пяти атрибутов: $G = (E, T, F, Op, S)$, где $E$~--- генотип, $T$~--- терминальное множество, $F$~--- функциональное множество, $Op$~--- множество операторов, $S$~--- значение фитнеса на определённом элементе данных. Пример полной записи гена:

\begin{verbatim}
("*++-aabcd", "abcd", "+-*/", 0)
\end{verbatim}

Хромосома~--- объект, состоящий из четырёх атрибутов: $C = (G, T, L, Op, S)$, где $G$~--- набор генов, $L$~--- связующий оператор. Пример полной записи хромосомы:

\begin{verbatim}
C1=({G0, G1, G2}, "ab", +, "+-*/", 0.7)
\end{verbatim}

Пример символьной записи хромосомы:

\begin{samepage}
\begin{verbatim}
012345678 012345678 012345678
-b*babbab *Qb+abbba -*Qabbaba
\end{verbatim}
\end{samepage}

Данная хромосома содержит три гена, каждый из которых начинается с позиций, обозначенных индексом 0. Окончание рамки считывания каждого гена можно определить лишь после построения закодированного в нём синтаксического дерева; в приведённом примере декодирование первого гена завершается на позиции~4 (последний элемент), второго~---~5, третьего~---~5.

Каждое дерево может использоваться как по отдельности, с расчётом фитнеса для каждого гена, так и в совокупности, формируя более сложное дерево, фитнес которого и отражает приспособленность особи. Во втором случае каждое под-дерево является отдельной сущностью, компонентом иерархической структуры, представляющей б\'{о}льшую ценность, чем сумма её частей. Независимая друг от друга эволюция генов хромосомы как отдельных блоков иерархической системы при мультигенном подходе позволяет более эффективно решать сложные задачи.

Взаимодействие под-деревьев происходит с помощью связующих функций, для алгебраических выражений это, как правило, функция арифметического суммирования, для булевых~--- логическое ИЛИ. Обычно связующая функция устанавливается априори, однако может также быть добавлена в геном и выбираться в процессе эволюции. На рисунке~\ref{img:GEP_ET_sample_2} показано синтаксическое дерево, декодированное из рассматриваемой трёхгенной хромосомы с априорно заданной связью с помощью функции суммирования.

\begin{figure} [h]
  \center
  \begin{tikzpicture}[level distance=1cm,
    level 1/.style={sibling distance=3cm},
    level 2/.style={sibling distance=2cm},
    level 3/.style={sibling distance=1cm}]
    \tikzstyle{every node}=[-,thick]
    \node { $+$ }
      child[->] { node { $G_0$ }
        child { node { $-$ }
          child { node { $b$ } }
          child { node { $\times$ }
            child { node { $b$ } }
            child { node { $a$ } }
          }
        }
      }
      child[->] { node { $G_1$ }
        child { node { $\times$ }
          child { node { $\sqrt{}$ }
            child { node { $+$ }
              child { node { $a$ } }
              child { node { $b$ } }
            }
          }
          child { node { $b$ } }
        }
      }
      child[->] { node { $G_2$ }
        child { node { $-$ }
          child { node { $\times$ }
            child { node { $a$ } }
            child { node { $b$ } }
          }
          child { node { $\sqrt{}$ }
            child { node { $b$ } }
          }
        }
      }
      ;
  \end{tikzpicture}
  \caption{Пример дерева мультигенной хромосомы}
  \label{img:GEP_ET_sample_2}
\end{figure}

Такой простой способ кодирования синтаксических деревьев в виде линейных структур может быть использован и в других видах эволюционных алгоритмов, например, в иммунных сетях, как это показано в работе~\cite{karakasis2008efficient}.

%--------------------------------------------------------------------

\subsubsection{Сведение к ГА}

Простейшим представлением генома в ПЭГ является ген, состоящий из одного терминала. Организация хромосомы, состоящей из одноэлементных генов делает ПЭГ эквивалентным ГА. Однако для решения комбинаторных задач требуются особые связывающие функции, отличные от простых арифметических и булевых. Например, для задачи коммивояжёра связь отображает расстояние между городами, представленными соответствующими генами. Если обозначить каждый город из девяти отдельным символом, то следующая хромосома из девяти элементов представит один из возможных путей обхода:

\begin{verbatim}
CADEBHFIG
\end{verbatim}

Если оптимизационная задача содержит $N$ классов терминалов, хромосома может состоять из $N$ групп одноэлементых генов, называемых мультигенными группами (МГГ~--- multigene family, MGF)~\cite{ferreira:2002:ASIA}. Например, при отображении одного шестиэлементного множества $\{1, 2, 3, 4, 5, 6\}$ в другое $\{A, B, C, D, E, F\}$ хромосома будет состоять из двух шестиэлементных групп:

\begin{samepage}
\begin{verbatim}
012345 012345
632451 EDFCBA
\end{verbatim}
\end{samepage}

Особенностью МГГ является то, что она не содержит повторяющихся элементов, а отображает перестановку элементов некоторого конечного множества. Потому операторы мутации и рекомбинации к такой структуре генома неприменимы из-за возможности некорректных последовательностей с повторяющимися и отсутствующими элементами. Следовательно, для внесения генетического разнообразия должны быть созданы новые операторы, порождающие корректные структуры.

Один из таких операторов~--- инверсия, впервые описанная Холландом в 1975 году. Случайным образом выбирается участок МГГ и изменяется порядок следования элементов в нём. Так, например, результатом применения к хромосоме из предыдущего примера может стать следующий геном:

\begin{samepage}
\begin{alltt}
012345 012345
632451 E{\underline{CFD}}BA
\end{alltt}
\end{samepage}

Из проведённых экспериментов следует, что оператор инверсии обладает более сильной изменяющей способностью и, следовательно, наиболее способствует скорейшему обнаружению решения.
Оператор перемещения гена располагает случайным образом выбранный ген по другой позиции, сохраняя порядок следования остальных генов группы. Использование данного оператора вместе с более мощными, такими как инверсия, может быть полезно для точной подстройки решения. Однако эксперименты показали~\cite{ferreira:2002:ASIA}, что инверсия, используемая в одиночку, более эффективна.

Оператор перестановки генов меняет местами два случайным образом выбранных гена в пределах МГГ. Исследование оператора перемещения участков гена показало его крайнюю неэффективность~--- особи-потомки имели, как правило, худший фитнес, чем родительские особи.

%--------------------------------------------------------------------

\subsubsection{Иерархическая структура дерева}

В исходном алгоритме ПЭГ гены мультигенной хромосомы являются не связанными между собой под-деревьями, объединяемыми связующим оператором (например, арифметическим сложением). Каждый из генов задаёт свою собственную функцию, обнаруженную эволюционным алгоритмом. В терминологии ПЭГ эти функции называются автоматически определёнными (ADF~--- automatically defined function). Такие обособленные функции могут использоваться как законченный блок при построении итогового синтаксического дерева. Для этого требуется усложнить структуру хромосомы~--- выстроить иерархию генов. В такой сложной хромосоме первые $N$ генов задают функции, а следующие $M$ генов представляет собой клетки~--- деревья, терминальное множество которых состоит из $N$ терминальных символов, обозначающих функции первых $N$ генов хромосомы. Каждая клетка~--- это отдельный способ комбинации генов-функций.
Рассмотрим на примере хромосомы, состоящей из трех генов-функций и двух клеток:

\begin{samepage}
\begin{verbatim}
0123456 0123456 0123456 012345678 012345678
*Q-bbab Q*baabb -/abbab *+21Q1102 /*21+1011
\end{verbatim}
\end{samepage}

\begin{figure} [h]
  \center
  \begin{tikzpicture}[node distance = 3cm, level distance=1cm,
    level 1/.style={sibling distance=3cm},
    level 2/.style={sibling distance=2cm},
    level 3/.style={sibling distance=1cm}]
    \tikzstyle{every node}=[-,thick]
    \node (g0) { $G_0$ }
      child[->] { node { $\times$ }
        child { node { $\sqrt{}$ }
          child { node { $b$ } }
        }
        child { node { $-$ }
          child { node { $b$ } }
          child { node { $a$ } }
        }
      };
    \node[right = of g0] (g1) { $G_1$ }
      child[->] { node { $-$ }
        child { node { $/$ }
          child { node { $b$ } }
          child { node { $b$ } }
        }
        child { node { $a$ } }
      };
    \node[right = of g1 ] (g2) { $G_2$ }
      child[->] { node { $\times$ }
        child { node { $+$ }
          child { node { $G_1$ } }
          child { node { $\sqrt{}$ }
            child { node { $G_0$ } }
          }
        }
        child { node { $G_0$ } }
      };
    \node[right = of g2] { $Cell_0$ }
      child[->] { node { $/$ }
        child { node { $\times$ }
          child { node { $G_1$ } }
          child { node { $+$ }
            child { node { $G_1$ } }
            child { node { $G_0$ } }
          }
        }
        child { node { $G_2$ } }
      }
      ;
  \end{tikzpicture}
  \caption{Пример многоклеточного дерева}
  \label{img:GEP_ADF_sample}
\end{figure}

Многоклеточные хромосомы могут быть по-разному использованы при решении задач: либо как отдельные выходы (например, отражающие различные классы в задаче многоклассовой классификации), либо либо задействование какой-либо одной (обычно лучшей) из них в качестве единственного выхода.

В ходе экспериментов~\cite{Ferreira:2006:GSP} было выяснено, что при решении простых задач наиболее эффективной показала себя структура хромосомы, состоящая всего из одного гена-функции и одной клетки.

Для повышения выразительной силы мультигенной хромосомы можно применить другой, более простой вариант иерархической структуры: любой ген (кроме первого по порядку) может использовать значения генов, идущих в хромосоме перед ним~\cite{Dai:2008:MNE:1473243.1473311}. Для этого так же используются терминальные символы, соответствующие индексу гена, на значение которого ведёт ссылка. Продемонстрировать преимущества такого подхода можно на примере функции $a^{2^n}$ при функциональном множестве $\{+, -, \times, /\}$. Традиционное кодирование ПЭГ требует в этом случае $2^{n+1} - 1$ символ: $2^n$ символа <<$a$>> и $2^n - 1$ символов <<$*$>>, строка генома выглядит так: <<\verb|*******aaaaaaaa|>>. При подходе MERGE (Multi ExpRession GEne programming~--- ПЭГ с несколькими выражениями) для этого требуется всего $3 \times n$ символов и хромосома из трёх генов:

\begin{samepage}
\begin{verbatim}
0: *aa
1: *00
2: *11,
\end{verbatim}
\end{samepage}

где 0, 1~--- терминальные символы, кодирующие ссылки на гены с индексами~0~и~1. Для гарантии отсутствия циклических ссылок ген~0 не может ссылаться на другие, ген~1 может ссылаться на ген~0, ген~2~--- на гены~0~и~1, ген~3~--- на гены~0,~1~и~2, и т.д.

Кроме того, такая структура мультигенной хромосомы позволяет использовать значения каждого из своих генов как самостоятельный геном особи, имеющий собственное значение фитнеса. Максимальный из фитнесов генов выбирается фитнесом всей хромосомы.

Экспериментальная проверка подхода ПЭГ с несколькими выражениями показала его успешность в решении задач даже короткими генами с длиной головы в 1--2~элемента. Помимо этого, выяснилось, что установка длины гена выше оптимальной не ведёт к падению производительности алгоритма, в то время как чрезмерная сложность особей, кодируемых длинным геномом в исходном ПЭГ, не позволяла быстро находить приемлемые решения.

В работе~\cite{Li:2008:GNF:1473248.1474006} предложена другая модификация автоматически определяемых функций, описанных в~\cite{Ferreira:2006:GSP}. Символ ссылки на ген представляет собой уже не терминал, отображающий значение запрашиваемого гена для данного набора данных, а элемент особого функционального множества, декодируемого как узел дерева с дочерними узлами~--- аргументами функции. Отличие состоит в том, что запрашиваемый ген оперирует аргументами, подаваемыми геном-клеткой, а не переменными выборки данных. Модификация позволила~\cite{Li:2008:GNF:1473248.1474006} на треть ускорить сходимость.



%--------------------------------------------------------------------


\subsection{Методика тестирования}

Для проведения экспериментов была реализована система ПЭГ на языке программирования C, принимающая в качестве аргумента файл с набором точек (заданных координатами) и выдающая лучшую обнаруженную формулу, набор точек~--- модель, полученную с использованием этой формулы,~--- и динамику СКО лучшей особи с течением времени. В исходный алгоритм затем поочерёдно добавлялись описанные далее модификации, с возможностью включения и выключения каждой из них в любых комбинациях.  Так, например, при каждом запуске можно выбрать оператор селекции (рулетка, отбор по плотности либо турнир), вероятность мутации, способ кодирования и т.п.

Рандомизированная природа алгоритма ПЭГ приводит к необходимости использовать статистические методы для оценки его эффективности. Для этого каждая исследуемая комбинация модификаций алгоритма запускалась множество раз (обычно 100), анализу подвергались значения СКО лучших моделей, полученных в ходе каждого запуска. Наиболее интересными показателями являются: СКО наилучшей модели среди всех запусков $e_{b}$ и доля успешных запусков $r_{f}$~--- т.е. таких запусков, в ходе которых была получена модель с приемлемой для данной задачи точностью.

Было обнаружено, что в большинстве случаев подбор комбинаций параметров и модификаций позволяет добиться улучшения какой-либо одной из характеристик $e_{b}$ и $r_{f}$, но не обеих одновременно. Направленность на снижение $e_{b}$ наиболее оправдано в тех случаях, когда процесс поиска модели не стеснён во времени, а целью является получение как можно более точной модели. Повышение доли успешных запусков $r_{f}$ является целью при поиске оптимального алгоритма решения задач, требующих быстрого построения моделей с приемлемой заданной точностью.

Таким образом, $e_{b}$ отражает способность алгоритма к обнаружению лучшей из возможных моделей, а $r_{f}$~--- вероятность получения модели достаточной точности при очередном запуске алгоритма.

Сравнение систем, собранных с различными параметрами, проводилось на трёх задачах поиска формул, описывающих такие модели:
\begin{enumerate}
  \item $y(x) = \sin x, x \in [-5, 4.6]$;
  \item Функция Розенброка: $z(x, y) = {(1 - x)}^2 + 100 {(y - x^2)}^2, x \in [-3, 3], y \in [-3, 3]$;
  \item Сумма четырёх сигмоид:
    \begin{multline}
      z(x, y) = \frac{1}{2} \times \exp\left(-\frac{(9x-2)^2 - (9y-2)^2}{4}\right) + \frac{3}{4} \times \exp\left(-\frac{(9x+1)^2}{49} + \frac{(9y+1)^2}{10}\right) + \\
      + \frac{1}{2} \times \exp\left(-\cfrac{(9x-7)^2 + (9y-3)^2}{4}\right) -\frac{1}{5} \times \exp\left(-(9x-4)^2 - (9y-7)^2\right), \\
      x \in [-2, 2], y \in [-2, 2]
    \end{multline}
\end{enumerate}

При моделировании функций синуса и Розенброка использовалось функциональное множество, состоящее из четырёх арифметических действий, в последнем случае~--- полное множество, в состав которого входят арифметические (сложение, вычитание, умножение, деление), тригонометрические ($\sin$, $\cos$, $\tan$, $\sinh$, $\cosh$, $\tanh$), степенн\'{ы}е (возведение в степень, взятие квадратного корня, экспонента, логарифм). Полученные с применением этих формул модели изображены на рисунке~\ref{img:functionz}. Значения приемлемых СКО моделей, при которых задача будет считаться решённой, составляют для этих задач, соответственно, $0.08$, $0.05$ и $0.06$.

\begin{figure} [h]
  \center
  \begin{subfigure}[t]{0.3\linewidth}
    \center
    \begin{tikzpicture}[domain=-5:4.6]
      \begin{axis}[width=\linewidth, title={$y=\sin{x}$}, no marks]
        \addplot{sin(deg(x))};
      \end{axis}
    \end{tikzpicture}
  \end{subfigure}
  \begin{subfigure}[t]{0.3\linewidth}
    \center
    \begin{tikzpicture}
      \begin{axis}[width=\linewidth, title={$z=Rosenbrock(x, y)$}]
        \addplot3[mesh, domain=-3:3]{(1 - x)^2 + 100 * (y - x^2)^2};
      \end{axis}
    \end{tikzpicture}
  \end{subfigure}
  \begin{subfigure}[t]{0.3\linewidth}
    \center
    \begin{tikzpicture}
      \begin{axis}[width=\linewidth, title={$z=sigmoids(x, y)$}]
        \addplot3[mesh, samples=40, domain=-1.5:1.5]{0.5 * e^(- (((9*x-2)^2) + ((9*y-2)^2)) / 4) + 0.75 * e^(- ((9*x+1)^2) / 49 - ((9*y+1)^2) / 10) + 0.5 * e^(- (((9*x-7)^2) + (9*y-3)^2) / 4) -0.2 * e^(-  ((9*x-4)^2) - (9*y-7)^2)};
      \end{axis}
    \end{tikzpicture}
  \end{subfigure}
  \caption{Тестовые модели}
  \label{img:functionz}
\end{figure}



%--------------------------------------------------------------------


\subsection{Схемы кодирования дерева}

\subsubsection{Префиксное кодирование}

В исходном алгоритме ПЭГ декодирование дерева производится путём линейного извлечения элементов генома и установки в узлы формируемого дерева при его обходе в ширину. При замене обхода дерева в ширину обходом в глубину повышается связность генома~--- уменьшается расстояние между элементами, соответствующими соседним узлам дерева, а под-дерево кодируется непрерывной строкой в геноме. Такой подход способствует сохранению субструктур, что делает менее разрушительным применение операторов скрещивания~\cite{Li:gecco05lbp}. Сохранение в популяции субструктур (суб-деревьев) лучших особей и их рекомбинация позволяет ускорить сходимость алгоритма: оптимальное решение обнаруживается на более ранних поколениях (в несколько раз быстрее), качество оптимального решения зачастую выше, чем результат работы оригинального алгоритма ПЭГ.

Однако при таком подходе разрушается гарантия получения синтаксически корректного дерева из любого генома, сформированного согласно правилам. Потому требуется ввести процедуру динамической валидации дерева. К участию в эволюционном процессе допускаются только те особи, декодирование которых порождает корректное дерево. Такая система получила название P-GEP.

От предыдущих авторов. Предложенная линейная структура генома делает возможнным легко выделять из него отдельные под-деревья, что, в свою очередь, позволяет производить поиск повторяющихся субструктур. Появление часто встречающихся субструктур в лучших особях популяции означает их высокую полезность при составлении конечного решения. Выделив набор таких субструктур популяции авторами предлагается формировать из них т.н. элитную группу~\cite{Substructures(ICMLA05)_XLi}. Каждый элемент этой группы обозначается одним дополнительным терминальным символов.

Для взаимодействия между особями и элитной группой были введены два новых генетических оператора: оператор сжатия, заменяющий последовательность в геноме особи соответствующим терминальным символом из элитной группы, и оператор раскрытия, выполняющий обратную операцию~--- замену терминального символа на соответствующую последовательность из элитной группы.
Внесение данной модификации в P-GEP привело к обнаружению решений, имеющих значительно меньшую погрешность.

%--------------------------------------------------------------------

\subsubsection{Кодирование с наложениями}

В работе~\cite{conf/icnc/PengTZY05} рассматривается разновидность декодирования генома путём обхода дерева в глубину, при котором последовательно идущие под-деревья накладываются друг на друга. Элементы генома последовательно считываются:
\begin{itemize} \itemsep0pt \parskip0pt \parsep0pt
  \item Если текущий символ принадлежит к терминальному множеству~--- поместить его в синтаксическое дерево как листовой узел.
  \item Если текущий символ принадлежит к функциональному множеству~--- поместить его в синтаксическое дерево как узел с количеством дочерних узлов, равным количеству аргументов помещаемой функции. Первый дочерний узел~--- следующий в геноме за текущим, второй~--- следующий за ним и т.д. Дочерние под-деревья заполняются следуя этой же процедуре.
  \item При выходе за пределы генома в качестве текущего символа берётся первый элемент терминального множества.
\end{itemize}

Коренное отличие описанного способа декодирования коренным образом отличается от методов обхода в ширину и обхода в глубину тем, что элементы генома могут быть задействованы в построении дерева неоднократно, и размер получаемого дерева, как правило, больше при равном размере генома.

Рассмотрим на примере строки <<\verb|**a+*bc|>>, синтаксическое дерево, образующееся при её декодировании, показано на рисунке~\ref{img:GEP_ET_sample_EAOGE}.

\begin{figure} [h]
  \center
  \begin{tikzpicture}[level distance=1cm,
    level 1/.style={sibling distance=3cm},
    level 2/.style={sibling distance=2cm},
    level 3/.style={sibling distance=1cm}]
    \tikzstyle{every node}=[-,thick]
    \node { $\times$ }
      child[->] { node { $\times$ }
        child { node { $a$ } }
        child { node { $+$ }
          child { node { $\times$ }
            child { node { $b$ } }
            child { node { $c$ } }
          }
          child { node { $b$ } }
        }
      }
      child[->] { node { $a$ } }
      ;
  \end{tikzpicture}
  \caption{Пример дерева, декодированного с наложениями}
  \label{img:GEP_ET_sample_EAOGE}
\end{figure}

При использовании традиционного метода декодирования, принятого в ПЭГ, данному дереву соответствует строка <<\verb|**aa+*bbc|>>, содержащая на 2 символа больше.

Выразительная сила генома при декодировании с наложениями значительно выше, что показано в таблице~\ref{tbl:EAOGE_expression_power}, отражающей соответствие максимального возможного размера (количества узлов) синтаксического дерева при равных размерах гена.

\begin{table}[h]
  \caption{Сравнение выразительной силы ПЭГ с наложениями}
  \label{tbl:EAOGE_expression_power}
  \begin{center}
    \begin{tabular}{|c|c|c|}
      \hline
      Размер гена & ПЭГ & ПЭГ с наложениями \\
      \hline
      3 & 3 & 9 \\
      5 & 5 & 25 \\
      7 & 7 & 67 \\
      9 & 9 & 177 \\
      11 & 11 & 465 \\
      13 & 13 & 1219 \\
      15 & 15 & 3193 \\
      17 & 17 & 8361 \\
      \hline
    \end{tabular}
  \end{center}
\end{table}

Повышение выразительной силы генома позволяет использовать более короткие хромосомы, что, в свою очередь, значительно понижает вычислительную сложность алгоритма: разница в скорости обработки экспоненциально растёт, при длине хромосомы в 23~элемента ускорение достигает 10~раз. Авторы подхода назвали систему на его основе EAOGE~--- Evolutionary Algorithm based on Overlapped Gene expression~--- эволюционный алгоритм на осове ПЭГ с наложениями. 

Сравнение эффективности реализованной системы ПЭГ с данной модификацией, исходным алгоритмом и ПЭГ с префиксным кодированием приведено в таблице~\ref{tbl:cmp_tree_depths}, для возможности сопоставления производительности алгоритмов было зафиксировано время, выделяемое каждому запуску.

При небольшой длине генома ПЭГ с наложениями существенно превосходит конкурирующие модификации в способности обнаруживать отличные решения. Дальнейшее увеличение максимально возможного размера дерева до глубины в 4~узла несколько меняет картину: префиксное ПЭГ значительно выигрывает в вероятности успешного обнаружения решения, в то время как выходом ПЭГ с наложениями является лучшая модель. При допущении деревьев глубиной в 5~узлов наилучшее показатели имеет префиксное ПЭГ, показатели всех алгоритмов достигают своего максимума.

Таким образом, обе модификации превосходят исходный алгоритм в производительности, однако каждая из них занимает свою нишу, потому выбор одной из них определяется в зависимости от нужд исследователя.

\input{cmp_tree_depths}

%--------------------------------------------------------------------

\subsubsection{Нейтральные гены}

После декодирования хромосомы ПЭГ и конструирования синтаксического дерева количество узлов полученного дерева в предельном случае будет равно размеру хромосомы, однако в большинстве случаев будет меньшим. Те элементы в хвостах генов, которые не были использованы при построении дерева, называются некодирующими участками. Кроме того, полученное дерево может содержать выражения вида $(a - a)$, $(a \times 0)$, $(a - 0)$ и т.п., которые при семантическом анализе упрощаются, устраняя данные интроны и образуя более компактное дерево. Множество подобных участков~--- часто повторяющихся последовательностей, интронов, псевдогенов и пр.~--- встречается в природе в ДНК живых существ. Потому логично предположить, что и в искусственном геноме нейтральные участки могут быть полезны. Среднее количество нейтральных последовательностей в популяции растёт с увеличением как количества генов в хромосоме, так и размера каждого из них.

Исследования работы алгоритма с различной длиной генома показали~\cite{journals/advcs/Ferreira02}, что эволюционирование особей с высокой степенью избыточности генома происходит эффективней. Это связано с тем, у каждого решения, представленного в наиболее компактной форме, существует множество менее компактных вариантов представления, увеличенных в размерах за счёт интронов. Кроме того, накапливаемые в некодирующих участках мутации могут быть задействованы в дальнейшем, если изменения в голове гена (добавление функциональных узлов) приведут к увеличению размера дерева. В таблице~\ref{tbl:cmp_tree_depths} приведены результаты моделированя с различной длиной генома. Удлиннение генома ведёт к возрастанию пространства поиска, появлению деревьев б\'{о}льшего размера, следовательно, ведёт к более длительному расчёту популяции. Имеет смысл сравнивать модели, на построение которых было затрачено равное время. В приведённой таблице видно, что для каждой задачи имеется некая оптимальная длина генома. Меньший размер генов не позволяет выразить деревья достаточной сложности, а при б\'{о}льшем требуется на порядки больше вычислительного времени.



%--------------------------------------------------------------------



\subsection{Создание начальной популяции}

В изученных работах, посвящённых ПЭГ, не уделяется внимания процедуре создания начальной популяции, а именно конструированию случайной особи. Указывается, что строка хромосомы обходится с начала и по порядку, и каждый символ задаётся случайным образом: тип узла может быть равновероятно задан как функциональный элемент, как переменная (подаваемая на вход модели) и как числовая константа. Очевидно, что размер дерева определяется количеством подряд (близко) идущих функциональных узлов, начиная с корневого узла (первого символа хромосомы): из условия равновероятности типов узлов (3 типа~--- 33.333\%~каждый) следует, что вероятность создания дерева, состоящего всего из одного элемента (терминала: переменной либо константы), составляет~66.67\%. При этом можно заметить, что средняя длина получаемых деревьев не зависит от задаваемой длины хромосомы, но определяется установленной вероятностью появления того или иного типа.

При оригинальном кодировании элементов путём обхода узлов дерева в ширину логично предположить, что вероятность появления функциональных узлов в головной части хромосомы следует повысить, в то время как в хвостовой части могут быть помещены исключительно терминальные узлы. В таком случае определение оптимальных численных значений вероятностей представляет собой отдельную задачу для исследования. Кроме того, данный подход не применим к другим способам кодирования, потому не обладает должной универсальностью.

Для решения указанной проблемы и закрепления взаимосвязи между размером дерева и длиной генома была разработана следующая процедура инициализации. Обход строки так же производится посимвольно, начиная с её начала (независимо от способа кодирования). При этом не составляет трудности определить местонахождение кодируемого символом узла в дереве, а именно его глубину: первый символ является корневым узлом, далее~--- в соответствии со способом кодирования.

Вырожденное дерево, состоящее из одного терминала, как правило, не представляет интереса, так как решаемая таким образом задача не требует применения ПЭГ. В таком случае вероятность задания функционального типа корневому узлу устанавливается равным~100\%. В тоже время для ограничения размера дерева заданным значением (несмотря на способность префиксного кодирования и ПЭГ с наложениями создания деревьев, превышающих в размере длину генома) в целях удобства пользования уместно ограничить узлы максимальной возможной глубины листовыми. Следовательно, вероятность появления функционала в указанных местах равна нулю. Требует решения задача распределения вероятности функциональных узлов в зависимости от глубины вложенности.

Предлагается использование следующей формулы:

\begin{equation}
\label{eq:init_func_prob}
P_{func}(level, maxlevel) = \sqrt{1 - \left(\frac{level - 1}{maxlevel - 1}\right)^2}
\end{equation}

где $level$~--- глубина узла, $maxlevel$~--- максимально возможная глубина дерева. Пример такого распределения показан на рисунке~\ref{img:init_func_prob}.

\begin{figure} [h]
  \center
  \begin{tikzpicture}[domain=1:5]
  	\begin{axis}[xlabel={$level$}, ylabel={$P_{func}(level, 5)$}, width=0.4\linewidth, no marks]
      \addplot{sqrt(1 - ((x - 1) / (5 - 1))^2)};
    \end{axis}
  \end{tikzpicture}
  \caption{Вероятность появления функционального узла в зависимости от глубины узла в дереве глубиной 5}
  \label{img:init_func_prob}
\end{figure}

Таким образом, процедура инициализации задает узлу глубины~$level$ тип функции при выполнении условия $random(0\ldots1) < P_{func}(level, maxlevel)$, обеспечивая сбалансированную начальную популяцию особей прогнозируемого размера. Данный метод конструирования новой особи случайным образом показал свою эффективность и был использован во всех экспериментах.



%--------------------------------------------------------------------


\subsection{Выводы}

В данной статье был проанализирован алгоритм ПЭГ, рассмотрены различные схемы кодирования синтаксических деревьев для их последующей линеаризации и хранения, описана детальная процедура создания начальной популяции, которой не было уделено внимания в изученных работах. Была разработана универсальная методика, позволяющая определить производительность модификаций алгоритма по любым метрикам, две из которых были отобраны как представляющие наибольший интерес. Сравнение схем кодирования выявило их эффективное применение.


\clearpage
\section{Статья 2: Исследование генетических операторов в программировании с экспрессией генов}

\subsection{Введение}

В ПЭГ применяются следующие операторы: репликации, мутации, копирования со сдвигом, копирования со двигом в корень, перестановки генов, одно- и двухточечной рекомбинации, генной рекомбинации. Рассмотрим каждый из них.

Репликация представляет собой самый простой оператор, не вносящий разнообразие в геном. Используется в паре с оператором отбора для копирования особей в новую популяцию в соответствии со значением фитнеса и случайностью, вводимой оператором рулетки.

Мутация (замена символа~--- элемента гена, соответствующего узлу дерева) может возникнуть в любом месте хромосомы. Вероятность оператора, как правило, задаётся эквивалентной двум мутациям в хромосоме. Элементы, подвергающиеся мутации в голове гена, могут быть изменены на любой другой символ (функцию или терминал), в хвосте~--- только на терминал. Данное ограничение нацелено на сохранение структуры хромосомы и обеспечение декодирования синтаксически верного дерева. Среди всех модифицирующих операторов мутация является самым важным и эффективным, т.к. в отличие от остальных, комбинирующих существующие участки генома, мутация направлена на создание новых элементов, а потому вносит наиболее радикальные изменения, расширяя область поиска решений.

В ходе анализа существующих методов и средств, направленных на ускорение получения модели алгоритмом ПЭГ и повышение её качества, а также проведённых экспериментов был выявлен ряд ограничений производительности алгоритма, таких как длительность вычисления фитнеса, отсутствие тонкой подстройки числовых констант, вляние размера хромосомы на скорость сходимости, проблемы с поиском сложных моделей. Далее описаны методы и подходы, созданные в ходе преодоления данных ограничений.

Алгоритм ПЭГ заключается в выполнении следующих этапов:
\begin{enumerate} \itemsep0pt \parskip0pt \parsep0pt
  \item Создание начальной популяции
  \item Декодирование особей
  \item Выполнение программ особей
  \item Вычисление фитнеса особей
  \item Проверка критерия останова алгоритма (достигнута требуемая точность, либо истекло лимитированное время выполнения)
  \item Копирование лучшей особи (Элитизм)
  \item Отбор
  \item Репликация
  \item Мутация
  \item Операторы переноса
  \item Операторы рекомбинации
  \item Возврат к пункту 2
\end{enumerate}


%--------------------------------------------------------------------


\subsection{Методика тестирования}

В предыдущей статье цикла [ссылка на статью 1] описана методика оценки эффективности модификаций алгоритма ПЭГ, заключающаяся в статистической обработке результатов множества запусков с целью получения таких основных метрик: СКО наилучшей модели среди всех запусков ($e_{b}$) и доля успешных запусков ($r_{f}$)~--- таких, в ходе которых была получена модель с приемлемой для данной задачи точностью.

Тестирование производилось на следующих задачах:
\begin{enumerate}
  \item $y(x) = \sin x, x \in [-5, 4.6]$;
  \item Функция Розенброка: $z(x, y) = {(1 - x)}^2 + 100 {(y - x^2)}^2, x \in [-3, 3], y \in [-3, 3]$;
  \item Сумма четырёх сигмоид:
    \begin{multline}
      z(x, y) = \frac{1}{2} \times e^{-\left\lbrack\cfrac{(9x-2)^2 + (9y-2)^2}{4}\right\rbrack} + \frac{3}{4} \times e^{-\left\lbrack\cfrac{(9x+1)^2}{49} + \cfrac{(9y+1)^2}{10}\right\rbrack} + \\
      + \frac{1}{2} \times e^{-\left\lbrack\cfrac{(9x-7)^2 - (9y-3)^2}{4}\right\rbrack} -\frac{1}{5} \times e^{-\left\lbrack(9x-4)^2 + (9y-7)^2\right\rbrack}, \\
      x \in [-2, 2], y \in [-2, 2]
    \end{multline}
\end{enumerate}



%--------------------------------------------------------------------



\subsection{Операторы отбора}

Процесс эволюции основан на генетической изменчивости и процедуре отбора. Эти же механизмы задействованы во всех эволюционных алгоритмах. Однако способ обеспечения генетической изменчивости, который можно было бы назвать лучшим, не был выявлен. Исследователи разделяют эти способы на мутацию и рекомбинацию. Успешность эволюции также сильно зависит от применяемых алгоритмом генетических операторов, размера и состава начальной популяции.

Сравнивая эффективность исходного алгоритма ПЭГ при разных размерах популяции следует учесть, что время выполнения алгоритма прямо пропорционально размеру популяции. Для оценки данного фактора в условиях ограниченного времени время каждому запуску было отведено фиксированное время работы, потому алгоритмы с б\'{о}льшим размером популяции имели возможность расчитать меньшее число поколений. Результаты продемонстрированы в таблице~\ref{tbl:cmp_pop_sizes}, из которых также следует наличие пика эффективности при размере популяции 80--120~особей.

\input{cmp_pop_sizes}

Одно из важнейших применений ПЭГ~--- символьная регрессия, цель которой~--- поиск выражения, наиболее соответствующего известным заданным значениям с определённой погрешностью. Установка небольшого значения (абсолютного или относительного) этой погрешности позволит обнаружить хорошие решения некоторых математических задач, однако в большинстве случаев маленькая допустимая погрешность замедляет эволюционирование по причине более строгого отбора особей. Если же задать слишком большую допустимую погрешность, отбор пройдёт множество особей, не являющихся приемлемыми решениями, однако значение фитнеса которых будет очень высоким.

Отбор особей в ПЭГ осуществляется пропорциональным оператором рулетки с простым элитизмом: лучшая особь популяции переносится в следующее поколение, шансы остальных прямо пропорциональны их фитнесу. Такая форма элитизма гарантирует, что лучшее обнаруженное решение не будет утрачено. Каждый запуск рулетки выбирает одну особь, соответственно, количество запусков рулетки равно размеру популяции.
После отбора новой популяции поочерёдно выполняются генетические операторы, применяемые к случайным образом выбранным особям. Например, если вероятность оператора составляет~70\%, то~7 из~10 особей будут им модифицированы. Каждая особь может быть модифицирована сразу несколькими операторами, однако оператор может быть применён к особи лишь однократно. Это отличает ПЭГ от~ГП, где особь не изменяется более чем одним оператором за итерацию. Получаемое таким образом потомство существенно отличается от родительских особей.

Наиболее универсальный оператор отбора с достаточной эффективностью, применяемый в ПЭГ~--- алгоритм рулетки, в котором вероятность дальнейшего участия особи в процессе эволюции напрямую определяется её фитнесом, что ведёт к сохранению только особей с высоким фитнесом. Однако если фитнес одной из особей популяции в определённый момент значительно превышает фитнес остальных, это приводит к застреванию алгоритма в локальном оптимуме и потере множества особей с тенденцией к улучшению, но небольшим текущим значением фитнеса.

В работе~\cite{conf/icnc/PengTZY05} предложена формула отбора, почерпнутая из иммунных алгоритмов, основанная на понятии плотности $D$ особи:

\begin{equation}
\label{eq:EAOGE_density}
D(I_k) = \frac{1}{\sum\limits_{i=1}^N{|f(I_k) - f(I_i)|}}, k=1,2,\ldots,N,
\end{equation}
где $N$~--- количество особей в популяции, $f$~--- функция фитнеса. Значения плотности используются затем для вычисления вероятности отбора:

\begin{equation}
\label{eq:EAOGE_probability}
P(I_k) = \frac{\sum\limits_{i=1}^N{D(I_k)}}{D(I_k)}, k=1,2,\ldots,N
\end{equation}

Таким образом, чем больше особей, похожих (по критерию вероятности) на особь $I_k$, тем меньшую вероятность отбора она имеет. Авторами не отмечается тот факт, что особи с принципиально разными синтаксическими деревьями, но равным значением фитнеса будут неотличимы друг от друга при данном подходе. Высказано предположение о том, что данная формула отбора гарантирует разнообразие генетического материала в популяции.

Сравнение различных подходов к механизму отбора предоставлено в таблице~\ref{tbl:cmp_probability_density_selection_and_tournament}. Во всех случаях осуществление простого элитизма осуществлялось путём сохранения лучшей особи. Анализ этих результатов позволяет сделать определённые выводы. Наиболее общим является целесообразность использования популяции размером в 50~особей: это экспериментально выведенное оптимальное значение характерно для всех девяти проведённых экспериментов. Исходный алгоритм ПЭГ (с кодированием путём обхода графа в ширину) достигает максимума своей производительности при использовании турнирного отбора. Для ПЭГ с наложениями таковым является отбор по плотности. Глобально же наилучшие результаты показывает префиксное ПЭГ в комбинации с оператором рулетки.

Таким образом, рассматривая операторы отбора вне зависимости от способа кодирования, наиболее оптимальным будет выбор оператора рулетки, как и было предложено автором оригинального алгоритма ПЭГ.

\input{cmp_probability_density_selection_and_tournament}



%--------------------------------------------------------------------



\subsection{Оператор мутации}

\subsubsection{Влияние на динамику эволюции}

В системах на основе генотипа и фенотипа пространство поиска отделено от пространства решений, что существенно улучшает производительность таких систем. Чем меньше ограничений накладывается на процедуру проецирования генотипа на фенотип и обратно~--- тем более эффективна система, т.к. практически любой оператор может быть использован для исследования пространства поиска, включая мутацию.

Так, например, в системе DGP (Developmental Genetic Programming~--- ГП с развитием~\cite{poli2008field}) в результате трансляции генотипа в фенотип не всегда получается синтаксически верное дерево, что приводит к необходимости проведения дополнительных операций по устранению непригодных особей. Потому мутация в DGP не превосходит по показателям операторы кроссовера.

Для сравнения эффективности каждого из генетических операторов, применяемых в ПЭГ, по отдельности были сопоставлены динамические данные процесса эволюции: графики фитнеса лучшей особи с течением поколений (итераций), и графики среднего фитнеса популяции~\cite{ferreira:2002:FEA}.

В ходе поиска модели, созданной формулой $y(x) = x^4 + x^3 + x^2 + x$ была исследована динамика эволюционного процесса с различными значениями вероятности применения оператора мутации ($p_{m_1}, p_{m_2}, p_{m_3}, p_{m_4}$), значения фитнеса приведены на рисунке~\ref{img:cmp_health}.

\begin{figure} [h]
  \center
  \begin{subfigure}[h]{0.3\linewidth}
    \center
    \begin{tikzpicture}
      \begin{axis}[title={а) $p_{m_1} = 0$}, width=\linewidth, no marks, ymax = 1000, ylabel={Фитнес}, xlabel={Время, мс}, legend entries={{Лучший фитнес},{Средний фитнес}}, legend to name=legend_local_refcmp_health, legend columns=2]
        \addplot[mark=x,red] table[x=time,y=best] {data_health_0};
        \addplot[mark=*,blue] table[x=time,y=mean] {data_health_0};
      \end{axis}
    \end{tikzpicture}
  \end{subfigure}
  \begin{subfigure}[h]{0.3\linewidth}
    \center
    \begin{tikzpicture}
      \begin{axis}[title={б) $p_{m_2} > p_{m_1}$}, width=\linewidth, no marks, ymax = 1000, ylabel={Фитнес}, xlabel={Время, мс}]
        \addplot[mark=x,red] table[x=time,y=best] {data_health_1};
        \addplot[mark=*,blue] table[x=time,y=mean] {data_health_1};
      \end{axis}
    \end{tikzpicture}
  \end{subfigure}
  \\
  \begin{subfigure}[h]{0.3\linewidth}
    \center
    \begin{tikzpicture}
      \begin{axis}[title={в) $p_{m_3} > p_{m_2}$}, width=\linewidth, no marks, ymax = 1000, ylabel={Фитнес}, xlabel={Время, мс}]
        \addplot[mark=x,red] table[x=time,y=best] {data_health_2};
        \addplot[mark=*,blue] table[x=time,y=mean] {data_health_2};
      \end{axis}
    \end{tikzpicture}
  \end{subfigure}
  \begin{subfigure}[h]{0.3\linewidth}
    \center
    \begin{tikzpicture}
      \begin{axis}[title={г) $p_{m_4} > p_{m_3}$}, width=\linewidth, no marks, ymax = 1000, ylabel={Фитнес}, xlabel={Время, мс}]
        \addplot[mark=x,red] table[x=time,y=best] {data_health_3};
        \addplot[mark=*,blue] table[x=time,y=mean] {data_health_3};
      \end{axis}
    \end{tikzpicture}
  \end{subfigure}
  \\
  \ref{legend_local_refcmp_health}
  \caption{Динамика эволюции при разных вероятностях мутации}
  \label{img:cmp_health}
\end{figure}

На рисунке~\ref{img:cmp_health}а видно, что график лучшего фитнеса практически совпадает с графиком среднего фитнеса, особенно в поздних поколениях. Такие популяции называются умеренно инновационными в силу небольшого разнообразия между особями и медленного процесса эволюции. Вероятность успешного обнаружения решения задачи, поставленной в эксперименте, составила~16\% (отношение запусков, в ходе которых решение было обнаружено, к общему количеству запусков алгоритма).

На следующем рисунке заметно, что форма графиков совпадает (не принимая во внимание колебания графика среднего фитнеса), однако они не пересекаются, между ними наблюдается разрыв. Процентное соотношение успешных запусков составило~47\%. Такая модель эволюции является здоровой, но слабой.

С повышением вероятности мутации возрастает успешность алгоритма, достигая максимума при $p_{m_3}=0.1$, показанного на рисунке~\ref{img:cmp_health}в: здоровая и сильная модель. Характерен б\'{о}льший разрыв между графиками, чем в предыдущем случае, однако с сохранением тенденции к росту среднего фитнеса.

Наконец, последняя приведённая динамика со средним фитнесом на постоянном низком уровне с небольшими колебаниями является примером полностью хаотической системы, в которой, несмотря на элитизм, каждое поколение является по сути случайным.

Исследование операторов рекомбинации в качестве единственных источников, вносящих генетическое разнообразие, выявило процесс усреднения популяции~--- быстрое сокращение разрыва между графиками лучшего и среднего фитнеса с последующим их перекрытием. Это означает, что геном всех особей популяции совпадает, и разнообразие устранено, что является следствием преждевременной сходимости такого подхода к локальному оптимуму.

В начальной популяции, заполненной особями, созданными случайным образом, жизнеспособные решения~--- событие редкое, особенно при решении сложных задач. Удачным подходом в таком случае будет начало процесса эволюции с одной или нескольких особей-основателей~\cite{HreFer02}. Эффект основателя, описанный Э.~Майром, как создание новой популяции из особей-основателей, может быть инициирован дрейфом генов~--- изменением частоты вариантов генов вследствие случайных процессов и работы операторов отбора. Наиболее выраженный случай эффекта основателя~--- колонизация необитаемой области (создание новой популяции) одной особью.

Это означает, что при определённом значении вероятности мутации можно добиться максимальной эффективности работы всего алгоритма. Кроме того, оператор мутации уменьшает тенденцию популяции к гомогенизации (утрате разнообразия), и устраняет сильную зависимость эффективности алгоритма от размера популяции.

%--------------------------------------------------------------------

\subsubsection{Влияние на производительность}

Влияние наиболее вероятных значений среднего количества мутаций на хромосому на работу алгоритмов с различными способами кодирования синтаксических деревьев показано в таблице~\ref{tbl:cmp_mutation_probabilites}. Отметим общий для всех способов кодирования максимум эффективности при 1--2~мутациях на хромосому. В случае ПЭГ с наложениями максимум эффективости выделить значительно сложнее, однако всё же прослеживается падение производительности при количестве мутаций более~2. В дальнейшем будет использоваться значение в 2~мутации.

\input{cmp_mutation_probabilites}

%--------------------------------------------------------------------

\subsubsection{Модификации}

Наибольшим недостатком алгоритма ПЭГ является его склонность к преждевременной сходимости к локальному оптимуму, потому любые техники, направленные на решение этой проблемы, приводят к существенному росту производительности ПЭГ, что выражается в сокращении времени сходимости, улучшении средней приспособленности решений, повышению вероятности успешного обнаружения оптимального решения.

Для усиления способности алгоритма к поиску предлагается~\cite{2008acat.confE.66T} следующий алгоритм динамической мутации DM-GEP. При максимальном количестве поколений $T$ процесс эволюции разбивается на три фазы:
\begin{enumerate} \itemsep0pt \parskip0pt \parsep0pt
  \item Начальная фаза: поколения от первого до $T_1$, $0 < T_1 < T$. Вероятность мутации $p_m$ возрастает каждые два поколения со значения 0.022 до 0.44 с шагом 0.001.
  \item Метафаза: поколения от $T_1$ до $T_2$, $T_1 < T_2 < T$. Вероятность мутации $p_m$ возрастает каждые пять поколений со значения 0.022 до 0.66 с шагом 0.002.
  \item Анафаза: поколения от $T_2$ До $T$. Вероятность мутации $p_m$ уменьшается каждые десять поколений со значения 0.066 до 0.022 с шагом 0.001.
\end{enumerate}

В работе~\cite{conf/dews/Tang06} предлагается использовать динамическую установку вероятности мутации, как самого разрушительного оператора, делая ей индивидуальной для каждой особи и зависимой от фитнеса особи. Чем выше фитнес особи, тем более она приспособлена, тем больше внимания требуется уделять локальному поиску, тем меньшая устанавливается вероятность мутации:

\begin{equation}
\label{eq:dynaminc_mutation_probability}
p_m(I) = (1 - f(I) / 1000)*(p_{m_{max}} - p_{m_{min}})+p_{m_{min}}
\end{equation}
где 1000~--- максимальное значение фитнеса, $p_{m_{min}}=0.044$ и $p_{m_{max}}=0.1$~--- нижний и верхний пределы вероятности мутации.

Производительность ПЭГ сильно зависит от заданных вероятностей применения генетических операторов. Снизить влияние этих значений на работу алгоритма возможно путём задействования подхода выбора нового значения, принятого в методе симуляции отжига~\cite{Siwei:2005:pICWCNMC}. Суть применения симуляции отжига состоит во введении зависящего от времени (номера итерации алгоритма) параметра $T$, называемого температурой. Чем выше температура в данный момент времени, тем более вероятна замена исходной особи новой, полученной путём применения генетического оператора. Температура системы уменьшается с каждой последующей итерацией, способствуя поиску окрестностей глобального оптимума на первой фазе алгоритма, и приводя затем к уточнению его местоположения. Скорость понижения температуры управляется аргументом $a$.

На каждой итерации алгоритма ПЭГ к каждой родительской особи $old$ последовательно применяются генетические операторы, каждый из которых порождает новую дочернюю особь $new$. Дочерняя особь занимает место родительской в популяции при выполнении следующего условия:

\begin{equation}
\label{eq:anneal_simulation}
\min(1, e^{-\left\lbrack\frac{f(old) - f(new)}{T_i}\right\rbrack}) > random[0, 1],
\end{equation}
где $f(old)$, $f(new)$~--- фитнесы родительской и дочерней особей соответственно, \mbox{random[0, 1]}~--- случайная величина в диапазоне [0, 1], $T_i$~--- температура на i-той итерации алгоритма. Данная модификация позволила несколько улучшить эффективность ПЭГ.



%--------------------------------------------------------------------



\subsection{Операторы рекомбинации}

Операторы рекомбинации перемещают последовательные участки генома в пределах хромосомы. Три типа таких участков накладывают различные ограничения на операторы.

Оператор копирования со сдвигом копирует последовательность символов генома в любую позицию головы гена, кроме первой. Ограничение на первую позицию обусловлено тем, что перемещаемая последовательность может начинаться с терминального символа, помещение которого в корень приведёт к вырожденному дереву из одного элемента. Ген-источник копируемой последовательности остаётся неизменным. Элементы головы гена-приёмника начиная с позиции, в которую будет скопирована целевая последовательность, сдвигаются для освобождения места, вышедшие за пределы головы элементы отбрасываются.

Оператор копирования со сдвигом в корень отличается от предыдущего тем, что целевая последовательность начинается с элемента-функционала и копируется в корень гена-приёмника. Оба оператора копирования вносят значительные изменения, и потому наравне с мутацией отлично подходят для внесения генетического разнообразия, предотвращая застревание в локальном оптимуме и ускоряя поиск хороших решений.

Оператор перестановки генов вносит изменения в результат вычисления декодированного дерева только при условии использования некоммутативной связующей функции в мультигенной хромосоме, такой как <<ЕСЛИ, ТО>>. Однако более всего преобразующая сила данного оператора проявляет себя в связке операторами кроссовера. Например, возможно появление дублирующихся генов~--- явление, играющее важную роль в биологии и эволюции.

Все три вида рекомбинации производят обмен генетическим материалом между двумя родительскими хромосомами, порождая две новые особи. Вероятность того, что будет применён один из трёх описанных ниже операторов обычно задают~70\%.

В операторе одноточечной рекомбинации две особи скрещиваются вокруг линии, проходящей через случайным образом выбранную позицию хромосомы. Рассмотрим на примере двух родительских хромосом:

\begin{samepage}
\begin{alltt}
012345012345
XXXXXXXXXXXX
OOOOOOOOOOOO
\end{alltt}
\end{samepage}

Взяв для примера позицию~3 в первом гене как точку кроссовера, оператор меняет местами участки хромосом особей от точки кроссовера между собой, что выглядит следующим образом:

\begin{samepage}
\begin{alltt}
012345012345
XXXOOOOOOOOO
OOOXXXXXXXXX
\end{alltt}
\end{samepage}

В большинстве случаев полученное потомство проявляет характеристики своих родителей, что делает одноточечную рекомбинацию наиболее часто используемым в ПЭГ оператором, наравне с мутацией.

Отличие двухточечной рекомбинации состоит в том, что обмен участками генома происходит между двумя точками. Выбрав в качестве примера позицию~2 в первом гене и~4~--- во втором, и применив оператор, дочерние хромосомы выглядят как описано ниже:

\begin{alltt}
012345012345
XXOOOOOOOOXX
OOXXXXXXXXOO
\end{alltt}

Преобразующая сила (способность вносить генетическое разнообразие) двухточечной рекомбинации выше, чем у одноточечной, потому применяется чаще для решения сложных задач с использованием мультигенных хромосом.

В генной рекомбинации производится замена генов, занимающих в обоих хромосомах позицию, определяемую случайным образом.

Следует отметить, что использование рекомбинации и/или операторов перемещения как единственного вида применяемых операторов сводится к перемешиванию существующих участков генома, и не обеспечивает создание нового генетического материала. Потому для решения сложных задач в таком случае может понадобиться популяция большого размера. Созидательная способность ПЭГ заключается в применении всех описанных техник.



%--------------------------------------------------------------------



\subsection{Числовые константы в ПЭГ}

Создание числовых констант с плавающей запятой необходимо для выполнения символической регрессии. В исходном алгоритме ПЭГ для обозначения константы в геноме используется терминальный символ <<?>>, а начальный набор констант представлен доменом $D_c$~\cite{ferreira:2002:WSC}. Каждый ген располагает собственным массивом, содержащим используемые им константы, этот массив заполняется на основе домена $D_c$ при генерации начальной популяции. Численное значение константным символам назначается только при экспрессии генов. Для внесения генетического разнообразия был добавлен специальный оператор мутации констант.

Начальный набор возможных численных констант может быть задан вручную: как перечислением, так и заданием диапазона~--- успешное решение задач таким методом прямого управления константами возможено только при наличии априорных знаний о решении, позволяющих задать корректный диапазон. Потому на практике используется другой метод, который заключается в инициализации значений констант случайными числами.

%--------------------------------------------------------------------

\subsubsection{Динамические константы}

При старте алгоритма ПЭГ на этапе создания новой популяции начальные значения констант в хромосомах предлагается отбирать из множества $A$=\{0.1316, 0.2128, 0.3441, 0.5571, 0.9015, 1.4588, 2.3605, 3.8195, 6.1804, 10.0007\}, созданного таким образом, чтобы соотношение соседних элементов являлось золотым сечением~\cite{Peng:2007:FFC:1304604.1305824}. В данной модификации предлагается следующий оператор мутации констант. Для каждой константы~$C_j$ в геноме выбранной особи:
\begin{enumerate} \itemsep0pt \parskip0pt \parsep0pt
  \item Из фиксированного глобального массива первичных констант~A случайным образом выбирается константа~$V$.
  \item Случайным образом выбирается оператор~$Op$ из множества~$\{+, -, /, *$\}.
  \item $C_j \leftarrow Op(C_j, V)$.
\end{enumerate}

%--------------------------------------------------------------------

\subsubsection{Плавная и случайная мутация констант}

В работе~\cite{li:2004:lbp} описаны исследования операторов мутации, основанных на техниках плавной и случайной мутации. Был задан оператор одноточечной мутации константы: при начальной конфигурации ПЭГ задаётся отсортированный список возможных констант. Данный оператор изменяет одну символ-константу в гене, при случайной мутации на замену выбирается любой другой элемент списка, при плавной~--- какой-либо из соседних текущему. Данная операция применяется к каждому гену хромосомы. Локальный поиск должен производиться в сторону оптимального решения, это означает применение результата мутации только тогда, когда её результат повышает фитнес особи.

Всего было исследовано пять подходов к мутации:
\begin{enumerate} \itemsep0pt \parskip0pt \parsep0pt
  \item Плавная мутация лучшей особи в конце расчёта поколения.
  \item Случайная мутация лучшей особи в конце расчёта поколения.
  \item Плавная мутация каждой особи в первые $\alpha\%$~поколений в конце расчёта поколения.
  \item Случайная мутация каждой особи в первые $\alpha\%$~поколений в конце расчёта поколения.
  \item Интервальная случайная мутация: первые $g$~поколений случайной мутации подвергаются все особи популяции в конце расчёта поколения.
\end{enumerate}

Применение подхода, при котором мутации подвергается только лучшая особь поколения, никак не повлияло на эффективность алгоритма. Из этого можно сделать вывод, что лучшая особь, полученная исходным алгоритмом ПЭГ, является по определению достаточно хорошей и не требует модификаций. Также это подтверждает, что решающую роль в поиске оптимальных решений играет именно эволюционный процесс ПЭГ: выполнение генетических модификаций и отбор.

Применение мутации к каждой особи популяции значительно повышает фитнес популяции, особенно в первых поколениях. Однако с течением времени процентное соотношение особей, улучшенных после мутации, падает. Это происходит вследствие того, что спустя определённое количество поколений формируются группы суб-оптимальных решений, константы которых уже должным образом настроены, потому на этом этапе важнее операторы, приводящие к б\'{о}льшим изменениям.

Задействование методов, применяющих мутации констант к каждой особи, влечёт за собой повышение времени выполнения алоритма, однако получение более качественных решений не позволяет напрямую сравнить эффективность таких модифицированных алгоритмов с исходным ПЭГ. Кроме того, применение указанных трёх техник к различным задачам не позволило выявить наилучшую из них.

%--------------------------------------------------------------------

\subsubsection{Плавно-динамические константы}

Недостатком изученных операторов мутации численных констант является отсутствие тонкой подстройки коэффциентов особей: степень изменения ничем не ограничена, не предоставляя гарантий поступательного движения в сторону оптимума. Для направления процесса эволюции была построена следующая процедура, комбинирующая динамический подход работы с константами ПЭГ и метод иммунных сетей~\cite{SergMir_02_2014_radio}. Создание нового терминала-константы выполняется согласно динамическому подходу: новому узлу (при создании начальной популяции, либо после изменения типа узла на терминал-константу оператором мутации) задаётся значение одно из элементов массива <<золотых сечений>>~$A$. Оператор плавно-динамической мутации констант изменяет значение в пределах~$\pm10\%$ от текущего:

\begin{equation}
\label{eq:zerg_const_mutation}
V = V \times (1 + random(-0.1\ldots0.1))
\end{equation}

В таблице~\ref{tbl:cmp_dynamic_constants} продемонстрировано преимущество в эффективности такого подхода по сравнению с неограниченной мутацией. Данный оператор инициализации и мутации численных констант был использован во всех представленных экспериментах.

\input{cmp_dynamic_constants}



%--------------------------------------------------------------------



\subsection{Функции фитнеса}

Для повышения давления отбора предлагается~\cite{2008acat.confE.66T, SergMir_03_2013_comint} ввести пороговое значение фитнеса. Особи, фитнес которых не достигает этого порогового значения, не допускаются к репродукции. Установка порогового значения требует соблюдения баланса между ускорением сходимости и уменьшением разнообразия, что приведёт к преждевременной сходимости. Целесообразно применять динамический порог, обновляющийся на каждой итерации и составляющий значение среднего фитнеса популяции, умноженного на коэффициент масштабирования. В большинстве случаев в зависимости от задачи используется значение коэффициента в диапазоне [0.15, 1.5], с оптимальным значением~1.25. Давление на процесс эволюции, оказываемое порогом отбора, положительно влияет на поиск оптимального решения, в 4~раза сокращая время поиска.

В схему расчёта фитнеса в работе~\cite{Lopes:2004:AMCS} был добавлен коэффициент давления отбора, отражающий количественную степень влияния возрастания среднеквадратичной ошибки решения на падение его приспособленности:

\begin{equation}
\label{eq:EGIPSYS_fitness}
f(i, g) = 1000 / (1 + k * err(i, g))
\end{equation}
где $i$~--- индекс особи в популяции, $g$~--- поколение, $err(i, g)$~--- среднекваратичная ошибка $i$-го решения, $k$~--- коэфффициент давления отбора, 1000~--- максимальное значение фитнеса. Ниже приведён график зависимости функции фитнеса от погрешности для различных значений~$k$ (рисунок~\ref{img:kz}), далее~--- влияние значения $k$ на эффективность модификаций алгоритмов (таблица~\ref{tbl:cmp_mse_k}).

\begin{figure} [h]
  \center
  \begin{tikzpicture}[domain=0:10]
    \begin{axis}[xlabel={$MSE$}, ylabel={$f(MSE)$}, width=0.4\linewidth, mark repeat=5, legend entries={{$k=0.01$},{$k=0.1$},{$k=1.0$},{$k=10.0$}}, legend to name=legend_local_refkz, legend columns=4]
      \addplot[mark=x,red]{1000 / (1 + 0.01 * x)};
      \addplot[mark=*,blue]{1000 / (1 + 0.1 * x)};
      \addplot[mark=o,black]{1000 / (1 + 1.0 * x)};
      \addplot[mark=square,black]{1000 / (1 + 10.0 * x)};
    \end{axis}
  \end{tikzpicture}
  \\
  \ref{legend_local_refkz}
  \caption{Графики функции фитнеса для различных значений $k$}
  \label{img:kz}
\end{figure}

\input{cmp_mse_k}

В результате экспериментов по символьной регрессии различных наборов данных было выявлено оптимальное значение коэффициента давления отбора~$k=10.0$. Как видно, данная величина отличается от значения, принятого в исходном алгоритме ПЭГ, следовательно, применение данной модификации можно считать успешным.

%--------------------------------------------------------------------

\subsubsection{Частичный подсчёт фитнеса}

Главным недостатком как эволюционных алгоритмов в целом, так и ПЭГ в частности~--- это их низкая скорость работы (по сравнению со скоростью посторения ИНН, регрессионных моделей)~\cite{ferreira:2001:wsc6Aa}. Любое ускорение алгоритма ПЭГ позволяет улучшить качество получаемых моделей по причине возможности произвести расчёт б\'{о}льшего количества поколений и выполнить больше независимых запусков за эквивалентное время.

Наиболее ресурсоёмкой частью алгоритма является вычисление фитнеса: на вход модели подаются все точки входных данных, которые затем обрабатываются в соответствии с программой (формулой, правилом), закодированной моделью. В ходе исследований, направленных на ускорение процедуры расчёта фитнеса, удалось получить универсальный метод~\cite{SergMir_11_2013_info_problem}, позволивший значительно улучшить оба показателя эффективности: как~$e_{b}$, так и~$r_{f}$.

В формулу расчёта фитнеса как правило входит СКО модели, а само значение фитнеса служит для сравнения эффективности моделей между собой. Было обнаружено, что для такой оценки не требуется вычисление СКО по полному набору подаваемых на вход данных. Если на определённом этапе текущее значение СКО (либо соответствующий фитнес) превысит некоторый порог~$threshold$~--- дальнейшие расчёты могут быть прерваны. Такой подход позволяет избежать лишних затрат на точное вычисление фитнеса плохо приспособленных особей, заменяя его приближённым оценочным значением.

Весь набор входных данных~$DataIn$ перемешивается для избежания последовательностей соседних точек, и разбивается на $G$~пакетов~$Dg_{k}$:

\begin{equation}
\label{eq:zerg_partial_mse_groups}
DataIn=\bigcup_{j=1\ldots G}{Dg_{j}}
\end{equation}

После обработки каждого пакета производится оценка фитнеса, и на её основании выносится решение о прерывании вычислений особи. Условием прерывания служит сравнение фитнеса по вычисленным первым~$K$ группам с пороговым значением~$threshold$, которое динамически изменяется в процессе эволюции и потому задаётся некоторой долей (например,~0.7) от среднего фитнеса популяции на предыдущем поколении:

\begin{equation}
\label{eq:zerg_partial_mse_fit_form}
f(i, g + 1) = f_{z}(i, 0.7 \frac{1}{N}\sum\limits_{j=1}^{N}{f(j,g)})
\end{equation}

\begin{equation}
\label{eq:zerg_partial_mse_fit_form_exp}
f_{p}(i, K) = f_{MSE}(i, \bigcup_{j=1\ldots K}{Dg_{j}}), K \le G
\end{equation}

где $f(i, g + 1)$, $f(i, g)$~--- возвращаемое значение фитнеса~$i$-той особи расчитываемого и предыдущего поколений, $f_{z}(i, threshold)$~--- функция взвешенного фитнеса, $N$~--- размер популяции, $K$~--- количество обработанных групп входных данных. Для того, чтобы внести различие между особями, расчёт которых был прерван в различные моменты времени (пороговое значение ошибки было превышено при разном количестве обработанных групп входных данных), был добавлен коэффициент штрафа~$K/G$:

\begin{equation}
\label{eq:zerg_partial_mse_fitness}
f_{z}(i, threshold) =
	\begin{cases}
		f_{p}(i, G) & f_{p}(i, G) \ge threshold \\
		f_{p}(i, K)\frac{K}{G} & f_{p}(i, K) < threshold
	\end{cases}
\end{equation}

Таким образом, плохие решения будут занимать меньшее количество ресурсов, в то время как лучшие особи будут вычислены наиболее точно. В таблице~\ref{tbl:cmp_fitnesses} приведено сравнение производительности исходных алгоритмов с модифицированными частичным подсчётом СКО.

\input{cmp_fitnesses}




%--------------------------------------------------------------------



\subsection{Выводы}

Какие-то выводы

\clearpage
\section{Статья 3: Исследование модификаций эволюционного процесса в программировании с экспрессией генов}

\subsection{Введение}

В ходе анализа существующих методов и средств, направленных на ускорение получения модели алгоритмом ПЭГ и повышение её качества, а также проведённых экспериментов был выявлен ряд ограничений производительности алгоритма, таких как длительность вычисления фитнеса, отсутствие тонкой подстройки числовых констант, вляние размера хромосомы на скорость сходимости, проблемы с поиском сложных моделей. Далее описаны методы и подходы, созданные в ходе преодоления данных ограничений.

Алгоритм ПЭГ заключается в выполнении следующих этапов:
\begin{enumerate} \itemsep0pt \parskip0pt \parsep0pt
  \item Создание начальной популяции
  \item Декодирование особей
  \item Выполнение программ особей
  \item Вычисление фитнеса особей
  \item Проверка критерия останова алгоритма (достигнута требуемая точность, либо истекло лимитированное время выполнения)
  \item Копирование лучшей особи (Элитизм)
  \item Отбор
  \item Репликация
  \item Мутация
  \item Операторы переноса
  \item Операторы рекомбинации
  \item Возврат к пункту 2
\end{enumerate}



%--------------------------------------------------------------------



\subsection{Методика тестирования}

В предыдущей статье цикла [ссылка на статью 1] описана методика оценки эффективности модификаций алгоритма ПЭГ, заключающаяся в статистической обработке результатов множества запусков с целью получения таких основных метрик: СКО наилучшей модели среди всех запусков ($e_{b}$) и доля успешных запусков ($r_{f}$)~--- таких, в ходе которых была получена модель с приемлемой для данной задачи точностью.

Тестирование производилось на следующих задачах:
\begin{enumerate}
  \item $y(x) = \sin x, x \in [-5, 4.6]$;
  \item Функция Розенброка: $z(x, y) = {(1 - x)}^2 + 100 {(y - x^2)}^2, x \in [-3, 3], y \in [-3, 3]$;
  \item Сумма четырёх сигмоид:
    \begin{multline}
      z(x, y) = \frac{1}{2} \times \exp\left(-\frac{(9x-2)^2 - (9y-2)^2}{4}\right) + \frac{3}{4} \times \exp\left(-\frac{(9x+1)^2}{49} + \frac{(9y+1)^2}{10}\right) + \\
      + \frac{1}{2} \times \exp\left(-\cfrac{(9x-7)^2 + (9y-3)^2}{4}\right) -\frac{1}{5} \times \exp\left(-(9x-4)^2 - (9y-7)^2\right), \\
      x \in [-2, 2], y \in [-2, 2]
    \end{multline}
\end{enumerate}



%--------------------------------------------------------------------



\subsection{Обеспечение разнообразния начальной популяции}

Для наиболее эффективного исследования пространства поиска особи начальной популяции должны быть как можно меньше похожи между собой. Создание начальной популяции случайным образом не предполагает никаких процедур по обеспечению генетического разнообразия.

Сравнение особей между собой наиболее удобно производить по генотипу по причине его линейности и лёгкости считывания. Целесообразно при этом сравнивать только кодирующие участки, участвующие в построении синтаксического дерева.

Для количественного измерения близости хромосом между собой в качестве метрики предлагается~\cite{Duan:2007:SID:1304604.1305918} использовать правило, возвращающее максимальное количество $r$ подряд идущих совпадающих элементов. Если две сравниваемые последовательности равны друг другу, $r$ будет равно длине последовательности. Два гена считаются близкими, если значение $r$ превышает определённый порог, авторами использовалось значение 7.
Процедура создания начальной хромосомы с использованием описанной техники выглядит следующим образом:

\begin{enumerate} \itemsep0pt \parskip0pt \parsep0pt
  \item Создание пустой популяции.
  \item Создание новой особи случайным образом.
  \item Сравнение этой особи с каждой особью, добавленной в популяцию.
  \item Если новая особь близка какой-либо особи в популяции, перейти к шагу~2, иначе добавить особь в популяцию.
  \item Если популяция полностью заполнена, завершить процедуру, иначе перейти к шагу~2.
\end{enumerate}

Были проведены эксперименты по моделированию различных наборов данных, предоставляемых Лёвенским университетом. Полученные модели обладают б\'{о}льшей корреляцией с тестовыми данными, чем результаты работы исходного алгоритма ПЭГ.



%--------------------------------------------------------------------



\subsection{Модификации эволюционного процесса}

\subsubsection{Возврат в исходное состояние}

Когда эволюционный процесс проходит определённое количество поколений, средний фитнес популяции достаточно высок, однако разнообразие практически устранено, что приводит к преждевременному схождению к локальному оптимуму, уменьшая шансы успешной глобальной оптимизации. Для решения этой проблемы было предложено~\cite{zhong2006improve} реализовать явление атавизма~--- появление свойств далёких предков. 

Современная генетика описывает следующие причины возникновения атавизмов: рекомбинация утраченного гена предка в результате скрещивания или мутации, и устранение стопового элемента генома, заблокировавшего на определённом этапе экспрессию гена предка. Из природы рассмотренных причин следует обратимость процесса схождения популяции. Кратковременно развернуть процесса эволюции в обратном направлении можно при помощи реализации возврата популяции ПЭГ в исходное состояние (Backtraced GEP).

В основе алгоритма возврата лежит структура данных стек, хранящая <<контрольные точки>>~--- состояния популяции. Если фитнес лучшей особи в новой популяции (полученной в результате репликации и применения генетических операторов), выше, чем у лучшей особи в популяции на верхушке стека (в последней контрольной точке), это означает верное направление эволюции, потому новая популяция заталкивается в стек, формируя новую контрольную точку. В противном случае можно сделать вывод о тупиковой ветви эволюции, потому последняя контролная точка выталкивается из верхушки стека.

Применение методики возврата к предыдущему состоянию привела к значительному улучшению качества получаемых решений.

%--------------------------------------------------------------------

\subsubsection{Группировка особей и их параллельная эволюция}

Начиная с определённого поколения, находясь в поздней фазе эволюции, популяция обладает следующими признаками: сложная структура синтаксических деревьев, замедление скорости эволюции, незначительное разнообразие популяции. Однако было замечено~\cite{Jiang:2008:MPT:1473248.1474007}, что разбиение популяции на группы и их параллельная обработка позволяет избежать преждевременной сходимости.

Разбиение популяции на группы происходит следующим образом. Особи популяции сортируются в порядке возрастания значений фитнеса:

\begin{equation}
\label{eq:niches_G}
G = \{I_i | f(I_{i+1}) \ge f(I_i), 1 \le i \le N\}.
\end{equation}

Группой считается последовательность, в которой разница значений фитнесов соседних особей не превышает заданую величину $d$:

\begin{equation}
\label{eq:niches_G_i}
G_i = \{I_j | f(I_{j+1}) - f(I_j) \le d, 1 \le j \le N, 1 \le i \le Q\}
\end{equation}
где $Q$~--- количество образованных групп. Под плотностью группы понимается отношение её размера (количества особей) к размеру популяции:

\begin{equation}
\label{eq:niches_p_i}
p_i = \frac{|G_i|}{N}
\end{equation}

Тогда сумма плотностей групп одной популяции всегда будет равна единице:

\begin{equation}
\label{eq:niches_p_i_sum_1}
\sum\limits_{i=1}^Q{p_i} = 1
\end{equation}

Энтропия популяции, математическое ожидание и дисперсия фитнеса вычисляются таким образом:

\begin{equation}
\label{eq:niches_E}
E(G) = - \sum\limits_{i=1}^Q{p_i \times \log{p_i}}
\end{equation}

\begin{equation}
\label{eq:niches_M}
M(G) = \sum\limits_{i=1}^Q{\hat{f_i} \times p_i}
\end{equation}

\begin{equation}
\label{eq:niches_D}
D(G) = \frac{1}{N} \times \sum\limits_{i=1}^{N}{{f(I_i) - \hat{f}}^2}
\end{equation}

\begin{equation}
\label{eq:niches_f_i_hat}
\hat{f_i} = \frac{1}{|G_i|} \times \sum\limits_{j=1}^{|G_i|}{f(I_j)}
\end{equation}

\begin{equation}
\label{eq:niches_f_hat}
\hat{f} = \frac{1}{N} \times \sum\limits_{i=1}^{N}{f(I_i)}
\end{equation}

Если энтропия и дисперсия популяции меньше определённых заданых пороговых значений, можно сделать вывод о недостаточной степени генетического разнообразия. В этом случае имеет смысл заменить худшие 10\% особей новыми, созданными случайным образом. Указанные операции требуется применять к каждому поколению. Итоговый алгоритм работы алгоритма представлен в листинге~\ref{algo:niches}.

\begin{algorithm}
\SetAlgoLined
\While{Не достигнуто максимальное поколение}
{
  Создание начальной популяции\;
  Разбиение популяции на группы\;
  \ForEach{группы популяции}
  {
    Перенос лучшей особи\;
    Применение генетических операторов\;
  }
  Расчет энтропии и дисперсии популяции\;
  Замена худших особей популяции при необходимости\;
  Разбиение популяции на группы\;
}
Вывести лучшую особь\;
\caption{Алгоритм ПЭГ с группировкой особей}
\label{algo:niches}
\end{algorithm}

Авторами не было произведено сравнение производительности полученной системы с исходным ПЭГ или другими его модификациями.

Вляние отдельно взятой модификации, удаляющие худшие особи, показано в таблице~\ref{tbl:cmp_replace_worst}. Из этих результатов следует изменение баланса двух основных вероятностных показателей алгоритмов: улучшается точность наилучшей возможной модели при большом количестве запусков, но в то же время страдает вероятность обнаружения приемлемого решения.

\input{cmp_replace_worst}

%--------------------------------------------------------------------

\subsubsection{Популяции неоднородных особей}

Одна из проблем алгоритма ПЭГ~--- определение оптимального размера головы гена (и размера решения). Из-за отсутствия процедуры априорного задания приходится запускать алгоритм множество раз с разными параметрами для поиска наиболее подходящих. В качестве другого способа предлагается~\cite{Lopes:2004:AMCS} использовать в одной популяции хромосомы различной длины: половина популяции заполняется особями пропорционально с диапазоном размеров, заданным пользователем, длина генов особей второй половины устанавливается случайным образом. Операторы рекомбинации в таком случае применяются только к особям с хромосомами равной длины.

Для подстройки констант предлагается использовать градиентный алгоритм, обладающий высокой вычислительной стоимостью, потому применяющийся с определённой вероятностью. Константа либо заменяется случайно выбранной, либо изменяется в пределах 10\%. Если мутированная особь лучше исходной, то заменяет её, иначе константа заменяется полусуммой последних двух значений. Процесс повторяется до тех пор, пока мутировавшая особь не будет хуже исходной, либо по достижении предела в 10 итераций.

Аналогичные исследования проведены в работе~\cite{journals/acisc/BrowneS10}. Предложено использовать хромосомы с переменным числом генов и переменным размером каждого гена. Введены новые операторы, направленные на изменение длины генома:
\begin{itemize} \itemsep0pt \parskip0pt \parsep0pt
  \item Удаление одного ген из хромосомы.
  \item Создание и добавление одного нового гена в хромосому.
  \item Перенос участка головы одного гена в голову другого, что приводит к укорачиванию первого и удлиннению второго.
  \item Рекомбинация разнородных хромосом.
\end{itemize}

Такой подход привёл к двукратному сокращению длины гена и, соответственно, размера синтаксического дерева решения.

%--------------------------------------------------------------------

\subsubsection{Дополнительная популяция}

В качестве одной из мер повышения вероятности обнаружения решения в авторских работах~\cite{SergMir_03_2013_vkntu, SergMir_04_2013_smolensk} было предложено использование дополнительной параллельной независимой популяции. Если при очередной итерации (поколении) работы алгоритма фитнес лучшей особи дополнительной популяции превысит фитнес лучшей особи основной популяции~--- данная особь копируется на место худшей особи основной популяции. Из результатов, представленных в таблице~\ref{tbl:cmp_additional_population} можно сделать вывод об успешном решении поставленной задачи при использовании данного подхода. Негативным следствием является снижение точности наилучшей из получаемых моделей. Это происходит по причине возрастания вычислений в расчёте на итерацию алгоритма, следовательно при равном отводимом времени работы модифицированный алгоритм успеет расчитать меньшее количество поколений.

\input{cmp_additional_population}

%--------------------------------------------------------------------

\subsubsection{Инкрементальная эволюция}

В приведённой таблице~\ref{tbl:cmp_tree_depths} отчётливо заметно наличие некоторого оптимального размера генома, при котором достигается максимум производительности алгоритма. Данный размер требует отдельного определения для каждой задачи и каждого способа кодирования, для его выбора могут применяться как простой перебор параметров, так и описанные выше популяции неоднородных особей.

\input{cmp_tree_depths}

Более эффективным показал~\cite{SergMir_04_2013_varna, SergMir_04_2013_sovr} себя метод инкрементальной эволюции, в ходе которого производится ряд последовательных запусков алгоритма с наращиванием длины хромосомы при каждом запуске и копированием лучшей особи предыдущего запуска в начальную популяцию текущего. Такой подход требует меньших вычислительных ресурсов по сравнению с независимыми запусками, т.к. позволяет производить постепенное усложнение дерева. Результаты работы данной модификации приведены в таблице~\ref{tbl:cmp_incremental}.

\input{cmp_incremental}



%--------------------------------------------------------------------



\subsection{Комбинирование множества запусков алгоритма}

\subsubsection{Взвешенная сумма моделей}

Полученные в ходе нескольких запусков алгоритма ПЭГ модели можно~\cite{journals/jikm/AbrahamG06, guo2012novel} обобщить в одну, представляя итоговую формулу в виде:
\begin{equation}
\label{eq:ensembles}
M = a \times M_1 + b \times M_2 + c \times M_3 + \ldots
\end{equation}
где $a$, $b$, $c$, $\ldots$~--- комбинирующие коэффициенты, $M_1$, $M_2$, $M_3$, $\ldots$~--- модели, полученные в ходе запусков ПЭГ.

Для подобра комбинирующих коэффициентов с целью минимизации погрешности итоговой модели, повышению её корреляции с выборками данных был использован генетический алгоритм NSGA II (non-dominated sorting genetic algorithm II).

Как правило, обобщающая модель обладает лучшими характеристиками, чем её составляющие по отдельности.

%--------------------------------------------------------------------

\subsubsection{Итеративный разностный подход}

Одна из наиболее распространённых задач, для решения которой применяется ПЭГ~--- поиск математической формулы, описывающей набор численных данных, этот процесс называется символьной регрессией, либо, в более простом варианте, аппроксимацией функций. Примером такой задачи может служить обнаружение формулы, описывающей сложную поверхность. В ряде случаев сложность моделируемого объекта такова, что обеспечить приемлемую точность при описании компактной формулой невозможно, и требуется увеличить размер искомого дерева, что приводит к резкому росту пространства поиска, а следовательно и вычислительного времени.

Популярным средством повышения сложности формулы с минимальным влиянием на производительность алгоритма является развитие идеи сложных мультигенных хромосом с иерархической структуров, которые были описаны выше.

Ещё более действенным показал себя разработанный в ходе данных исследований разностный подход~\cite{SergMir_03_2014_info_problem, SergMir_04_2014_smolensk}, основанный на простоте комбинирования синтаксических деревьев: математические формулы могут быть легко объединены, например, при помощи функции арифметического сложения, правила классификаторов~--- булевыми <<И>> и <<ИЛИ>>, и т.д. Эта особенность позволяет составить сложную формулу из ряда простых, компактных и быстро вычисляемых по отдельности.

Суть подхода заключается в последовательном применении алгоритма ПЭГ с неизменным набором параметров к поверхностям ошибки~--- наборам численных данных, полученных путём вычитания очередной полученной модели из моделируемых данных. Тем самым разностных подход принципиально отличается от идеи эволюции мультигенных хромосом, где алгоритм пытается обнаружить решение с первой же итерации. При первом запуске на вход алгоритма ПЭГ подаётся набор данных~$T(O_{0})$, с ожиданием на выходе модели~$M$, возвращающей набор данных~$O$:

\begin{equation}
\label{eq:zerg_diff_m_1}
\{M_{1}, O_{1}\} = GEP(T = O_{0})
\end{equation}

На каждом следующем этапе на вход подаётся разность моделей:

\begin{equation}
\label{eq:zerg_diff_m_i}
\{M_{i+1}, O_{i+1}\} = GEP(O_{i} - O_{i - 1})
\end{equation}

Итоговой моделью после N запусков является:

\begin{equation}
\label{eq:zerg_diff_model}
M = M_{1} + M_{2} + \ldots + M_{N}
\end{equation}

В таблице~\ref{tbl:cmp_differential} показано сильнейшее положительное влияние разностного подхода на показатели алгоритма ПЭГ. Следует, однако, учитывать при этом возрастающее пропорционально количеству разностей время, затрачиваемое алгоритмом на поиск каждого слагаемого формулы модели.

\input{cmp_differential}



%--------------------------------------------------------------------



\subsection{Параллелизация}

Выполнение алгоритма требует значительных вычислительных ресурсов. Самым ресурсоёмким является этап расчета приспособленности программы~--– эту операцию требуется выполнять для каждой особи популяции по всему обучающему набору входных и выходных данных на каждой итерации алгоритма. Как правило, фитнес-функция в ПЭГ при решении задачи регрессии основывается на среднеквадратичном отклонении.

Для ускорения процесса целесообразно задействовать такой ресурс компьютера, как наличие нескольких процессоров. Одним из способов автоматизации распараллеливания программ является использование библиотеки, реализующей стандарт OpenMP.

При расчете фитнеса особи не возникает зависимости по данным от остальных особей популяции. Доступ к входным и выходным значениям восстанавливаемой функции предоставляется только на чтение, что позволяет обезопасить разделяемые данные. Выполнение этих условий необходимо и достаточно для эффективного распараллеливания расчета фитнеса популяции~--– к циклу программы можно добавить соответствующую директиву компилятора.

Кроме того, распараллеливанию поддается эволюционный этап алгоритма: текущая популяция объявляется разделяемыми данными с доступом на чтение, при этом в промежуточной популяции (заполняемой в ходе последовательного выполнения генетических операторов) нет зависимости по данным между особями.

В описываемом алгоритме значительная доля общего объема вычислений может быть получена параллельными расчетами, что позволило добиться ускорения выполнения в симметричных многопроцессорных системах~\cite{SergMir_05_2013_sevas}.



%--------------------------------------------------------------------


\subsection{Выводы}

Какие-то выводы

\clearpage
\section{Применение программирования с экспрессией генов}

Построение модели объекта на основе некоторых данных о нём~--- одна из основных задач, для решения которых применяется ПЭГ. На вход алгоритму ПЭГ подаются данные в формате, аналогичном входным данным при обучении ИНС: набор кортежей, содержащих параметры объекта, численные оценки воздействий на него, и количественные характеристики состояния объекта. Цель алгоритма ПЭГ~--- обнаружить явные и скрытые взаимосвязи между входными значениями~--- параметрами объекта и воздействиями на него,~--- и выходными~--- реакцией объекта и его состоянием. По окончании работы алгоритм ПЭГ выдаёт модель объекта, в зависимости от желаемого набора функций представленную в виде математической формулы (символьная регрессия), компьютерной программы, дерева принятия решений, правила классификации, числового вектора (вырождение ПЭГ в ГА в случае пустого функционального множества), и т.п.

Полученные модели применяются в различных целях. Выявленные зависимости могут помочь при дальнейшем анализе объекта, например, для автоматического открытия законов. В большинстве случаев модель отлично подходит для аппроксимации объекта, в ряде случаев~--- и для прогнозирования.

При построении модели, как и при обучении ИНС, набор известных об объекте данных разбивается на выборки: обучающую и тестовую. Возможен вариант деления на три выборки, в таком случае к указанным двум добавляется валидационная. Обучающая выборка используется для оценки фитнеса особей популяции и дальнейшего отбора. Тестовая выборка используется для определения лучшей особи популяции в последнем либо каждом поколении. Валидационная выборка используется для финальной оценки пригодности модели, полученной алгоритмом ПЭГ.

\subsection{Аппроксимация и символьная регрессия}

Среди применений ПЭГ известно использование алгоритма для моделирования толщины слоёв дорожного покрытия~\cite{Terzi:2005:JAS, saltan:2005:IJEMS}. Данная задача поставлена необходимостью оценки состояния, структурной целостности и ресурса покрытия для определения как целесообразности проведения восстановительных работ, так и вида этих работ.

Слои дорожного покрытия характеризуются модулями упругости, которые на практике принято оценивать исходя из степени деформации, проводя неразрушающее тестирование дефлектометром. Методика получения данных: удар при падении груза известной массы с различной высоты создаёт волну, длина которой измеряется сенсорами. На вход модели подаются данные сенсоров и высота, с которой сбрасывается груз; на выход модели поступает толщина верхнего слоя дорожного покрытия.

В работе~\cite{saltan:2005:IJEMS} было проведено сравнение моделей, полученных при решении описанной задачи двумя недетерминированными эвристическими методами: обучение однослойной ИНС и получение математической формулы с помощью ПЭГ. Оба метода показали способность решать задачу, используя набор данных без априорных знаний. Однако обученная ИНС показала более точные результаты (коэффициент корреляции $R^2=0.9995$), чем формула, выданная ПЭГ ($R^2=0.7963$). В то же время полученная простая математическая формула более удобна в обращении, чем ИНС, количество структурных элементов которой равно произведению количества входов, нейронов входного слоя, выходов, функций суммирования и функций активации.

ПЭГ также нашло применение в задаче оценки испарений, основываясь на данных метеорологической станции возле озера Эгирдир, таких как температура воздуха, влажность и количество солнечного излучения. Точность полученной модели (простой формулы) в~\cite{OzlemTerzi:2005:JAS} оказалась выше, чем у общепринятого в отрасли метода Пенмана-Монтейта.

В работе~\cite{Baykasoglu:2005:ICRM} успешно проведено моделирование производственной линии с использованием GEP и многоцелевого поиска табу.

При разработке цифровых сигнальных процессоров широко используются фильтры с конечной импульсной характеристикой (FIR-фильтры~--- finite impulse response). Преимущества таких фильтров~--- в их устойчивости, нерекурсивности (отсутствии обратных связей), возможности реализации с линейной фазой. Один из параметров, указываемых перед разработкой нового фильтра~--- его порядок. Поиск формул для оценки минимального порядка фильтров с линейной фазой с помощью ПЭГ описан в работе~\cite{GonzalezMunoz:2005:RVK}. Анализ полученных моделей показал, что их точность не уступает другим методам оценки порядка, приведённым в статье.

Аналитическое моделирование компонентов турбомашин проводится при разработке новых устройств и изучении внутренних состояний и потоков на протяжении более чем столетия. Для построения моделей на основе больших баз данных в отрасли помимо линейной регрессии применяются как ИНС, так и алгоритмы генетического программирования, в частности ПЭГ~\cite{MECE2005-79414R3, IMECE2005-79416-R3}. Однако системы, полученные с помощью ИНС, хоть и могут обладать большой предсказательной силой, являются лишь упрощением, в то время как выходная формула ПЭГ может представлять собой фундаментальную модель (при условии подачи независимых аргументов). Стоит отметить, что зачастую полученные с помощью ПЭГ модели имеют слишком большую сложность и требуют дальнейшего упрощения с участием человека.

В работе~\cite{al2010new} предложено использовать ПЭГ для моделирования механических конструкций, применяемых в строительстве, в критических условиях. Механические свойства стали стремительно ухудшаются с ростом температуры (например, под воздействием огня): снижение жёсткости и допустимой нагрузки могут привести к крушению постройки. Особенно описанному явлению подвержены стыки и соединения. Для изучения влияния разрушающих факторов на полужёсткие соединения стальных структур активно применяется компьютерное моделирование с применением таких техник, как ИНС, ГА.

Для накопления обучающих данных, подаваемых на вход алгоритму ПЭГ, и данных для тестирования полученных моделей были проведены эксперименты по измерению характеристик четырёх различных стандартных соединений, применяемых в строительстве, под воздействием пламени с регулируемой температурой. На вход ПЭГ подаются обучающая выборка из 331 элемента, состоящая из 16 параметров: температура, механический момент, механическая прочность, количество болтов, геометрические свойства компонентов соединения.
Полученная модель представляет собой формулу, значение которой отражает поворот конструкции с высокой точностью~--- коэффициент корреляции составил 0.89~--- и которая поэтому может быть использована при проектировании строительных конструкций.


\subsection{Глубинный анализ данных, построение правил классификации}

Правила классификации, как правило, представляются в виде синтаксических деревьев, содержащих логические операторы (<<И>>, <<ИЛИ>>, <<НЕ>>, сравнение и пр.), либо операторы нечёткой логики. Деревья легко поддаются линеаризации (записи в геноме), потом чего к ним становится применим механизм ПЭГ. Полученные правила с лёгкостью записывается в виде небольшой функции на любом языке программирования, например, C.

При бинарной классификации в качестве функции фитнеса зачастую используется следующая формула:
$$
f = \frac{t_p}{t_p+f_n} \times \frac{t_n}{t_n+f_p},
$$
где $t_p$~--- количество верно распознанных положительных элементов данных, $t_n$~--- верно распознанных отрицательных элементов, $f_p$~--- неверно распознанных положительных элементов, $f_n$~--- неверно распознанных отрицательных элементов.

Проведённое в работах~\cite{P1120535113, P1120535114} сравнение производительности ПЭГ с системой классификации {C4.5} на тестовых наборах данных (ирисы Фишера и пр.) показало некоторое преимущество ПЭГ в точности: 65\% верно классифицированных данных {C4.5} и 71.51\% ПЭГ.

Сравнение различных систем классификации также было проведено в~\cite{conf/adma/WeinertL06}: ПЭГ (точность классификации составила 88.425\%), {C4.5} (точность~--- 82.125\%), CSGP (Constrained-Syntax GP~--- ГП с ограниченным синтаксисом, точность~--- 79.375\%) и BGP (Booleanized GP~--- ГП с набором функций, ограниченным до булевых <<И>>, <<ИЛИ>> и <<НЕ>>, точность~--- 85.55\%). Данные для эксперимента были взяты в репозитории машинного обучения Калифорнийского университета в Ирвайне: медицинская статистика состояний пациентов с диагнозом рака молочной железы, болями в груди, дерматологическими проблемами.

Из результатов эксперимента можно сделать вывод о практически равной точности указанных методов, однако система на основе ПЭГ требует меньшего количества вычислений (1500 поколений ПЭГ против 25000 поколений CSGP). Кроме того, ПЭГ позволяет устанавливать желаемую сложность правил, управляя количеством генов.

Аналогичный эксперимент по построению системы диагностирования рака молочной железы описан в~\cite{ferreira:2004:rdbic}. Данные архива PROBEN1 университета Карнеги-Меллон содержат 9 атрибутов. Алгоритмом было обнаружено правило бинарной классификации, корректно описывающиее все 100\% элементов обучающей выборки и 97.14\% тестовой.

При построении бинарного классификатора пациентов из базы данных Национального центра биотехнологической информации США~\cite{ncbinlm2006} на предмет предрасположенности к раку груди на основе 12625 атрибутов-генов нескольких тысяч человек авторы~\cite{conf/rskt/ValdesB06} столкнулись с необходимостью предварительной обработки данных с целью выделения необходимого подмножества атрибутов в многомерном наборе данных. После сокращения количества атрибутов путём кластеризации до десяти система на основе ПЭГ успешно справилась с построением классификатора, при этом было задействовано девять атрибутов. Анализ полученной формулы позволил легко определить наиболее существенные атрибуты: при переводе в полиномиальную форму два атрибута (из девяти) имели высшие степени. При дальнейшем анализе данных удалось выявить единственный атрибут, достаточный для проведения по нему классификации.

Задача другого типа рассматривается в~\cite{journals/iajit/EldrandalyN08}: разработка системы прогнозирования скачков давления в гидросистемах на основе ПЭГ. Такие системы применяются в целях обеспечения равномерного распределения энергии потоков воды в системе, поддержки уровня воды в оросительных каналах и т.п. На вход алгоритму ПЭГ подаётся большой набор обучающих данных, включающих в себя такие параметры, как плотность и динамическая вязкость жидкости, скорость потока, длина отрезка трубы и пр. В целях сравнения была построена модель процессов с использованием множественной регрессии, основанная на двух предположениях: линейное соотношение входных и выходных данных, и независимость входных переменных от выходных. Сравнение моделей проводилось на основании их коэффициентов корреляции $R^2$ и среднеквадратичного отклонения. Модель ПЭГ показала лучшие результаты ($R^2=0.914$ против $R^2=0.873$), однако её формула в то же время обладает б\'{о}льшей сложностью.

Исследованию террористических организаций~--- сетей, состоящих из небольших групп людей, связанных одной целью,~--- и тому, как отличить террористов от главных подозреваемых, посвящена работа~\cite{conf/wisi/QiaoTPFX06}. Обучение и тестирование классификаторов проводились на наборах данных, содержащих несколько сотен записей с такими атрибутами: религия, происхождение, пол, образование, возраст, судимости. Эффективность системы, построенной с помощью ПЭГ, сравнивалась с {C4.5}: при значительно меньшем времени выполнения получены формулы в несколько раз компактнее.

ПЭГ был применён также для построения классификатора, отделяющего сигнал от шума среди событий, зарегистрированных детекторами ускорителя элементарных частиц в Национальной ускорительной лаборатории SLAC в Стенфорде. В ходе работы~\cite{Teodorescu:2006:IEEETNS} было выяснено, что размер функционального множества не оказывает влияния на точность модели. В то же время меньшее количество функций как правило ведёт к б\'{о}льшему размеру синтаксического дерева (количеству узлов в нём), однако данный эффект не проявляется, если ввести в функцию фитнеса штраф за размер дерева.

Эксперименты с увеличением количества входных переменных показали, что алгоритм ПЭГ нечувствителен к их числу (при наличии должного количества релевантных) и способен игнорировать переменные, не представляющие ценности в задаче классификации.

Кроме ПЭГ, к данной задаче были также применены обучение ИНС и построение деревьев принятия решений методом градиентного добавления (BDT~--- Boosted Decision Trees). Разница в эффективности полученных моделей составила 1--3\%. В отличие от ИНС, ПЭГ не подвержен проблеме перетренированности. Скорость работы всех трёх предложенных алгоритмов можно считать равной равной, при б\'{о}льшей обобщающей способности (разнице между результатами на тренировочных и тестовых выборках) метода ПЭГ.

Помимо прочего, существующие системы поиска правил классификации, построенные на основе ГП, могут быть адаптированы под использование ПЭГ. В работе~\cite{conf/iwcls/Wilson07} рассмотрено изменение механизма эволюционирования системы XCSF с ГП на ПЭГ. Модифицированная система значительно превзошла в производительности исходную.

Наиболее часто используемый метод определения принадлежности единицы данных $X$ к определённому классу заключается в проверке условия $gep(X) - t > 0$, где $gep$~--- функция, обнаруженная алгоритмом ПЭГ, $t$~--- граница принадлежности к классу. Значение $t$ зачастую выбирается субъективно, эмпирически, что требует множества запусков алгоритма для определения оптимальной величины.

Для динамического определения границы классов t предлагается применить следующий метод разделения. Пусть $G = \{g_i, i=1, 2, ..., n\}$ будет множество значений, возвращённых gep - функцией, обнаруженной алгоритмом ПЭГ для определения принадлежности данных к классу $c$ - для всех $n$ элементов данных обучающей выборки.

Тогда среднее значение этих значений задаётся как $Mean_c = \frac{g_1 + g_2 + ... + g_n}{n}$.

Пусть $G' = \{g'_i, i=1, 2, ..., n\}$ будет отсортированный по возрастанию массив $G$, тогда значение медианы можно извлечь из срединного элемента: $Median_c = g'_{n/2 + 1}$.

Значение любой из указанных метрик может быть использовано в качестве значения $t$. В случае нескольких классов берётся среднее арифметическое значений метрик для каждого из них. Тестирование данного подхода на наборе данных ирисов Фишера показало его эффективность применительно к формированию правил классификации~\cite{conf/adma/DuanTZWZ06}.

Многоцелевая классификация представляет собой задачу поиска правил классификации, которые удовлетворяли бы сразу нескольким критериям одновременно, таким как точность правил, их понятность (выраженная в количестве вовлечённых в правило атрибутов). Из множественности критериев следует их конкуренция между собой и взаимная противоречивость. Свойства эволюционных алгоритмов привлекают исследователей возможностью получать группы правил классификации.

Системы классификации, основанные на ГА разделяются на две категории, в основе которых лежат два принципиально различных подхода: мичиганский и питтсбургский. Основое различие между ними заключается в схеме кодирования хромосомы.

В мичиганском подходе хромосома фиксированной длины кодирует одно правило классификации. Для поиска всех правил группы требуется либо отдельный запуск алгоритма для вывода каждого из них, что требует б\'{о}льших вычислительных затрат, либо расширение алгоритма путём кодирования особью всей группы правил.

В питтсбургском подходе каждая особь представляется строкой переменной длины и кодирует группу правил целиком. Питтсбургский подход показывает лучшие результаты в пакетом режиме, когда все образцы обучающей выборки доступны на протяжении всего процесса обучения, в то время как мичиганский более подходит для обучения онлайн, в котором домены переменных динамически изменяются.

Система классификации, разработанная в рамках работы~\cite{Dehuri:2008:MCR:1471604.1472090}, кодирует правила в хромосомах фиксированной длины, используя свойство ПЭГ хранить синтаксические деревья переменного размера. Каждая особь хранит одно правило. В функциональном наборе задействованы арифметические действия, правила записываются в двух формах: <<ЕСЛИ <выражение> > <число> ТО ПРИНАДЛЕЖИТ КЛАССУ>> и <<ЕСЛИ <число> < <выражение> < <число> ТО ПРИНАДЛЕЖИТ КЛАССУ>>. Результаты тестирования системы на распространённых наборах данных, таких как ирисы Фишера, показали высокую способность системы к поиску простых, удобных для дальнейшего анализа оператором правил классификации, выявляющих скрытые взаимосвязи в поданных данных.


\subsection{Сжатие изображений}

Первые попытки применения ПГ к задаче сжатия изображений путём их представления в виде математических функций отражены в работе~\cite{fukunaga1998evolving}. Спустя почти десять лет с развитием доступных вычислительных мощностей оказались возможными первые практические результаты, описанные в отчётах об исследованиях~\cite{techrep/Ashworth06, techrep/Gempeler06}.

Первый этап предложенного алгоритма~--- препроцессинг: представление изображения в форме, подходящей для дальнейшей подачи на вход эволюционному процессу. Для этого выполняются первые шаги спецификации сжатия изображений JPEG: разбиение исходного изображения на блоки 8x8 пикселей, дискретное косинусное преобразование каждого из них, умножение на матрицу квантования, округление до целых чисел, линеаризация путём обхода элементов полученной матрицы по зигзагу, отбрасывание элементов, следущих за последним ненулевым \cite{jpeg1993}. Кроме того, от каждого элемента полученного массива отнимается среднее значение массива, таким образом не требуется поиск константы вертикального смещения.

Второй этап алгоритма заключается в эволюционном поиске формул, описывающих каждый блок. На последнем этапе осуществляется кодирование полученных формул, записывая элементы в виде соответствующих им двоичных последовательностей.

При помощи представленного алгоритма удалось сжать тестовое изображение 256x256 пикселей в оттенках серого со среднеквадратичным отклонением 16.3/255 с коэффициентом сжатия 1.5 за несколько часов. Таким образом, средств исходного алгоритма ПЭГ, простых техник препроцессинга изображений и выбранного способа кодирования недостаточно для эффективного решения задачи сжатия.


\subsection{Использование ПЭГ при построении систем}

Построение систем определения допусков отклонений радиокомпонентов~--- одна из важнейших задач при построении аналоговых цепей, наряду c автоматическим выравниванием компонентов на печатной плате. Суть данной NP-сложой задачи состоит в поиске оптимальных допустимых отклонений таких параметров элементов электрической цепи, как сопротивление, ёмкость и индуктивность. Из-за отклонений этих параметров от их номинальных значений невозможно достичь удовлетворения всех устройств всем спецификациям. При решении задачи с помощью ПЭГ на вход алгоритму подаются векторы, содержащие допуски параметров и соответствующую производительность цепи. Формулы, полученные в ходе проведённых в рамках~\cite{DT_ICSES} экспериментов, были верифицированы при помощи метода Монте-Карло, достигнута приемлемая точность в 10\% при значительном превосходстве в скорости про сравнению с требующим больших вычислительных затрат методом Монте-Карло.

Одно из исследований ПЭГ было проведено на задаче определения большинства голосов при анонимном голосовании на выборах из двух кандидатов~\cite{conf/eurogp/Ferreira02}. В качестве подхода к решению был построен механизм клеточных автоматов.

Простейший клеточный автомат представляет собой кольцевой буфер из N клеток, каждая из которых соединена с соседями с обеих сторон. Двоичное состояние каждой клетки обновляется в соответствии с определённым правилом. Правило применяется одновременно ко всем клеткам, процесс повторяется на протяжении $t$ шагов.

Начальная конфигурация (состояние клеток) отображает голоса избирателей: 0 означает голос за одного кандидата, 1~--- за второго. Требуется обнаружить с помощью ПЭГ правило, переводящее состояние всех клеток в 0, если большинство проголосовало за первого кандидата (б\'{о}льшая плотность нулей в начальной конфигурации), либо в 1 в противном случае (б\'{о}льшая плотность единиц). Использование такого правила позволяет сохранить анонимность голоса и избежать прямого опроса избирателей.

В наиболее часто изучаемой версии задачи классификации клеточных автоматов количество клеток N=149. Центральная клетка обозначается символом u, три клетки слева от неё - <<c>>, <<b>>, <<a>>, три клетки справа - <<1>>, <<2>>, <<3>>.

Плотность начальной конфигурации~--- это функция от N аргументов, потому действия каждой клетки с ограниченными информацией и коммуникационными возможностями должны быть согласованы со всеми остальными для корректной классификации начальной конфигурации. Более того, ручная обрабока пространства поиска, включающего $2^{2^7}=2^{128}$ правил перехода, является практически невозможной задачей, потому для поиска правил, лучших чем уже известное правило Гача-Курдюмова-Левина, был использован эволюционный подход. При помощи коэволюционного обучения Поллаком и Жиллем были обнаружены два новых правила (Коэволюция\_1 и Коэволюция\_2), значительно превосходящие по характеристикам все известные до этого: их эффективность составляет 85.1\% и 86.0\% соответственно. Однако эти правила представлены в виде битовых таблиц, а булевы выражения этих правил неизвестны, потому извлечение знаний из них не представляется возможным.
Пространство вариантов правил булевых функций семи аргументов огромно~--- $2^{2^7}=2^{128}$ правил, но ещё больше пространство компьютерных программ, состоящих как из функциональных, так и терминальных элементов. Потому поиск булевых функций, представляющих правила Коэволюция\_1 и Коэволюция\_2, является нетривиальной задачей.

Применение ПЭГ к этой задаче потребовало несколько предварительных оптимизационных запусков. Лучшее решение запуска использовалось в последующем запуске. Такая стратегия применяется при решении сложных задач, т.к. при поиске с нуля непросто найти даже промежуточное хорошее решение. Функциональное множество было составлено из булевых НЕ, И, ИЛИ, ИСКЛЮЧАЮЩЕЕ ИЛИ, И-НЕ, ИЛИ-НЕ, терминальное~--- из c, b, a, u, 1, 2, 3. Фитнес отражает количество корректно вычисленных элементов выборки. Алгоритмом ПЭГ были успешно обнаружены формулы, со 100\% точностью описывающие искомые правила.

В работе~\cite{Banks:gecco05lbp} рассматривается применение ПЭГ в задаче поиска захоронённых боеприпасов путём анализа показаний электромагнитных сенсоров. Испрользовались данные эксперимента JPG-IV, проводимом армией США на полигоне Джефферсон. Обучение и тестирование системы проводилось по данным, полученным в результате эксперимента JPG-IV, проводимом армией США на полигоне Джефферсон. Полученная система показала эффективность, превосходящую большинство описанных в работе решений.

В работе~\cite{visoiu2011deriving} показано использование ПЭГ для построения рыночной модели по биржевым данным.

Задача автоматического реферирования (Automatic text summarization) исследуется на протяжении десятков лет. Большинство подходов к её решению сводится к комбинированию статистических методов и лингвистического анализа. Значительно реже предлагается воспользоваться машинным обучением. В работе~\cite{ZhuliXie:2004:COLING} предлагается применить для обучения эволюционный алгоритм ПЭГ. В реализованной системе каждое предложение $s$ представлено пятью нормализованными характеристиками:
\begin{enumerate}
  \item Положение абзаца $P = Y / M$, где $M$~--- общее количество абзацев, $Y$~--- индекс абзаца, в котором находится предложение.
  \item Положение предложения $S = X / N$, где $N$~--- общее количество предложений в абзаце, $X$~--- индекс предложения.
  \item Длина предложения $L$: $$L = \frac{1-e^{-\alpha}}{1+e^{-\alpha}}, \alpha=\frac{l(s) - \mu(l(s))}{std(l(s))},$$ где $l(s)$~--- количество слов в предложении, $u(l(s))$~--- среднее значение этой величины, $std(l(s))$~--- её стандартное отклонение.
  \item Заголовок $H=1$, если предложение является названием, заголовком либо подзаголовоком, иначе $H=0$.
  \item Част\'{о}ты слов $F$:$$F=\frac{1-e^{-\alpha}}{1+e^{-\alpha}}, \alpha=\frac{CW(s) - \mu(CW(s))}{std(CW(s))},$$ $$CW(s)=-\sum\limits_{i=1}^k\log{[Freq(w_i)]}, w_i\in{s},$$ где $Freq(w_i)$~--- количество употреблений слова $w_i$ в статье.
\end{enumerate}

Основой системы является предположение о том, что к каждому определённому типу документов применим один и тот же механизм реферирования, потому для построения рефератов набора документов достаточно найти один алгоритм. Функция фитнеса такой системы~--- количественно оценённая похожесть полученного реферата с составленным человеком, путём векторизации текстов и их последующего скалярного произведения. В качестве реферата возвращаются $N$ предложений с максимальной оценкой, выданной системой, $N$ задаётся в зависимости от желаемой длины реферата.

За недоступностью других систем автоматического реферирования для сравения с ними эффективности полученной системы были разработаны три простых метода:
\begin{itemize}
  \item Составление реферата из первых предложений первых пяти абзацев.
  \item Случайным образов выбираются пять предложений из полного текста.
  \item Из случайном образом выбранных пяти абзацев отбираются первые предложения.
\end{itemize}

Полученная система, основанная на ПЭГ, на 58--160\% превосходит приведённые простые методы. Тем не менее, приемлемая эффективность получена не была. Основная причина~--- субъёктивность оценки путём сравнения с <<эталонными>> рефератами, написанными человеком, т.к. стили написания могут различаться и даже противоречить друг другу.

В работе~\cite{journals/jucs/AbrahamG06} предлагаются три варианта применения ГП для мониторинга электронных цепей и систем в реальном времени. Надёжность цепи оценивается на основе показаний сенсоров восприимчивости цепи на электромагнитные воздействия, которые затем подаются на обработку эволюционному алгоритму для вынесения конечного решения о состоянии цепи.

Вероятность отказа компонента зависит как от стрессора (электрического, механического либо другого физического воздействия на компонент на протяжении всего срока его работы), так и чувствительности компонента к данному стрессору. Кроме того, большинство компонентов имеют несколько механизмов возникновения отказа по причинам: перегрева (стрессоры: рассеивание тепла, высокая температура окружающей среды, термическое сопротивление, теплоёмкость), различных видов пробоев (изменение проводимости компонента из-за примесей в материалах либо вследствие изменения температуры, воздействие электромагнитного поля), коррозии, утечки тока и пр.

Существуют два способа получения набора стрессоров цепей: компьютерная симуляция моделей компонентов и цепей, либо анализ экспериментальных измерений и построение моделей <<стрессор-восприимчивость>>. Второй способ позволяет применить алгоритм ПЭГ к решению поставленной задачи.

Проведено сравнение эффективности решения поставленной задачи следующими методами: линейное генетическое программирование (LGP~--- Linear Genetic Programming), мультигенное ГП (MEP~--- Multy Expression Programming), ПЭГ, ИНС, а также деревья классификации и регрессии (CART~--- Classification and Regression Trees).

Линейное ГП является ответвлением классического ГП, в котором программа представляется не древовидной, и линейной структурой, наиболее похожей по форме на построчный ассемблерный листинг, каждая строка которого~--- один оператор. Сходство подтверждает также использование временных переменных-<<регистров>> для передачи данных между строками программы.

В мультигенном ГП каждая хромосома состоит из заданного количества генов - элементов, представляющих либо терминал, либо функцию и индексы аргументов. Аргументы для исключения циклических зависимостей ограничены лишь ссылками на предыдущие гены хромосомы. Таким образом, результатом декодирования хромосомы является синтаксическое дерево. Некоторые элементы генома могут пропущены при построении дерева, если на них в конкретно взятом образце не было ссылок, а потому не использованы.

На вход алгоритма подаются значения напряжения и силы тока, проходящего через цепь, выход (моделируемая величина) - температура узлов цепи и сила токов утечки. Достоинством ГП являето то, что полученная программа легко может быть реализована аппаратно. Модели, полученные при помощи ГП отличаются высокой точностью решения задачи и лёгкостью в использовании.

В работе~\cite{conf/ijcnn/ValdesB06} представляется метод конструирования пространств виртуальной реальности для визуального анализа данных при помощи многоцелевой оптимизации средствами ПЭГ. 

Поставленная задача состоит в составлении компактного аналитического представления отображения многомерного пространства на двух- или трёхмерное. Данный подход был применён к практической задаче обнаружения подземных полостей, как правило, заполненных водой. Карты подземных течений местности зачастую составлены неточно, т.к. пустоты в большинстве случаев не имеют выхода на поверхность, потому для их обнаружения требуется исследование местности геофизическими методами. В их число входят: измерения электрического потенциала поверхности почвы в сухое и дождливое времена года, значение вертикальной компоненты электромагнитного поля в низкочастотном спектре, показания интенсивности гамма-излучения, топография местности (высотная карта). Результатом измерения каждой физической характеристики является поверхность - набор значений, соответствующих точкам местности.

После обработки алгоритмом многоцелевой оптимизации NSGA-II эти данные подаются на вход ПЭГ. Критерием успешности модели была выбрана минимизация среднеквадратичного отклонения от ожидаемых величин.
Полученные в результате эксперимента формулы с высокой точностью описали распределение моделируемых величин, а задействованные переменные позволили выявить величины, анализируя которые можно диагностировать циркулирующие аномалии.

В работе~\cite{Kwasnicka:2006:FIMCSIT} рассматривается задача автоматизации компьютерной анимации трёхмерных моделей. Под анимацией понимается описание движения моделей персонажей, представленных в виде подвижно сочленённых недеформируемых блоков. Был разработан новый метод создания компьютерной анимации трёхмерных моделей с помощью ПЭГ. Обычно в этих целях использются следующие методы:
\begin{itemize}
  \item Метод ключевых кадров~--- задание главным аниматором позиций моделей в ключевых кадрах и последующая отрисовка деталей движения в промежуточных кадрах менее искусными аниматорами либо автоматически.
  \item Прямая, обратная кинематики~--- преобразование скелетной анимации, заданной в координатах суставов скелета с учётом их свойств, в декартову систему координат сцены.
  \item Физическая симуляция: динамика. Применяется для достижения максимальной реалистичности, учёта сил и моментов, гравитации и инерции.
  \item Поведенческие техники~--- моделируемая система представляется в виде частиц, управляемых набором относительно простых правил, позволяющих создать сложное движение.
  \item Оптимизационные методы:
  \begin{itemize}
    \item Минимизация энергии.
    \item Задание пространственно-временных условий и ограничений.
    \item Эволюционные алгоритмы.
  \end{itemize}
\end{itemize}

Большинство описанных методов, в особенности метод ключевых кадров, прямая и обратная кинематики, требуют высокой квалификации аниматора для установления соответствия движений природе персонажа, а деталей визуализации~--- физическим законам. Потому даже частичная автоматизация процесса позволит существенно сократить количество ручной работы.

Основные правила создания <<идеальной>> анимации были разработаны в начале XX века, в частности усилиями студии Диснея. Всего таких принципов 12:
\begin{itemize}
  \item Темп~--- скорость движения объекта должна соответствовать причине его движения, давать представление о его массе, упругости, пр.
  \item Плавность движения~--- движение не может начинаться и заканчиваться рывком.
  \item Плавность траектории~--- следует избегать изломов в направлении движения, заменять их изгибами. При моделировании живых персонажей соединения костей скелета описываются вращательными движениями суставов.
  \item Обусловленность~--- каждое действие анимируемого персонажа обычно состоит из трёх фаз: подготовка, само движение и его окончание. Обусловленность относится к подготовке к движению: перед прыжком персонаж, как правило, сгибает ноги в коленях, что связано с физикой прыжка.
  \item Гиперболизация~--- усиление эффекта движеня для концентрации на нём внимания зрителя.
  \item Деформация (сжатие и разжатие) отражает воздействие на объект силы, приводящей его в движение.
  \item Второстепенные движения~--- свойственные живым организмам разнообразные движения, такие как дыхание, сложная механика бега, включающая слежение взглядом за окружением.
  \item Совмещение движений~--- плавный переход окончания первого движения в начало второго.
  \item Покадровая и позиционная анимация~--- два взаимоисключающих подхода к созданию анимации. Первый предполагает последовательную ручную отрисовку от начального кадра до конечного. Второй~--- создание ключевых кадров с автоматическим построением промежуточных на основе свойств сочленений.
  \item Постановка~--- ясное и понятное представление основной идеи анимации.
  \item Обращение к зрителю~--- привлекательность происходящего, сценария и персонажей.
  \item Реалистичность.
\end{itemize}

В предложенном решении на основе ПЭГ движения персонажей (моделей) контролируются компьютерными программами, создаваемыми в ходе эволюции. Участие аниматора заключается в количественном оценивании полученных вариантов анимации, эти оценки служат критерием успешности при построении очередной популяции вариантов. Программа-контроллер управляет подвижными частями модели учитывая при этом заданные ограничения, такие как силу мышц персонажа, его массу. Таким образом можно косвенно управлять поведением модели: увеличивая параметр, отвечающий за массу персонажа, ему становится доступным только ходьба, но не прыжки.

При оценке учитывается расстояние между требуемым и полученным местоположениями модели и <<стиль>> движения (персонаж может достичь цели ползком, прыжком, бегом). Стиль включает в себя следующие категории: безопасность~--- отсутствие столкновений с другими объектами; время~--- назначение штрафа за слишком медленное или слишком быстрое передвижение и рывки; достижение равновесия~--- степень нейтральности конечного положение (персонаж стоит, а не лежит); прочее~--- в зависимости от типа движения.

Для демонстрации работоспособности предложенной автоматизации процесса анимации были получены следующие сюжеты: приземление персонажа, состоящего из пяти элементов, в заданную область тремя прижками; передвижение паукообразного существа в заданную точку. Была достигнута высокая реалистичность поведения персонажей, на что в случае ручной отрисовки потребовалось бы длительное время.

Другая задача компьютерной анимации~-- создание трёхмерных моделей деревьев и других растений для дальнейшей визуализации в составе сцены. Обычно в этих целях используются процедурные техники, состоящие из инструкций по построению дерева, но не описывающие явным образом его геометрию. Представление в виде программы позволяет применить к поставленной задаче эволюционный подход. Качество полученных деревьев, как и в предыдущей работе, предлагается оценивать аниматору на основе их эстетического вида, эти оценки используются для оператора отбора особей~\cite{conf/afrigraph/VenterH07}.

Одной из лучших процедурных техник для построения моделей растений являются L-системы (L-Systems~--- Lindnenmayer Systems~--- системы Линденмайера), вернее, их детерминированная контексто-независимая разновидность, состоящая из алфавита символов, аксиомы и набора правил, заменяющих символ-предшественник строкой-последователем. Вариант решения при этом случае описывается аксиомой (начальной строкой) и набором правил. Для кодирования таких решений используются мультигенные хромосомы. Первый ген кодирует аксиому, остальные - правила.

Несмотря на то, что оценка аниматором зависит не только от характеристик оцениваемого дерева, но и от характеристик других дереьев популяции и состояния самого аниматора, это не мешает работе алгоритма, потому как имеет значение лишь отношение качества оцениваемого дерева к качеству остальных деревьев популяции. Так, если одна особь по решению оператора в два раза лучше другой~--- её фитнес также будет в два раза выше, как и вероятность отбора.

Для ускорения работы алгоритма количество рекурсивных итераций было сокращено~--- полученные деревья при этом достаточны для оценки. При дальнейшем использовании полученной L-системы для построения финального дерева количество рекурсивных итераций устанавливается максимальным.

Полученная система позволяет автоматизированно получить качественную модель растения, не прибегая к трудоёмкому ручному построению.

Сети беспроводных сенсоров (Wireless Sensor Network ~--- WSN) широко применяются для обнаружения, определения местоположения и отслеживания перемещений движущихся объектов. В помещении определённым либо произвольным образом устанавливается множество сенсоров, определяющих и запоминающих местоположение цели в моменты времени $t_0, t_1, ..., t_i$. Поставленной задачей является расчёт траектории цели и прогнозирование её положения в моменты времени $t_{i+1}, t_{i+2}, ...$ При использовании сети сенсоров в больших помещениях стоит задача балансирования между энергопотреблением аппаратных сенсоров сети (временем включения и выключения) и точностью распознавания движения.

Одно из возможных решений задачи заключается в следующем. Состояние цели включает в себя положение, направление и скорость. На каждом такте работы системы сенсоры, находящиеся рядом с целью, кластеризуют показания, а центры полученных кластеров подаются на вход фильтра Калмана. Реализация такого подхода не представляет сложностей в централизованной среде, однако затруднена в распределённых сетях, таких как WSN, состоящих из компактных устройств с ограниченными вычислительными возможностями.

В работе~\cite{Dai:2009:ETA:1726588.1727798} была предложена система, лишённая указанного недостатка. Её конструирование состоит из трёх этапов.

На первом этапе для распознавания движения был разработан алгоритм распределённого ПЭГ, выполняющийся на нескольких взаимодействующих сенсорах. Движение каждой цели описывается формулой, полученной на основании $h$ последних координат цели~--- т.н. алгоритм скользящего окна.

На втором этапе алгоритм скользящего окна настраивается на быстрое обучение алгоритма распределённого ПЭГ: при большом расхождении предсказанного движения с наблюдаемым предыдущие накопленные данные о цели отбрасываются, определяя таким образом изменения характера движения цели и позволяя алгоритму ПЭГ построить новую формулу движения цели.

Третий этап представляет собой моделирование работы системы.

Алгоритм распределённого ПЭГ:
\begin{enumerate}
  \item Активация сенсоров планировщиком либо при ожидаемом поступлении цели в область видимости.
  \item Запуск узлами алгоритма ПЭГ при наличии в области видимости целей для поиска их траекторий. Каждый узел-сенсор, построивший формулу движения цели, отправляет её всем соседним узлам.
  \item Узлы, получившие рассчитанную траекторию цели от другого узла, прерывают запущенный алгоритм ПЭГ, и начинают использовать полученную формулу движения для дальнейшего слежения за целью.
\end{enumerate}

Критерии останова алгоритма ПЭГ: произведено максимальное количество поколений, превышено время выполнения, получение траектории от другого узла, успешный расчёт траектории.

В ходе моделирования системы была показана 25\% экономия энергопотребления по сравнению с системой на основе фильтра Калмана и системой на основе алгоритма максимального сближения (Enhanced Closest Point of Approach~--- ECPA).


\subsection{Системы дифференциальных уравнений}

Одна из задач идентификации системы состоит в построении модели, представленной обыкновенным дифференциальным уравнением (максимальный порядок задаётся пользователем), выведенным из поданных данных~\cite{conf/iscis/FloresG05}.

При таком подходе синтаксическое дерево представляет собой правую часть дифференциального уравнения, записаного в нормальной форме:
$$
y^{(n)} = f(t, y, y', y'', ..., y^{(n-1)}).
$$

Терминальное множество дополняется производными функциями, как показано на рисунке:

\begin{figure} [h]
  \center
  \begin{tikzpicture}[level distance=1cm,
    level 1/.style={sibling distance=3cm},
    level 2/.style={sibling distance=2cm},
    level 3/.style={sibling distance=1cm}]
    \tikzstyle{every node}=[-,thick]
    \node { $+$ }
      child[->] { node { $\times$ }
        child { node { $7$ } }
        child { node { $\dfrac{dx}{dt}$ } }
      }
      child { node { $+$ }
        child { node { $12$ } }
        child { node { $\times$ }
          child { node { $10$ } }
          child { node { $x$ } }
        }
      }
    ;
  \end{tikzpicture}
  \caption{Пример дерева с терминалами-производными}
  \label{img:example_tree_with_differentials}
\end{figure}

Моделируемая функция рассчитывается из полученной синтаксической конструкции методом Рунге-Кутты. Проверка эффективности данного подхода проводилась на задачах моделирования линейного маятника, нелинейного маятника с трением, колебательной системы <<пружина-масса>> и линейной электрической цепи. Во всех четырёх экспериментах был достигнут коэффициент корреляции $R^2 > 0.99$. Данный подход обладает следующими преимуществами: возможность моделирования системы, представленной в виде ОДУ, автоматическое определение порядка системы, возможность получения линейной модели.

Аналогичные исследования систем дифференциальных уравнений проведены в работе~\cite{conf/acsc/ZarnegarVS09}. При изучении генома, например, при сравнении развития раковых и нормальных клеток, широко используются ДНК-микрочипы, упрощающие исследование экспрессии генов, их функций. Последовательности ДНК при таком подходе закрепляются на твёрдой основе ДНК-микрочипа, формируя двумерный микромассив. Информация об изменениях экспрессии генов в микромассиве с течением времени записывается в виде генной сети. Генная сеть может быть представления либо вероятностными методами, такими как байесовские сети, либо детерминированными, например, в виде систем дифференциальных уравнений~--- наиболее широко распространённый метод.

Системы дифференциальных уравнений, полученные из набора данных с помощью ПЭГ, численно решаются методом Монте-Карло.

Несмотря на способность ПЭГ обнаруживать структуру решений, метод в чистом виде не эффективен в оптимизации параметров-констант. Поэтому для этих целей к каждой особи популяции был применён метод наименьших квадратов, что существенно улучшило точность результатов. Наложение гауссового шума на данные лишь незначительно влияет на точность алгоритма.


\subsection{Построение искусственных нейронных сетей}

Применение ИНС чрезвычайно популярно при решении задач классификации, машинного обучения, идентификации систем и пр. Однако от оператора требуется, как правило, предварительно указать топологию сети, перед дальнейшим обучением. Структура сети определяется набором таких параметров как количество слоёв, количество нейронов в каждом из них, связи между нейронами и соотвествующие функции активации. Тестирование каждой конфигурации параметров требует отдельного запуска процедуры обучения и тестирования сети.

Для автоматического выбора наиболее эффективной топологии сети возможно использование эволюционных алгоритмов. Применимость генетических алгоритмов к указанной задаче показана в работе~\cite{stanley1996efficient}. Процедура кодирования генотипа и фенотипа ПЭГ может быть легко применена к древовидной структуре ИНС, аналогичной структуре синтаксического дерева. Таким образом возможно закодировать ИНС целиком в геноме особи. В работах~\cite{ferreira:2004:wsc9, Ferreira:wsc9, li2005new, conf/gecco/JohnsS09} в качестве функционального множества используются элементы, представляющие нейроны разных типов. Хромосома дополняется массивом весов связей и массивом пороговых значений функций активации.

Таким образом, в ходе эволюции популяция может содержать особи с разной структурой, следовательно, становится возможным как автоматический выбор структуры, так и настройка весов (обучение).

В качестве примеров использования таких систем приводится построение правил классификации.


\subsection{Кластеризация}

Большинство существующих методов кластеризации требуют задания определённых параметров, например, количество кластеров либо их радиус. Задание этих параметров требует априорных знаний о данных и на практике затруднительно. Кроме того, наиболее популярный метод кластеризации~--- К-средних~--- к тому же чувствителен к начальным значениям центров кластеров, вследствие чего зачастую сходится к локальному оптимуму. Для устранения данных недостатков в рабоет~\cite{Chen:2007:CWP:1304603.1305730} был предложен метод на основе алгоритма ПЭГ.

Функциональный набор содержит всего два элемента: $\bigcup$~--- оператор сегментации, объединяющий две точки в множество, и $\bigcap$~--- оператор аггрегации, возвращающий центроид - точку с координатами, равными полусуммам координат точек-аргументов. В случае, когда аргумент~--- множество, а не точка, участвуют все входящие в него точки. Функция финтеса~--- величина, обратная средней сумме квадратов расстояний точек до центров их кластеров. По окончании работы алгоритма близкие кластеры объединяются в один.

Проведённые эксперименты с синтетическими данными показали высокую (96\%) способность предложенного метода обнаруживать адекватный набор кластеров. В то же время эффективность алгоритма снижается при увеличении размерности данных. Другой существенный недостаток~--- чувствительность к шуму.


\subsection{Параметрическая регрессия}

Параметрическая символьная регрессия~--- задача поиска математических выражений, отличается отличие от вышеописанных случаев видом искомых выражений: вместо записи $y(x)$ производится поиск системы вида $y(t), x(t)$, где $t$ - параметр.

Перед алгоримом ПЭГ была поставлена~\cite{banks:2004:msa:erban} задача поиска брахистохроны~--- кривой скорейшего запуска, описываемой следующими уравнениями:
\begin{eqnarray}
X & = a\times(\theta - \sin{\theta})\\
Y & = a\times(1 - \cos{\theta})
\end{eqnarray}

Параметрическое описание выбрано по причине невозможности описания в виде $Y=Y(X)$ (хотя существует обратное решение в виде $X=X(Y)$).

Несмотря на многократные запуски алгоритма с различными параметрами, точные выражения искомой кривой обнаружены не были, хотя и было найдено множество решений, очень близких к оптимальному. На основании этого сделан вывод о том, что ГП не способно открыть уравнения брахистохроны, потому были исследованы возможные причины этого.

Множество кривых, с приемлемой точностью описывающих заданный набор данных, является множеством локальных оптимумов. Пространство решений и их присобленности можно визуализировать в виде зубной щётки, каждая щетинка которой практически неотличима от другой, и представляет собой локальный оптимум. Их обилие делает крайне сложным различие среди них определённого решения. Данная аналогия позволяет понять, почему истинные параметрические уравнения не будут обнаружены алгоритмом даже после сколь угодно большого числа часов работы.

Проведённным в работе~\cite{chen2012some} математическим анализом исходного ПЭГ, представленного в виде марковской модели, было доказано, что ПЭГ не является алгоритмом оптимизации, гарантированно сходящимся к глобальному оптимуму даже за бесконечное время. Однако вдалее было показано, что таким свойством обладает многократный запуск ПЭГ с механизмом сквозного простого элитизма (копированием лучшей особи предыдущего запуска в начальную популяцию следующего).


\subsection{Прогнозирование временных рядов}

Прогнозирование временн\'{ы}х рядов~--- одна из важных задач глубинного анализа данных. Традиционно при её решении применяется метод скользящего окна, и применимо к ПЭГ данный метод выглядит следующим образом: задаются набор данных (точек), полученных через фиксированные промежутки времени $\Delta t$ $X = (x_0, x_1, ... x_n)$, длина истории $h$, требуется найти формулу $f$, описывающую для любого $m, (n - h + 1 \leq m \leq n)$ прогнозируемое значение $\hat{x_m}$:
$$
\hat{x_m} = f(x_{m-h}, x_{m-h+1}, ..., x_{m-2}, x_{m-1}), (h < m \leq n)
$$

Оценкой формулы может быть абсолютная погрешность $|\hat{x_m} - x_m|$, относительная погрешность $\frac{|\hat{x_m} - x_m|}{x_m}$ либо другие измерения, например коэффициент корреляции.

Процедура может быть описана как скользящее по временн\'{о}й шкале окно шириной $h + 1$, формула $f$ прогнозирует будущие значения на основе $h$ предыдущих. Однако несмотря на простоту и скорость, методу скользящего окна не хватает семантической силы, возможностей раскрывать ключевые свойства, скрытые во временн\'{ы}х рядах. Для преодоления этого недостатка предложен~\cite{conf/waim/ZuoTLYC04, conf/cilamce/Lopes04} метод обыкновенных дифференциальных уравнений, состоящий из двух этапов: построение дифференциальных уравнений высших порядков из обучающей выборки данных, и прогнозирование с использованием полученных уравнений и начальных условий. Таким образом, в терминальное множество кроме символа $x(t)$ добавляются символы $x'(t), x''(t), ...$. Полученные дифференциальные уравнения затем вычисляются методом Рунге-Кутты.

Для устранения влияния зашумлённости данных на поиск начальных условий высокочастотная составляющая данных удаляется с применением преобразований Фурье.

Система, реализующая прогнозирование временных рядом, описана в работе~\cite{viento}, посвященной предсказанию ветра для дальнейшего использования этих данных при планировании размещения ветряных электростанций при помощи GEP, и сравнению полученных результатов с аналогичными полученными с использованием интегрированной модели авторегрессии со скользящим средним (ARIMA~--- autoregressive integrated moving average). Эксперимент показал, что модель, полученная в результате эволюционного процесса ПЭГ, имеет значительно меньшую погрешность, чем полученная статистической процедурой ARIMA. Кроме того, в работе~\cite{buarbulescu2009time} представлен анализ применения к данной задаче адаптивной модификации ПЭГ~--- AdaGEP.

\clearpage

\addcontentsline{toc}{chapter}{\bibname} % Добавляем список литературы в оглавление
\bibliography{related_work}              % Подключаем BibTeX-базы

\end{document}
